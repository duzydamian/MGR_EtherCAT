%\documentclass[a4paper,12pt]{report}
\documentclass[a4paper,12pt]{article}
%\documentclass[a4paper,12pt]{book}
\usepackage{polski}
\usepackage[polish]{babel}
\usepackage[utf8]{inputenc}
\usepackage[top=2.5cm, bottom=2.5cm, left=3cm, right=2.5cm]{geometry}
\usepackage{graphicx}
\usepackage{setspace}
\usepackage{ifthen}
%\usepackage[utf8]{inputenc}
\usepackage{a4wide}
\usepackage{fullpage}
\usepackage{verbatim}
\usepackage[usenames,dvipsnames]{color}
\usepackage{hyperref}
\usepackage{subfig}
\usepackage{listings}
\usepackage{mdwlist}
\usepackage{titlesec}
\usepackage{lipsum}

\hypersetup{
	bookmarks=true,
%	pdftitle={INŻ; Projekt – Sprawozdanie},
	pdftitle={Abstrakt do },
	pdfauthor={Damian Karbowiak},
	pdfsubject={System sterowania i wizualizacji pracy robota 3D},
	pdfkeywords={INŻ sprawozdanie Politechnika Śląska System sterowania i wizualizacji pracy robota 3D raport końcowy projekt magisterski},
	colorlinks=true,
	linkcolor=Blue,
	urlcolor=Blue
	}

\let\subsubsubsection\paragraph
%\setcounter{secnumdepth}{6} % subsubparagraph ???
% that is, subsubsubsubsubsection :-)
\setcounter{secnumdepth}{4}

\newcommand{\tytul}{System sterowania i wizualizacji pracy robota 3D.}
\newcommand{\data}{czerwiec 2013}
\newcommand{\promotor}{dr~inż. Jacek Stój}
\newcommand{\autor}{Damian Karbowiak}
%\newcommand{\konsultant}{mgr inż. Wojciech Domagała }
\newcommand{\konsultant}{}

\titlespacing{\section}{1cm}{*4}{*1.5}
\titlespacing{\subsection}{1cm}{*4}{*1.5}
\titlespacing{\subsubsection}{1cm}{*4}{*1.5}

\begin{document}
\lstset{backgroundcolor=\color{white}, boxpos=c, captionpos=b}
\lstset{numbers=left, stepnumber=1, numbersep=10pt, frame=single}
\lstset{frameround=tttt}
\renewcommand{\lstlistlistingname}{\vspace*{-13mm}}
\renewcommand{\listfigurename}{\vspace*{-13mm}}
%\renewcommand*\l@figure[2]{\indent}
\renewcommand{\listtablename}{\vspace*{-13mm}}
\renewcommand*{\refname}{\vspace*{-13mm}}
\renewcommand{\lstlistingname}{Kod źródłowy} 

Głównym celem pracy było oprogramowanie sterownika PLC Siemens S7-300 do sterowania pracą modelu Robota~3D firmy Fischertechnik oraz stworzenie graficznej prezentacji pracy tego modelu. Dodatkowym elementem zrealizowanej pracy inżynierskiej jest element pozainformatyczny, czyli zbudowanie modelu magazynu z~którym robot będzie współpracował. Magazyn po zakończeniu projektu posłuży jako pomoc dydaktyczna do~ćwiczeń laboratoryjnych.

Model robota posiada 4~silniki umożliwiajace mu poruszania się w 3~osiach oraz poruszanie chwytakiem. W ramach pracy powstał kod sterowania pojedynczym silnikiem, który po gruntownym testowaniu podlegał dalszemu rozwojowi. Wszystkie silniki poprzez odpowiednie wysterowanie przez sterownik mogą pracować w trybach: automatycznym, ręcznym z dołączonego do sterownika zadajnika sygnałów dyskretnych oraz ręcznym z poziomu stworzonej wizualizacji. Tak przygotowane oprogramowanie zostało wykorzystane do stworzenia kodu umożliwiającego obsługę zbudowanego przez autora modelu magazynu z wykorzystaniem kolejki zadań do realizacji. Każde zadanie składa się z indeksu komórki w magazynie oraz zmiennej mówiącej o ruchu chwytaka (zabranie lub upuszczenie przedmiotu). Wszystkie operacje wykonywane na magazynie są wykonywane z wykorzystaniem zaimplementowanej kolejki FIFO od momentu kliknięcia przycisku w wizualizacji aż do jej opróżnienia.

Podczas realizacji zostały rozwiązane problemy, z których najważniejszym zdaniem autora była bezwładność silników. W jej wyniku silnik po wyłączeniu dalej poruszał się przez nieokreślony czas aż do samoistnego zatrzymania. Wyeliminowanie negatywnych efektów tego zjawiska pozwoliło stworzyć w pełni funkcjonalne oprogramowanie spełniające stawiane przed nim zadania.
W~procesie tworzenia oprogramowania autor poruszył i wykorzystał do realizacji wiele zagadnień typowych dla sterowników firmy Siemens. Przykładowo bloki wywoływane podczas startu sterownika (OB100 / OB101 / OB102) z~czego na sterowniku S7-300 dostępnym w~laboratorium dostępny jest tylko blok OB100, czyli tzw. gorący restart (ang. \emph{warm restart}) wywoływany przy każdym przejściu ze~stanu STOP do RUN/RUN-P oraz podczas uruchomienia po zaniku zasilania. Kolejne charakterystyczne bloki to~te~wywoływane jako cykliczne przerwania (OB30~do~OB38). Tutaj niestety również występuje ograniczenie i~dostępny jest jedynie blok OB35, wywoływany domyślnie co 100~ms.

\end{document}
