%\documentclass[a4paper,12pt]{report}
\documentclass[a4paper,12pt]{article}
%\documentclass[a4paper,12pt]{book}
\usepackage{polski}
\usepackage[polish]{babel}
\usepackage[utf8]{inputenc}
\usepackage[top=2.5cm, bottom=2.5cm, left=3cm, right=2.5cm]{geometry}
\usepackage{graphicx}
\usepackage{setspace}
\usepackage{ifthen}
\usepackage{a4wide}
\usepackage{fullpage}
\usepackage{verbatim}
\usepackage[usenames,dvipsnames]{color}
\usepackage{hyperref}
\usepackage{subfig}
\usepackage{listings}
\usepackage{mdwlist}
\usepackage{titlesec}
\usepackage{lipsum}
\usepackage{multirow}
\usepackage{enumitem}

\usepackage{pgfplots}
\usepackage{tikz}
\usetikzlibrary{automata,positioning,shapes,shadows,arrows,backgrounds,trees,fit,calc,decorations.pathreplacing}

\hypersetup{
	bookmarks=true,
	pdftitle={Projekt magisterski: EtherCAT},
	pdfauthor={Damian Karbowiak},
	pdfsubject={EtherCAT},
	pdfkeywords={INŻ sprawozdanie Politechnika Śląska EtherCAT raport końcowy projekt magisterski},
	colorlinks=true,
	linkcolor=black,
	citecolor=black,
	urlcolor=black
	}
	
\let\subsubsubsection\paragraph
%\setcounter{secnumdepth}{6} % subsubparagraph ???
% that is, subsubsubsubsubsection :-)
\setcounter{secnumdepth}{4}

\newcommand{\tytul}{Projekt i realizacja stanowiska laboratoryjnego do badania zależności czasowych w sieci EtherCAT}
\newcommand{\data}{\today}
\newcommand{\promotor}{dr~inż. Jacek Stój}
\newcommand{\autor}{Damian Karbowiak}
\newcommand{\konsultant}{}

\titlespacing{\section}{1cm}{*4}{*1.5}
\titlespacing{\subsection}{1cm}{*4}{*1.5}
\titlespacing{\subsubsection}{1cm}{*4}{*1.5}

\begin{document}

\tikzstyle{abstract}=[rectangle, draw=black, rounded corners, fill=blue!40, drop shadow,
        text centered, anchor=north, text=white, text width=4cm]
\tikzstyle{comment}=[rectangle, draw=black, rounded corners, fill=green, drop shadow,
        text centered, anchor=north, text=white, text width=4cm]
\tikzstyle{myarrow}=[->, >=open triangle 90, thick]
\tikzstyle{line}=[-, thick]

\lstset{backgroundcolor=\color{white}, boxpos=c, captionpos=b}
\lstset{numbers=left, stepnumber=1, numbersep=10pt, frame=single}
\lstset{frameround=tttt}
\renewcommand{\lstlistlistingname}{\vspace*{-13mm}}
\renewcommand{\listfigurename}{\vspace*{-13mm}}
%\renewcommand*\l@figure[2]{\indent}
\renewcommand{\listtablename}{\vspace*{-13mm}}
\renewcommand*{\refname}{\vspace*{-13mm}}
\renewcommand{\lstlistingname}{Kod źródłowy} 

\begin{titlepage}
  \begin{center}
    \vspace*{5.1mm}
    \includegraphics[width=3.41cm]{images/logo}
    \vskip 1.58cm
    \scalebox{0.93}{\textbf{\uppercase{\textsf{\Large P{\large olitechnika} Ś{\large lĄska}}}}}
    \vskip 6.1mm
    \scalebox{0.93}{\textbf{\uppercase{\textsf{\Large W{\large ydziaŁ} A{\large utomatyki}, E{\large lektroniki} {\large i} I{\large nformatyki}}}}}
    \vskip 4.6mm
    %\scalebox{0.93}{\textbf{\uppercase{\textsf{\Large K{\large ierunek} I{\large nformatyka}}}}}
    {\textbf{\uppercase{\textsf{\large K{\normalsize ierunek} I{\normalsize nformatyka}}}}}
    \vskip 2.3cm
    \textsf{\LARGE Praca dyplomowa magisterska}
    \vskip 1.8cm
    \begin{onehalfspace}
      \textsf{\large \tytul}
    \end{onehalfspace}
  \end{center}
  \vfill

  \noindent\textsf{\large Autor: \autor}
  \vskip 3mm
  \noindent\textsf{\large Kierujący pracą: \promotor}
  \ifthenelse{\equal{\konsultant}{}}{
    \vskip 36mm
  }{
    \vskip 3mm
    \noindent\textsf{\large Konsultant: \konsultant}
    \vskip 27.7mm
  }
  \noindent\textsf{Gliwice, \data}
  \vspace*{2mm}
\end{titlepage}
 %

\tableofcontents
\addtocontents{toc}{\protect\vspace*{.05\baselineskip}}
\clearpage

\section{Wstęp}
%\subsection{Geneza}
Tematem projektu, którego dotyczy ta praca jest: „Projekt i~realizacja stanowiska laboratoryjnego do badania zależności czasowych w~sieci EtherCAT". Zagadnienia związane z~tworzeniem oprogramowania dla sterowników przemysłowych są dla autora niezwykle interesujące, a zrealizowany projekt miał na~celu dalsze pogłębienie jego wiedzy z tego zakresu. Wyboru tego konkretnego tematu autor dokonał, ponieważ protokół EtherCAT jest jeszcze nowością i według wielu źródeł stanowi przyszłość branży informatyki przemysłowej \cite{art1, art2}, a~praca nad tym tematem wydaje się być pomocna i~wartościowa w przyszłej pracy zawodowej lub na~ewentualnym dalszym etapie kształcenia.

\subsection{Stanowisko laboratoryjne}
Na potrzeby realizacji projektu wykorzystano dwa różne istniejące stanowiska laboratoryjne, które składały się odpowiednio~z:
\begin{table}[!htb]
\begin{center}
\begin{tabular}{| p{0.51\textwidth} || p{0.51\textwidth} |}\hline
Stanowisko typu CP (Rysunek~\ref{stanowisko:cp}) & Stanowisko typu CX (Rysunek~\ref{stanowisko:cx})  \\\hline
\begin{enumerate}
\item 2~silniki AM3021-0C00-0000,
\item Wyspa EK1100 z~zestawem modułów~IO:
\begin{itemize}
\item Terminal sieci EtherCAT EK1100,
\item 2-kanałowy moduł wyjść analogowych EL4132,
\item 4-kanałowy moduł wejść cyfrowych EL1004,
\item 2 4-kanałowe moduły wyjść cyfrowych EL2004,
\item 2-kanałowy moduł wejść analogowych EL3102,
\end{itemize}
\item Napęd serwomechnizmów AX5203 (2~osiowy napęd),
\item Komputer przemysłowy C6925,
\item Zasilacz.
\end{enumerate}
&
\begin{enumerate}
\item 2~silniki AM3021-0C00-0000,
\item Zestaw modułów IO:
\begin{itemize}
\item 2-kanałowy moduł wyjść analogowych EL4132,
\item 4-kanałowy moduł wejść cyfrowych EL1004,
\item 2 4-kanałowe moduły wyjść cyfrowych EL2004,
\item 2-kanałowy moduł wejść analogowych EL3102,
\end{itemize}
\item Terminal sieci EtherCAT EK1100,
\item Napęd serwomechnizmów AX5203 (2~osiowy napęd),
\item Modułowy komputer przemysłowy CX1020:
\begin{itemize}
\item Interfejs USB/DVI CX1020-N010 ,
\item Ethernet CX1020-N000,
\item CPU CX1020-0113,
\item Zasilacz CPU i magistrali I/O CX1100-0004,
\end{itemize}
\item Zasilacz.
\end{enumerate}
\\\hline                                            
\end{tabular}
\end{center}
\vspace*{-6mm}
  \caption{Dostępne stanowiska laboratoryjne}
	\label{in}
\end{table}

\begin{figure}[htbp]
 \centering
        \tikzstyle{background grid}=[draw, black!50,step=.25cm]
	\begin{tikzpicture}[node distance=1cm, auto, show background grid]
	\tikzset{
    	mynode/.style={rectangle,rounded corners,draw=black, top color=white, bottom color=yellow!50,very thick, inner sep=1em, minimum size=3em, 		text centered, text width=3cm},
	    myarrow/.style={->, >=latex', shorten >=1pt, thick},
	    myline/.style={-, =latex', shorten >=1pt, rounded corners, ultra thick},
	    mylabel/.style={text width=7em, text centered} 
	} 
	\node[mynode] (plc) {Komputer \\ przemysłowy C6925};  
	\node [left=of plc] (laptop) {\includegraphics[width=3cm]{images/laptop}};
	\node[mynode, right=of plc] (zasilacz) {Zasilacz};
	\node[below=3cm of plc] (dummy) {}; 
	\node[mynode, left=of dummy] (io) {Wyspa EK1100 z~zestawem modułów IO};  
	\node[mynode, right=of dummy] (naped) {Napęd \\ serwomechnizmów AX5203};
	\node[below=3cm of naped] (dummy2) {}; 
	\node[mynode, left=of dummy2, text width=4cm](silnik1){Silnik \\ AM3021-0C00-0000};
	\node[mynode, right=of dummy2, text width=4cm](silnik2){Silnik \\ AM3021-0C00-0000};
	
	\draw[myline,blue] (laptop.east) -- ++(-1, 0) -- (plc.west);
	
	\draw[myline,black] (zasilacz.west) -- (plc.east);
	\draw[myline,black] (zasilacz.south) -- (io.north);
	\draw[myline,black] (zasilacz.south) -- (naped.north);	
	
	\draw[myline,yellow] (plc.south) -- (io.north);	
	\draw[myline,yellow] (io.east) -- (naped.west);	

	\draw[myline, green, bend right=10] (naped.south) to (silnik1.north);
	\draw[myline, orange, bend left=10] (naped.south) to (silnik1.north);	
	\draw[myline, green, bend right=10] (naped.south) to (silnik2.north);
	\draw[myline, orange, bend left=10] (naped.south) to (silnik2.north);	
	%\draw[<->, >=latex', shorten >=2pt, shorten <=2pt, bend right=45, thick, dashed] 
    %(io.south) to node[auto, swap] {Competition}(naped.south); 
    
    \draw [yellow, line width=6] (6,-2) -- (6.5,-2); \node at (7.5,-2) {EtherCAT};
    \draw [blue, line width=6] (6,-2.5) -- (6.5,-2.5); \node at (7.5,-2.5) {Ethernet};
    \draw [black, line width=6] (6,-3) -- (6.5,-3); \node at (7.5,-3) {Zasilanie};
    \draw [green, line width=6] (6,-3.5) -- (6.25,-3.5); \draw [orange, line width=6] (6.25,-3.5) -- (6.5,-3.5); 
    \node [text width=2cm] at (7.75,-3.75) {Sterowanie silnikiem};    
\end{tikzpicture} 
\caption{Schemat stanowiska typu CP}
\label{stanowisko:cp}
\end{figure} %
\begin{figure}[htbp]
 \centering
        \tikzstyle{background grid}=[draw, black!50,step=.5cm]
	\begin{tikzpicture}[node distance=1cm, auto]%, show background grid]
	\tikzset{
    	mynode/.style={rectangle,rounded corners,draw=black, top color=white, bottom color=yellow!50,very thick, inner sep=1em, minimum size=3em, 		text centered, text width=3cm},
    	mynodemini/.style={rectangle,rounded corners,draw=black, top color=white, bottom color=yellow!50,very thick, inner sep=.5em, text centered},
	    myarrow/.style={->, >=latex', shorten >=1pt, thick},
	    myline/.style={-, =latex', shorten >=1pt, rounded corners, ultra thick},
	    mylabel/.style={text width=7em, text centered} 
	} 
	\node[mynode] (plc) {Modułowy komputer przemysłowy CX1020};  
	\node[mynode, above=of plc] (plc1) {CX1020-N010 \\ DVI/USB};
	\node[mynode, left=of plc1] (plc2) {CX1020-N000 \\ LAN};
	\node[mynode, right=of plc1] (plc3) {CX1020-0113 \\ CPU};
 	\node [fit=(plc1) (plc2) (plc3)] (fit) {}; 
 	\draw [decorate,decoration={brace,amplitude=10pt}, line width=1pt] (fit.south east) -- (fit.south west);
 		
	\node [left=of plc] (laptop) {\includegraphics[width=3cm]{images/laptop}};
	\node[below=2cm of plc] (dummy) {}; 
	\node[mynode, left=of dummy] (io) {Zestaw  \\modułów~IO}; 
 	\node[mynodemini, left=of io] (io2) {EL1004};
	\node[mynodemini, above=2mm of io2] (io1) {EL4132};  	
 	\node[mynodemini, above=2mm of io1] (io0) {CX1100}; 	 	
 	\node[mynodemini, below=2mm of io2] (io3) {EL2004}; 	 	
 	\node[mynodemini, below=2mm of io3] (io4) {EL2004};
 	\node[mynodemini, below=2mm of io4] (io5) {EL3102};
 	\node [fit=(io0) (io1) (io2) (io3) (io4) (io5)] (fit2) {};  
 	%\draw [decorate, xshift=-20pt,line width=4pt] (fit.south east) -- (fit.north east);
	\draw [decorate,decoration={brace, mirror,amplitude=10pt}, line width=1pt] (fit2.south east) -- (fit2.north east);
 	 	 	 	 	 	
	\node[mynode, below=5mm of io, text width=1.6cm] (ek) {Terminal EK1100};  
	
	\node[mynode, right=of dummy] (naped) {Napęd \\ serwomechnizmów AX5203};
	\node[below=2cm of naped] (dummy2) {}; 
	\node[mynodemini, left=2mm of dummy2, text width=4cm](silnik1){Silnik \\ AM3021-0C00-0000};
	\node[mynodemini, right=2mm of dummy2, text width=4cm](silnik2){Silnik \\ AM3021-0C40-0000};

	\draw[myline,black,dotted] (fit.south) ++(0, -0.4) -- (plc.north);	
	\draw[myline,black,dotted] (fit2.east) ++(0.4, 0) -- (io.west);
	\draw[myline,blue] (laptop.east) -- ++(-1, 0) -- (plc.west);
	
	\draw[myline,purple] (plc.south) -- (io.north);	
	\draw[myline,purple] (io.south) -- (ek.north);
	\draw[myline,purple] (io0.south) -- (io1.north);	
	\draw[myline,purple] (io1.south) -- (io2.north);
	\draw[myline,purple] (io2.south) -- (io3.north);		
	\draw[myline,purple] (io3.south) -- (io4.north);
	\draw[myline,purple] (io4.south) -- (io5.north);
				
	\draw[myline,yellow] (ek.east) -- (naped.west);	

	\draw[myline, green, bend right=10] (naped.south) to (silnik1.north);
	\draw[myline, orange, bend left=10] (naped.south) to (silnik1.north);	
	\draw[myline, green, bend right=10] (naped.south) to (silnik2.north);
	\draw[myline, orange, bend left=10] (naped.south) to (silnik2.north);	
	%\draw[<->, >=latex', shorten >=2pt, shorten <=2pt, bend right=45, thick, dashed] 
    %(io.south) to node[auto, swap] {Competition}(naped.south); 
    
    \draw [purple, line width=6] (6,-1) -- (6.5,-1); \node[text width=2cm] at (7.65,-1.3) {EtherCAT (E-bus)};    
    \draw [yellow, line width=6] (6,-2) -- (6.5,-2); \node[text width=2cm] at (7.65,-2.3) {EtherCAT (skrętka)};
    \draw [blue, line width=6] (6,-3) -- (6.5,-3); \node at (7.5,-3) {Ethernet};
    \draw [black, line width=6] (6,-3.5) -- (6.5,-3.5); \node at (7.5,-3.5) {Zasilanie};
    \draw [green, line width=6] (6,-4) -- (6.25,-4); \draw [orange, line width=6] (6.25,-4) -- (6.5,-4); 
    \node [text width=2cm] at (7.75,-4.25) {Sterowanie silnikiem};    
\end{tikzpicture} 
\caption{Schemat stanowiska typu CX.}
\label{stanowisko:cx}
\end{figure}
 %

\subsubsection{Sterownik PLC}
~~~~~~~~~~~~~~~~~~~~~~~~~~~~~~~~~~~~~~~~~~~~~~~~~~~~~~~~~~~~~~~~~~~~~~~~~~~~~~~~~~~~~~~~~~~~~
soft plc
ahrd plc
opisać zasadę działania plc w beckhoffie
~~~~~~~~~~~~~~~~~~~~~~~~~~~~~~~~~~~~~~~~~~~~~~~~~~~~~~~~~~~~~~~~~~~~~~~~~~~~~~~~~~~~~~~~~~~~~
Sterownik PLC wykorzystywany do realizacji projektu był wyposażony w następujące moduły:
\begin{enumerate}
\item SIMATIC S7-300, Jednostka centralna S7-300 CPU 315F-2 PN/DP,
\item SIMATIC S7-300, Zasilacz PS 307,
\item SIMATIC S7-300, Wejścia/Wyjścia cyfrowe SM 323,
\item SIMATIC S7-300, Wejścia/Wyjścia analogowe SM 334.
\end{enumerate}
%\begin{figure}[!htb] 	\centering 	\includegraphics[width=0.75\textwidth]{obrazki/sterownik.png} 	\caption{Siemens SIMATIC S7-300} \end{figure}
\indent
\indent Sterownik podłączony jest do sieci lokalnej Ethernet w~laboratorium, więc komunikacja z~nim odbywa się tak samo jak z~każdym innym urządzeniem sieciowym. Podstawy programowania i korzystania ze sterowników autor poznał zapoznając się z odpowiednią literaturą \cite{plc1,plc2,plc4,plc5,plc6}.
Konfigurację sterownika wraz z modułami przedstawia Rysunek~\ref{conf}.
%\begin{figure}[!htb] 	\centering 	\includegraphics[width=0.75\textwidth]{obrazki/conf.png} \caption{Konfiguracja sterownika PLC} \label{conf} \end{figure}
\subsubsection{Komputer}
Projekt w całości był realizowany na laptopie autora, podłączanym do~sieci w~laboratorium. Na~komputerze uruchomiana była maszyna wirtualna. Na~jednej zainstalowane było środowisko TwinCAT do~programowania sterownika oraz do~tworzenia i~uruchamiania wizualizacji. Wizualizacje tworzone w~środowisku TwinCAT można uruchomić bezpośrednio na~komputerze wyposażonym w~odpowiednie oprogramowanie lub na~urządzeniu docelowym po~podpięciu do~niego monitora (o~ile urządzenie docelowe posiada wyjście DVI lub odpowiedni interfejs systemowy w~postaci odrębnego modułu).

\subsection{Analiza tematu}
Analiza tematu polegała przede wszystkim na zapoznaniu się z~narzędziami programistycznymi do~tworzenia oprogramowania sterownika oraz wizualizacji.
W~wyniku analizy autor poznał podstawy obsługi środowiska TwinCAT oraz jego elementów składowych a w szczególności:
\begin{itemize}
\item TwinCAT System Manager - centralne narzędzie konfiguracyjne,
\item TwinCAT PLC - narzędzie do tworzenia programów,
\item TwinCAT NC/CNC - grupa narzędzi do sterowania osiami w różnych trybach.
\end{itemize} 

\subsection{EtherCAT}
zasada działania
topologia
źródła opóźnień
wady/zalety
itd.

\subsection{Założenia}
Oprogramowanie dla dostępnego stanowiska laboratoryjnego powinno zostać stworzone przy użyciu środowiska TwinCAT. Funkcjonalności robota wchodzące w~skład projektu, to:
\begin{itemize}
\item sterowanie ręczne z~pilota podłączonego bezpośrednio do~sterownika,
\item sterowanie ręczne z~wizualizacji,
\item sterowanie automatyczne, 
\item wizualizacja stanu stanowiska.
\end{itemize}
\indent
\indent Powyżej zostały wymienione założenia podstawowe, jednak autor nie wyklucza zrealizowania dodatkowych zadań, które nie zostały zamieszczone w~pierwotnej koncepcji realizacji projektu.

\subsection{Plan pracy}
Realizacja projektu została podzielona na następujące etapy:
\begin{itemize}
\item Zapoznanie się ze sterownikami Beckhoff oraz oprogramowaniem TwinCAT,
\item Zapoznanie się z dostępnymi serwonapędami Beckhoff,
\item Projekt i realizacja stanowiska,
\item Przygotowanie stanowiska do współpracy z systemem wizualizacji,
\item Testowanie i uruchamianie,
\item Przedstawienie projektu i~ewentualne korekty.
\end{itemize}
\indent
\indent Powyższy plan pracy stanowił dla autora wyznacznik kolejnych działań. Jednak powszechnie wiadomo, że w~praktyce poszczególne punkty są~wymienne i~wpływają na siebie wzajemnie.
 %
\section{Analiza tematu}
Analiza tematu polegała przede wszystkim na zapoznaniu się z~informacjami o~standardzie EtherCAT oraz narzędziami programistycznymi do~tworzenia oprogramowania sterownika oraz wizualizacji.

Najważniejszym punktem analizy było zapoznanie się z~informacjami znajdującymi się na stronie internetowej EtherCAT Technology Group, czyli twórców i organizacji zajmującej~się popularyzowaniem standardu \cite{ETG_doc}.
Dodatkowo autor przeczytał wiele artykułów polsko oraz anglojęzycznych poświęconych właśnie protokołowi EtherCAT \cite{art1_etherCAT, art2_etherCAT, art3_etherCAT, art4_etherCAT, art5_etherCAT, art6_etherCAT, art7_etherCAT, art8_etherCAT, art9_etherCAT}. Kilka z~nich (starszych) traktowało~go jako coś bardzo przyszłościowego i~obiecującego, natomiast pozostała część opisywała go już jako coś, co~aktualnie bardzo szybko popularyzuje się i~usprawnia procesy przemysłowe.
W~czasie tworzenia dokumentu opisującego przeprowadzone badania autor przeszukiwał Internet w~celu uzyskania bardziej szczegółowych informacji takich jak budowa nagłówków i~znaczenie oraz rozmiar poszczególnych pól. Poszukiwania te~pozwoliły znaleźć sporo interesujących informacji i~rozwiązań, które mogą stać się podstawą kolejnych badań w~przyszłości.

Zanim autor przeczytał materiały dotyczące standardu EtherCAT poznał podstawy obsługi środowiska TwinCAT oraz jego elementów składowych pozwalające na realizację tematu, a~w~szczególności:
\begin{itemize}
\item TwinCAT System Manager - centralne narzędzie konfiguracyjne,
\item TwinCAT PLC - narzędzie do tworzenia programów,
\item TwinCAT NC/CNC - grupa narzędzi do sterowania osiami w różnych trybach.
\end{itemize} 
Poznanie tych podstaw pozwoliło autorowi na~stworzenie potrzebnego oprogramowania oraz na~konfigurację stanowisk w~sposób odpowiedni do~przeprowadzenia założonych badań.
\subsection{EtherCAT}
%zasada działania
%topologia
%źródła opóźnień
%wady/zalety
%itd.
%%%%%%%%%%%%%%%%%%%%%%%%%%%%%%%%%%%%%%%%%%%%%%%%%%%%%%%%%%%%%%%%%%%%%%%%%%%%%%%%%%%%%%%%%%%

EtherCAT jest nowoczesnym protokołem sieciowym przeznaczonym do stosowania w~aplikacjach przemysłowych, szczególnie takich, które wymagają działania całego systemu w~czasie rzeczywistym. Nazwa standardu jest skrótem od hasła: „Ethernet for Control Automation Technology”. W~zakresie warstwy fizycznej bazuje na~Ethernecie. Dodatkowo zaimplementowano w!nim mechanizmy w zakresie organizacji transmisji danych pozwalające na!ominięcie głównych ograniczeń sieci Ethernet. Dzięki temu EtherCAT jest obecnie jednym z~popularniejszych oraz szybciej rozwijających się protokołów komunikacyjnych w przemyśle.

\subsubsection{Przetwarzanie ,,w locie''}
W dużej części aplikacji przemysłowych dane mają mały rozmiar rzędu pojedynczych bajtów. Wykorzystując standardowy Ethernet i~jego ramki stosunek danych użytecznych do narzutu protokołu jest bardzo nie korzystny. 
W~celu zapewnienie determinizmu oraz zwiększenia przepustowości stosuje się w~standardach przemysłowych różne rozwiązania. Jednym z przykładów jest zastępowanie procedury dostępu do medium transmisyjnego z~wykorzystaniem wielodostępu z~wykrywaniem nośnej oraz detekcją kolizji (ang. Carrier Sense Multiple Access/with Collision Detection w~skrócie CSMA/CD) na m.in mechanizm odpytywania.
Nie jest istotnym jaką metodę dokładnie zastosujemy dopóki ramki są rozsyłane pojedynczo z oraz do urządzeń. Działania te nie wyeliminuje problemu marnowania przepustowości kanału transmisyjnego i jego wykorzystania. Dlatego ten właśnie element transmisji danych z wykorzystaniem Ethernetu został potraktowany bardzo szczególnie przy projektowaniu standardu EtherCAT.

Zatem zamiast standardowej dla Ethernetu transmisji pakietowej z~koniecznością odtwarzania pofragmentowanych danych zastosowano mechanizm wykorzystujący telegramy zbudowane z datagramów, które są szczegółowe opisane w podrozdziale~\ref{subsec:telegram} .
Rozwiązanie to~polega na tym, że w pojedynczej ramce są zawarte informacje przeznaczone dla wielu różnych węzłów podrzędnych. Transmisja takiej ramka inicjowana jest przez węzeł nadrzędny, a~następnie przechodzi ona przez kolejne węzły podrzędne sieci, które przetwarzają ją w locie. W takim podejściu twórcy standardu EtherCAT potraktowali ramkę ethernetową jak pewnego rodzaju pamięć RAM, do której zapisywane i z~której odczytywane są dowolne informacje w~oparciu o~adresy lokalizujące pożądane dane w~tej pamięci.
Przetwarzanie to polega, że w~momencie odebrania ramki sprawdzane jest czy znajduje się w~niej informacja przeznaczona dla tego właśnie węzła. Jeżeli zostanie wykryte, że~tak~jest  to węzeł odczytuje odpowiedni fragment danych oraz uzupełnia ewentualnie ją o~informacje potwierdzające odbiór lub~inne wymagane przez węzeł nadrzędny. Następnie ramka jest przesyłana do kolejnego w topologi węzła lub zawraca jeśli dany węzeł jest ostatnim w~sieci. 

Dzięki takiemu podejściu protokół charakteryzuje się bardzo dużym wykorzystaniem kanału transmisyjnego w porównaniu do odpytywania w przedziale czasowym (ang. Pooling Timeslicing) oraz nadawania Master/Slave (ang. Broadcast Master/Slave) znanych ze~zwykłego EThernetu co pokazano na Rysunku~\ref{etherCAT:wykorzystanie}
\input{tikz/wykorzystanie}

Minimalny rozmiar ramka ethernetowej wynosi 8~bajtów. Załóżmy przykładowo, że~urządzenie okresowo przesyła 4 bajty informacji, na przykład informację o~swoich aktualnych ustawieniach, a w odpowiedzi otrzymuje również 4~bajty danych, na~przykład zestaw komend i~informacji kontrolnych, przy założeniu nieskończenie krótkiego czas odpowiedzi węzła, użyteczna przepustowość wyniesie zaledwie $\frac{4}{84}\approx4,8\%$. Jeżeli średni czas odpowiedzi będzie dłuższy, na przykład wyniesie 10 $\mu s$, to użyteczna przepustowość spadnie do zaledwie $1,9\%$.
\vspace{-1cm}
 \vspace{-1cm}
\begin{figure}[htbp]
 \centering
        \tikzstyle{background grid}=[draw, black!50,step=.25cm]
	\begin{tikzpicture}[node distance=1cm, auto]%, show background grid]
	\tikzset{
    	mynode/.style={rectangle,rounded corners,draw=black, fill=red!15,very thick, inner sep=1.2em, minimum size=2.5em, 		text centered, text width=2.5cm},
    	mynodemini/.style={rectangle,rounded corners,draw=black, fill=red!15,very thick, inner sep=.5em, text centered, text width=2.5cm},    	
	    myarrow/.style={->, >=latex', shorten >=1pt, ultra thick, blue},
	} 
	\node[mynode] (device2) {Urządzenie 2};  
	\node[mynode, fill=blue!15, left=5cm of device2] (device1) {Urządzenie 1};
	\node[right=0.1cm of device2] (loop) {};
	
	\draw[myarrow, red] (loop) to [out=290,in=70,looseness=20] (loop) node[right=0.7cm,black, text width=1.5cm] {Czas reakcji węzła};
	\draw[myarrow] ([yshift=5mm]device1.east) -- ([yshift=5mm]device2.west) node [midway,above,black] {4 bajty informacji};		
	\draw[myarrow] ([yshift=-5mm]device2.west) -- ([yshift=-5mm]device1.east) node [midway,below,black] {4 bajty odpowiedzi };		
\end{tikzpicture} 
 \vspace{-1cm}
\caption{Przykład transmisja małej ilości danych (4 bajty) zwykłem Ethernetem.}
\label{etherCAT:ethernet}
\end{figure}

Wydajność tego rozwiązania jest dosyć duża, choć zależy od~liczby elementów podłączonych do sieci. W~praktyce czas opóźnienia nie wzrasta powyżej 1~ms i~zazwyczaj jest znacznie (kilku- lub kilkunastokrotnie) krótszy. Czas synchronizacji nie przekracza 1~$\mu s$. Niniejsza praca miała na celu przebadanie i~sprawdzenie czy faktycznie protokół działa tak dobrze jak zapewniają jego twórcy.

Bardzo ważnym elementem węzła sieci jest tak zwana jednostka zarządzania pamięcią FMMU (ang. fieldbus memory management unit). Odpowiada ona między innymi za uniezależnienie szybkości transferu danych od wydajności i~mocy obliczeniowej jednostki lokalnej CPU, kontrolę ruchu bez opóźnień oraz za~odwzorowanie adresu logicznego na~adres fizyczny.

\subsubsection{Telegram EtherCAT}
\label{subsec:telegram}
Jak pokazano na Rysunku~\ref{etherCAT:ramka}, telegram EtherCAT jest upakowany w  ramce Ethernet i  zawiera jeden lub więcej datagramów EtherCAT dostarczanych do urządzeń podrzędnych. Dane pomiędzy węzłami są~przekazywane jako obiekty danych procesowym (ang. process data objects, w skrócie PDO). Każdy obiekt tego typu zawiera adres konkretnego węzła lub kilku węzłów typu slave.
\begin{figure}[htbp]
 \centering
        \tikzstyle{background grid}=[draw, black!50,step=.25cm]
	\begin{tikzpicture}[node distance=2mm, auto]%, show background grid]
	\tikzset{
    	mynode/.style={rectangle,rounded corners,draw=black, top color=white, very thick, inner sep=4mm, 		text centered,font=\footnotesize},
    	mynodemini/.style={rectangle,rounded corners,draw=black, top color=white, thick, inner sep=2mm, text centered,font=\scriptsize},    	
	    myarrow/.style={->, >=latex', shorten >=1pt, ultra thick},
	    myline/.style={-, =latex', shorten >=1pt, rounded corners, ultra thick},
	    mylabel/.style={text centered, font=\scriptsize\bfseries} 
	} 
	\node[bottom color=gray!50, mynode] (ethhdr) {Ethernet header};  
	\node[bottom color=gray!50, mynodemini, below=of ethhdr.205] (da) {DA};
	\node[mylabel, below=1mm of da] (das) {6B};	
	\node[bottom color=gray!50, mynodemini, right=of da] (sa) {SA};  
	\node[mylabel, below=1mm of sa] (sas) {6B};	
	\node[bottom color=gray!50, mynodemini, right=of sa] (typ) {Typ};	
	\node[mylabel, below=1mm of typ] (typs) {2/4B};	
	\node[mylabel, below=of sas] (ethhdradr) {EtherType 0x88A4};
	
	\node[bottom color=yellow!50, mynode, right=of ethhdr, text width=2cm] (ecat) {EtherCAT};
	\node[bottom color=yellow!50, mynodemini, below=of ecat, text width=2.4cm] (ecathdr) {EtherCAT header};
	\node[mylabel, below=1mm of ecathdr] (ecathdrs) {2B};
	
	\node[bottom color=yellow!50, mynode, right=of ecat, text width=6.3cm] (ecatt) {EtherCAT telegram};
	\node[bottom color=yellow!50, mynodemini, below=of ecatt.193] (ecatd1) {Datagram 1};  	 
	\node[mylabel, below=1mm of ecatd1] (ecatd1s) {(10+n+2)B};
	\node[bottom color=yellow!50, mynodemini, right=of ecatd1] (ecatd2) {Datagram 2};
	\node[mylabel, below=1mm of ecatd2] (ecatd2s) {(10+m+2)B};  	 	 		
	\node[bottom color=yellow!50, mynodemini, right=0.9cm of ecatd2] (ecatdn) {Datagram n};
	\node[mylabel, below=1mm of ecatdn] (ecatdns) {(10+k+2)B};
	\node [fit=(ecatd1s) (ecatdns) (ecatdns)] (fit) {};  
 	%\draw [decorate, xshift=-20pt,line width=4pt] (fit.south east) -- (fit.north east);
	\draw [decorate,decoration={brace,amplitude=10pt}, line width=1pt] (fit.south east) ++(-0.3,0.3) -- ++(-6.9,0) (fit.south west);		
	\node[mylabel, below=1mm of fit] (fits) {44--1498B};
			
	\node[bottom color=gray!50, mynode, right=of ecatt] (eth) {Ethernet};
	\node[bottom color=gray!50, mynodemini, below=of eth.222] (pad) {Pad.};
	\node[mylabel, below=1mm of pad] (pads) {0--32B};
	\node[bottom color=gray!50, mynodemini, right=of pad] (fcs) {FCS}; 	 
	\node[mylabel, below=1mm of fcs] (fcss) {4B}; 		
 
	\draw[myline,black,dotted] (ecatd2) -- (ecatdn); 	
	
	\node[mylabel, below=of ethhdradr] (dal) {DA -- Destination Address};
	\node[mylabel, right=of dal] (sal) {SA -- Source Address};
	\node[mylabel, right=of sal] (padl) {Pad. -- Payload};
	\node[mylabel, right=of padl] (fcsl) {FCS -- Frame Check Sequance (CRC)};			
	
\end{tikzpicture} 
\caption{Ramka w transmisji EtherCAT i jej podział na datagramy}
\label{etherCAT:ramka}
\end{figure} %

Dane w systemach opartych na sieciach EtherCAT mogą być przesyłane między różnymi sieciami poprzez protokół UDP, a nie tylko w ramach jednej podsieci. Tam gdzie konieczny jest routing pakietów w oparciu o~protokół IP, ramki EtherCAT są przesyłane w~ramach pakietów protokołu UDP lub TCP (Rysunki~\ref{etherCAT:ramkaUDP}~oraz~\ref{etherCAT:ramkaTCP}). Pakiety tego typu mogą zostać uformowane i nadane z wykorzystaniem zwykłych urządzeń ethernetowych, dzięki czemu możliwe jest sterowanie aparaturą polową EtherCAT z~poziomu klasycznego komputera~PC wyposażonego w~kartę sieciową.
\begin{figure}[htbp]
 \centering
        \tikzstyle{background grid}=[draw, black!50,step=.25cm]
	\begin{tikzpicture}[node distance=2mm, auto]%, show background grid]
	\tikzset{
    	mynode/.style={rectangle,rounded corners,draw=black, top color=white, very thick, inner sep=4mm, 		text centered,font=\footnotesize},
    	mynodemini/.style={rectangle,rounded corners,draw=black, top color=white, thick, inner sep=2mm, text centered,font=\scriptsize},    	
	    myarrow/.style={->, >=latex', shorten >=1pt, ultra thick},
	    myline/.style={-, =latex', shorten >=1pt, rounded corners, ultra thick},
	    mylabel/.style={text centered, font=\scriptsize\bfseries} 
	} 
	\node[bottom color=gray!50, mynode, text width=1.6cm] (ethhdr) {Ethernet header}; 
	\node[mylabel, below=1mm of ethhdr] (ethhdrs) {14/16B};		 
	\node[mylabel, below=of ethhdrs.east] (ethhdradr) {EtherType 0x0800};
	\node[mylabel, right=5mm of ethhdradr] (udpadr) {UDP Port 0x88A4};	
	
	\node[bottom color=yellow!50, mynode, right=of ethhdr, text width=1.4cm] (ip) {IP header};
	\node[mylabel, below=1mm of ip] (ips) {20B};	
	
	\node[bottom color=yellow!50, mynode, right=of ip, text width=1.5cm] (udp) {UDP header};		
	\node[mylabel, below=1mm of udp] (udps) {8B};	
		
	\node[bottom color=yellow!50, mynode, right=of udp, text width=2cm] (ecat) {EtherCAT header};
	\node[mylabel, below=1mm of ecat] (ecats) {2B};	
		
	\node[bottom color=yellow!50, mynode, right=of ecat, text width=2cm] (ecatt) {EtherCAT telegram}; 	 	  	
	\node[mylabel, below=1mm of ecatt] (ecatts) {44--1498B};	
			
	\node[bottom color=gray!50, mynode, right=of ecatt] (eth) {Ethernet}; 		
	\node[mylabel, below=1mm of eth] (eths) {4--36B};	
		
\end{tikzpicture} 
\caption{Ramka w transmisji EtherCAT z uwzględnieniem UDP i~IP.}
\label{etherCAT:ramkaUDP}
\end{figure} %
\input{tikz/ramkaTCPIP} %

Rozwiązanie to~zapewnia elastyczną komunikację pomiędzy urządzeniami pracującymi w~standardzie Ethernet oraz EtherCAT, a~także umożliwia wymianę danych pomiędzy urządzeniami znajdującymi się w segmentach sieci podłączonych do różnych routerów. W takim przypadku prędkość wymiany danych zależy w~dużym stopniu od~prędkości działania routerów, co~należy uwzględnić przy ewentualnym projektowanie systemu pracującego w~rozdzielonej sieci.

Każdy datagram EtherCAT jest komendą, która zawiera zgodnie z Rysunkiem~\ref{etherCAT:datagram} nagłówek, dane i~licznik roboczy. Nagłówek i~dane są~używane do~wyspecyfikowania operacji, które muszą być wykonane przez węzeł podrzędny, natomiast licznik roboczy jest aktualizowany przez niego informując węzeł nadrzędny, że~odebrał on i~przetwarza komendę.
\begin{figure}[htbp]
 \centering
        \tikzstyle{background grid}=[draw, black!50,step=.25cm]
	\begin{tikzpicture}[node distance=2mm, auto]%, show background grid]
	\tikzset{
    	mynode/.style={rectangle,rounded corners,draw=black, top color=white, very thick, inner sep=4mm, 		text centered,font=\footnotesize},
    	mynodemini/.style={rectangle,rounded corners,draw=black, top color=white, thick, inner sep=2mm, text centered,font=\scriptsize},    	
	    myarrow/.style={->, >=latex', shorten >=1pt, ultra thick},
	    myline/.style={-, =latex', shorten >=1pt, rounded corners, ultra thick},
	    mylabel/.style={text centered, font=\scriptsize\bfseries} 
	} 
	\node[bottom color=gray!50, mynode] (datagram) {Datagram};  
	\node[bottom color=gray!50, mynode, below=1cm of datagram] (part1) {Dane};   		
	\node[bottom color=gray!50, mynode, left=of part1] (part2) {Nagłówek};   		
	\node[bottom color=gray!50, mynode, right=of part1] (part3) {Licznik};   				
 
	\draw[myline,black,dotted] (datagram.south west) -- (part2.north west); 	
	\draw[myline,black,dotted] (datagram.south east) -- (part3.north east); 
\end{tikzpicture} 
\caption{Budowa datagramu}
\label{etherCAT:ramka}
\end{figure} %

Nagłówek datagramu przedstawiony na Rysunku~\ref{etherCAT:datagram_header}. Jak widać składa się on ze~sporej ilości pól wymagających dokładnego objaśnienia.
\begin{description}
\item[Cmd] Typ rozkazu EtherCAT (ang. EtherCAT Command Type)
\item[Idx] Indeks jest numerycznym identyfikatorem używanym przez węzeł nadrzędny do~wykrywania zdublowanych lub zagubionych datagramów, pole~to nie~powinno być nigdy modyfikowane przez węzły podrzędne
\item[Address] Adres urządzenia zapisany w~sposób zależny od trybu adresacji
\item[Len] Długość danych zawartych w analizowanym datagramie (rozmiar pola danych)
\item[R] Pole zarezerwowane, powinno mieć zawsze wartość 0
\item[C] Pole zabezpieczające ramkę przed zapętleniem\\
0: oznacza, że przetwarzana ramka nie jest zapętlona \\
1: oznacza, że za przetwarzana ramka wykonała już jeden obieg pętli
\item[M] Pole to określa czy w~aktualnie przetwarzanym telegramie są kolejne datagramy \\
0: oznacza, że przetwarzany datagram jest ostatni \\
1: oznacza, że za przetwarzanym datagramem są kolejne
\item[IRQ] Pole żądania zdarzeń od węzłów podrzędnych, pole jest wynikiem sumy logicznej (ang. logical OR) żądań wszystkich węzłów podrzędnych
\end{description}

\begin{figure}[htbp]
 \centering
        \tikzstyle{background grid}=[draw, black!50,step=.25cm]
	\begin{tikzpicture}[node distance=2mm, auto]%, show background grid]
	\tikzset{
    	mynode/.style={rectangle,rounded corners,draw=black, top color=white, very thick, inner sep=4mm, 		text centered,font=\footnotesize},
    	mynodemini/.style={rectangle,rounded corners,draw=black, top color=white, thick, inner sep=2mm, text centered,font=\scriptsize},    	
	    myarrow/.style={->, >=latex', shorten >=1pt, ultra thick},
	    myline/.style={-, =latex', shorten >=1pt, rounded corners, ultra thick},
	    mylabel/.style={text centered, font=\scriptsize\bfseries} 
	} 
	\node[bottom color=yellow!50, mynode] (header) {Nagłówek datagramu};  
	\node[bottom color=yellow!50, mynode, below=1.3cm of header.south west, text width=3.2cm] (mainadress) {Address};	
	\node[mylabel, above=1mm of mainadress] (mainadresss) {4B};		
	\node[bottom color=yellow!50, mynode, below=2.2cm of mainadress.25] (offset) {Offset}; 
	\node[bottom color=yellow!50, mynode, below=1mm of offset] (offset2) {Offset}; 	
	\node[mylabel, above=1mm of offset] (offsets) {2B};		  		
	\node[bottom color=yellow!50, mynode, left=of offset] (position) {Position};
	\node[bottom color=yellow!50, mynode, below=1mm of position] (adress) {Address}; 
	\node[bottom color=yellow!50, mynode, below=1mm of adress.south east, text width=3.2cm] (ladress) {Logical Address}; 			   		
	\node[below=3mm of ladress] (spacing) {};
	\node[mylabel, above=1mm of position] (positions) {2B};
	
	\node[bottom color=yellow!50, mynode, left=of mainadress] (idx) {Idx};   		
	\node[mylabel, below=1mm of idx] (idxs) {1B};
	\node[bottom color=yellow!50, mynode, left=of idx] (cmd) {Cmd};   		
	\node[mylabel, below=1mm of cmd] (cmds) {1B};		
	\node[bottom color=yellow!50, mynode, right=of mainadress] (len) {Len};   				
	\node[mylabel, below=1mm of len] (lens) {11b};	
	\node[bottom color=yellow!50, mynode, right=of len] (r) {R};   				
	\node[mylabel, below=1mm of r] (rs) {3b};	
	\node[bottom color=yellow!50, mynode, right=of r] (c) {C};   				
	\node[mylabel, below=1mm of c] (cs) {1b};		
	\node[bottom color=yellow!50, mynode, right=of c] (m) {M};   				
	\node[mylabel, below=1mm of m] (ms) {1b};	
	\node[bottom color=yellow!50, mynode, right=of m] (irq) {IRQ};   				
	\node[mylabel, below=1mm of irq] (irqs) {2B};	
				 
	\draw[myline,black,dotted] (datagram.south west) -- (cmd.north west); 	
	\draw[myline,black,dotted] (datagram.south east) -- (irq.north east); 
	
	\draw[myline,black,dotted] (mainadress.south west) -- (position.north west); 	
	\draw[myline,black,dotted] (mainadress.south east) -- (offset.north east); 
		
\end{tikzpicture} 
\caption{Budowa nagłówka datagramu.}
\label{etherCAT:datagram_header}
\end{figure} %

Oprócz części zawierającej dane pogrupowane w datagramy telegram EtherCAT zawiera również nagłówek, którego budowę  przedstawia Rysunek~\ref{etherCAT:header}. Pierwsze pole określa długość wszystkich datagramów w~danym telegramie, która zależy od ilości węzłów oraz długości wiadomości. Drugie pole jest bitem zarezerwowanym, a~jego wartość powinna wynosić 0. Ostatnie pole określa typ protokołu EtherCAT. Typ ten definiuje typ wiadomości, co~pozwala na prawidłową interpretację danych.
Standard przewiduje 3~dopuszczalne typy:
\begin{itemize}
\item Typ 1: Protokół EtherCAT dla urządzeń -- Wymiana Datagramów (ang. EtherCAT Device Protocol, EtherCAT Datagram(s)),
\item Typ 4: EAP Wymiana danych procesowych -- wymiany cykliczne (ang. EtherCAT Automation Protocol, Process data communications),
\item Typ 5: EAP Wymiana komunikatów na żądanie (wymiany acykliczne (ang. EtherCAT Automation Protocol, Mailbox communication).
\end{itemize}
\begin{figure}[htbp]
 \centering
        \tikzstyle{background grid}=[draw, black!50,step=.25cm]
	\begin{tikzpicture}[node distance=2mm, auto]%, show background grid]
	\tikzset{
    	mynode/.style={rectangle,rounded corners,draw=black, top color=white, very thick, inner sep=4mm, 		text centered,font=\footnotesize},
    	mynodemini/.style={rectangle,rounded corners,draw=black, top color=white, thick, inner sep=2mm, text centered,font=\scriptsize},    	
	    myarrow/.style={->, >=latex', shorten >=1pt, ultra thick},
	    myline/.style={-, =latex', shorten >=1pt, rounded corners, ultra thick},
	    mylabel/.style={text centered, font=\scriptsize\bfseries} 
	} 
	\node[bottom color=yellow!50, mynode] (header) {Nagłówek};  
	\node[bottom color=yellow!50, mynode, below=1cm of header] (part1) {Zarezerwowany}; 
	\node[mylabel, below=1mm of part1] (part1s) {1b};		  		
	\node[bottom color=yellow!50, mynode, left=of part1] (part2) {Długość};   		
	\node[mylabel, below=1mm of part2] (part2s) {11b};	
	\node[bottom color=yellow!50, mynode, right=of part1] (part3) {Typ};   				
	\node[mylabel, below=1mm of part3] (part3s) {4b};	
	 
	\draw[myline,black,dotted] (datagram.south west) -- (part2.north west); 	
	\draw[myline,black,dotted] (datagram.south east) -- (part3.north east); 
\end{tikzpicture} 
\caption{Budowa nagłówka EtherCAT}
\label{etherCAT:header}
\end{figure} %

\subsubsection{Topologia}
Kolejne ramki nadawane są przez urządzenie pełniące rolę kontrolera i~przesyłane do najbliższego urządzenia podrzędnego. Urządzenie to ma przydzielony adres, który jednoznacznie identyfikuje datagram, dzięki czemu może odczytać przeznaczone dla siebie dane. Niezależnie od tego, czy urządzenie coś zapisało, czy też tylko odczytywało fragment ramki, przesyła ją dalej do kolejnego z~urządzeń zgodnie z tym co pokazano na Rysunku~\ref{etherCAT:topologia}. Duża szybkość tej metody transmisji wynika m.in. z tego, że nie ma potrzeby dekodowania całej ramki danych. Pozwala to błyskawicznie przekazywać ramki pomiędzy urządzeniami.

\begin{figure}[htbp]
 \centering
        \tikzstyle{background grid}=[draw, black!50,step=.25cm]
	\begin{tikzpicture}[node distance=1cm, auto]%, show background grid]
	\tikzset{
    	mynode/.style={rectangle,rounded corners,draw=black, top color=white, bottom color=orange!50,very thick, inner sep=0.6em, minimum size=2.5em, 		text centered, text width=2.5cm},
    	mynodemini/.style={rectangle,rounded corners,draw=black, top color=white, bottom color=orange!50,very thick, inner sep=.5em, text centered},    	
	    myarrow/.style={->, >=latex', shorten >=1pt, ultra thick},
	    myline/.style={-, =latex', shorten >=1pt, rounded corners, ultra thick},
	    mylabel/.style={text width=7em, text centered} 
	} 
	\node[mynode] (master) {Węzeł nadrzędny \\ (ang. \textit{master})};  
	\node[mynode, below right=of master] (slave1) {Węzeł podrzędny 1 \\ (ang. \textit{slave})};
	\node[mynode, right=of slave1] (slave2) {Węzeł podrzędny 2 \\ (ang. \textit{slave})};
	\node[mynode, right=of slave2] (slaven) {Węzeł podrzędny n \\ (ang. \textit{slave})};  	 	 		
	
	\draw[myarrow,black] (master.350) -| (slave1.135);
	\draw[myarrow,black] (slave1.45) -- ++(0,1.5) -| (slave2.135);
	\draw[myarrow,black,dotted] (slave2.45) -- ++(0,1.5) -| (slaven.135);	
	\draw[myarrow,black] (slaven.45) |- (master.20);	
 
\end{tikzpicture} 
\caption{Przykładowa topologia sieci}
\label{etherCAT:topologia}
\end{figure} %

Łatwo zauważyć, że opisana procedura transmisji wymaga zastosowania topologii magistrali, która w przypadku klasycznego Ethernetu jest generalnie nieopłacalna i ma wiele wad. Problem ten został rozwiązany poprzez zwielokrotnienie portów ethernetowych instalowanych w dużej części urządzeń zgodnych z EtherCAT.
Dosyć powszechnie spotykane są urządzenia z~dwoma lub trzema portami, a~te wystarczają już do realizacji topologii gwiazdy, czy nawet zmieszanie topologii gwiazdy i~magistrali w~ramach jednej sieci i~utworzenie struktury mocno redundantnej.

\subsubsection{Warstwa fizyczna}
Jako medium komunikacyjne w~sieciach EtherCAT można wykorzystać kable miedziane (100Base-TX), światłowody (100Base-FX) lub łącze E-bus w~technologii LVDS. Te~ostatnie wprowadzono ze~względu na~to, że~transmisja w~sieciach EtherCAT często jest realizowana na~krótkich dystansach - E-bus sprawdza się w~realizacji łączności na~odległość do~około 10 m. Kable miedziane sprawdzają~się na~większych odległościach nieprzekraczających 100 metrów, natomiast użyteczna długość światłowodów może dochodzić aż do 20 kilometrów. Jedynym warunkiem związanym z~wykorzystaniem światłowodów jest obsługa pełnego dupleksu. Wymóg ten jest podyktowany faktem, że~transmisja danych jest tak szybka iż z~reguły ramka odpowiedzi dociera do urządzenia nadrzędnego zanim całe zapytanie zostanie wysłane. Z~tego powodu, aby przesył danych pomiędzy urządzenami nie był zakłócony musi istnieć możliwość jednoczesnego przekazywania informacji w dwóch kierunkach bez spadku transferu.

W~obrębie jednej sieci EtherCAT można dowolnie zmieniać medium transmisyjne zależnie od~potrzeb. Na przykład wewnatrz szafy sterowniczej gdzie występują niewielkie odległości miedzy węzłami można zastosować z powodzeniem łącze E-bus, natomiast do połączenia szafy z modułami znajdujacymi się przy maszynach wykonawczych możemy zastosować kabel miedzieny lub światłowód w~przypadku zdecydowanie większych odległości oraz prowadzenia przewodów w~otoczeniu występowania zaburzeń elektromagnetycznych. Niestety operacja takiego łączenia wymaga zastosowania dotatkowych modułów. Na przykład, aby połaczyć ze sobą swykłą skrętke miedzianą oraz przewód światłowodowy należy zastosować moduł sprzęgający EK1501 oraz terminal przyłączy EK1521 (oba produkcji firmy Beckhoff).

\subsubsection{Synchronizacja}
Idealna synchronizacja elementów składowych systemu jest bardzo istotna, a szczególnie w przypadku równoległej realizacji zależnych od siebie zadań.Specjalnie opracowana dla protokołu EtherCAT technika zegara rozproszonego pozwala zsynchronizować ze sobą urządzenia z dewiacją mniejszą niż 1~$\mu s$. Podejście to polega na~wykorzystaniu znaczników czasu zapisywanych przez każdy węzeł podrzędny. Na~ich podstawie węzeł~master oblicza opóźnienia propagacji sygnałów dla każdego węzła slave. Zegar w~każdym węźle podrzędnym jest regulowany z~wykorzystaniem wyliczonych opóźnień. Po zainicjowaniu połączenia w~celu utrzymania synchronizacji zegarów muszą one przekazywać miedzy sobą regularnie informację, by uniknąć powstania ewentualnych różnic. Rozwiązanie to jest zgodne z~protokołem precyzyjnej synchronizacji zegara dla sieciowych systemów kontrolno-pomiarowych lub inaczej protokół czasu precyzyjnego (ang. Precision Time Protocol, w skrócie PTP) IEEE~1588 opisanym w~\cite{ieee}.
\begin{figure}[htbp]
 \centering
        \tikzstyle{background grid}=[draw, black!50,step=.25cm]
	\begin{tikzpicture}[node distance=1cm, auto]%, show background grid]
	\tikzset{
    	mynode/.style={rectangle,rounded corners,draw=black, top color=white, bottom color=orange!50,very thick, inner sep=1em, minimum size=2.5em, 		text centered, text width=2.5cm},
    	mynodemini/.style={rectangle,rounded corners,draw=black, top color=white, bottom color=orange!50,very thick, inner sep=.5em, text centered},    	
	    myarrow/.style={->, >=latex', shorten >=1pt, ultra thick},
	    myline/.style={-, =latex', shorten >=1pt, rounded corners, ultra thick},
	    mylabel/.style={text width=7em, text centered} 
	} 
	
	\node[mynode] (master) {Węzeł nadrzędny \\ (ang. master)};  
	\node[mynode, below right=of master] (slave1) {Węzeł podrzędny 1 \\ (ang. slave)};
	\node[mynode, right=of slave1] (slave2) {Węzeł podrzędny 2 \\ (ang. slave)};
	\node[mynode, right=of slave2] (slaven) {Węzeł podrzędny n};  	 	 		
	
	\draw[myarrow,black] (master.350) -| (slave1.135);
	\draw[myarrow,black] (slave1.45) -- ++(0,1.5) -| (slave2.135);
	\draw[myarrow,black,dotted] (slave2.45) -- ++(0,1.5) -| (slaven.135);	
	\draw[myarrow,black] (slaven.45) |- (master.20);	
 
\end{tikzpicture} 
\caption{Schemat pracy zegarów rozproszonych}
\label{etherCAT:topologia}
\end{figure} %

\subsubsection{Realizacja węzłów EtherCAT}
Wiele najprostszych i najtańszych urządzeń z interfejsem EtherCAT można zrealizować z wykorzystaniem pojedynczego układu scalonego FPGA lub ASIC. Przykładami takich prostych urządzeń z zastosowaniem tego rozwiązania są moduły wejść/wyjść cyfrowych. Tego typu węzły nie wymagają tworzenia dodatkowego oprogramowania, ponieważ cała funkcjonalność jest realizowana w~pełni sprzętowo. Uogólniona i uproszczona architektura tej metody realizacji została przedstawiona na Rysunku~\ref{etherCAT:FPGA_ASIC}.
\begin{figure}[htbp]
 \centering
        \tikzstyle{background grid}=[draw, black!50,step=.25cm]
	\begin{tikzpicture}[node distance=1cm, auto]%, show background grid]
	\tikzset{
    	mynode/.style={rectangle,rounded corners,draw=black, fill=red!15,very thick, inner sep=1.2em, minimum size=2.5em, 		text centered, text width=2.5cm},
    	mynodemini/.style={rectangle,rounded corners,draw=black, fill=red!15,very thick, inner sep=.5em, text centered, text width=2.5cm},    	
	    myarrow/.style={<->, >=latex', shorten >=1pt, ultra thick},
	    myline/.style={-, =latex', shorten >=1pt, rounded corners, ultra thick},
	    mylabel/.style={text width=7em, text centered} 
	} 
	\node[mynode] (asic_fpga) {ASIC/FPGA \\ EtherCAT};  
	\node[mylabel, left=of asic_fpga] (io) {Cyfrowe \\ wejścia/wyjścia};
	\node[mynodemini, right=of asic_fpga.north east] (layer1) {Warstwa \\ fizyczna};
	\node[right=of layer1] (empty1) {};
	\node[mynodemini, right=of asic_fpga.south east] (layer2) {Warstwa \\ fizyczna};  	 	 		
	\node[right=of layer2] (empty2) {};	

%	\draw[myarrow] (io) -- (asic_fpga);	
    \foreach \i in {-2,...,2}{% 
      \draw[myarrow] ([yshift=\i * 0.4 cm]io.east) -- ([yshift=\i * 0.4 cm]asic_fpga.west) ;}
	
	\draw[myarrow] (asic_fpga.north east) -- (layer1);	
	\draw[myarrow] (asic_fpga.south east) -- (layer2);	
				
	\draw[myarrow] (layer1) -- (empty1);	
	\draw[myarrow] (layer2) -- (empty2);	
 
\end{tikzpicture} 
\caption{Budowa węzła EtherCAT z~wykorzystaniem pojedynczego układu FPGA lub ASIC.}
\label{etherCAT:FPGA_ASIC}
\end{figure} %

Jeżeli węzeł potrzebuje dodatkowej mocy obliczeniowej lub wymaga realizacji jakiegoś złożonego oprogramowania, którego nie~da~się zrealizować w~prosty sposób sprzętowo (w układzie FPGA) do~układu ASIC/FPGA EtherCAT dołączany jest zewnętrzny procesor często wyposażony w~pamięć typu Flash tak jak na~Rysunku~\ref{etherCAT:FPGA_ASIC+proc}. Procesor taki ma zapewnić obsługę przetwarzania na~poziomie aplikacji. Niestety koszt tego typu architektury jest wyższy niż w prostszym przypadku pozbawionym zewnętrznej jednostki, ale konstruktor bazujący na tym rozwiązaniu ma większe pole manewru w~doborze procesora odpowiedniego dla wymagań i~budżetu realizowanego projektu.
\begin{figure}[htbp]
 \centering
        \tikzstyle{background grid}=[draw, black!50,step=.25cm]
	\begin{tikzpicture}[node distance=1cm, auto]%, show background grid]
	\tikzset{
    	mynode/.style={rectangle,rounded corners,draw=black, fill=red!15,very thick, inner sep=1.2em, minimum size=2.5em, 		text centered, text width=2.5cm},
    	mynodemini/.style={rectangle,rounded corners,draw=black, fill=red!15,very thick, inner sep=.5em, text centered, text width=2.5cm},    	
	    myarrow/.style={<->, >=latex', shorten >=1pt, ultra thick},
	    myline/.style={-, =latex', shorten >=1pt, rounded corners, ultra thick},
	    mylabel/.style={text width=7em, text centered} 
	} 
	\node[mynode] (asic_fpga) {ASIC/FPGA \\ EtherCAT};  
	\node[mynode, fill=blue!15, left=3cm of asic_fpga] (proc) {Procesor};
	\node[mynodemini, right=of asic_fpga.north east] (layer1) {Warstwa \\ fizyczna};
	\node[right=of layer1] (empty1) {};
	\node[mynodemini, right=of asic_fpga.south east] (layer2) {Warstwa \\ fizyczna};  	 	 		
	\node[right=of layer2] (empty2) {};	

	\draw[myarrow] (proc) -- (asic_fpga) node [midway,above] {Interfejs};		
	\draw[myarrow] (proc) -- (asic_fpga) node [midway,below=0.6cm] {hosta};		
	
	\draw[myarrow] (asic_fpga.north east) -- (layer1);	
	\draw[myarrow] (asic_fpga.south east) -- (layer2);	
				
	\draw[myarrow] (layer1) -- (empty1);	
	\draw[myarrow] (layer2) -- (empty2);	
 
\end{tikzpicture} 
\caption{Budowa węzła z~wykorzystaniem układu ASIC/FPGA EtherCAT z~dołączonym zewnętrznym procesorem.}
\label{etherCAT:FPGA_ASIC+proc}
\end{figure} %

Kolejnym możliwym do~wykorzystania rozwiązaniem jest zastosowanie układu FPGA z~wbudowanym procesorem jak na~Rysunku~\ref{etherCAT:FPGA}. Wspólną cechą przedstawionych dotychczas możliwych konstrukcji jest fakt, że~wymagają z~reguły zastosowania dwóch układów. Wynika z tego niestety, ze zajmują więcej miejsca oraz zwiększają koszty urządzenia docelowego. Alternatywą pozwalającą wyeliminować oba opisane problemy jest zastosowanie elementów jednoukładowych. Ich wykorzystanie pozwala zredukować całkowity koszt konstrukcji nawet o 30\%.
\begin{figure}[htbp]
 \centering
        \tikzstyle{background grid}=[draw, black!50,step=.25cm]
        % Define a few styles and constants
	\tikzstyle{red}=[draw, fill=red!30, text width=5em, text centered, very thick]
	\tikzstyle{naveqs} = [red, text width=6em, fill=white!20, minimum height=1.8cm, rounded corners, very thick]
	\tikzset{
    	mynode/.style={rectangle,rounded corners,draw=black, fill=red!15,very thick, inner sep=1em, minimum size=2.5em, 		text centered, text width=2.5cm},
    	mynodemini/.style={rectangle,rounded corners,draw=black, fill=red!15,very thick, inner sep=.5em, text centered, text width=2.5cm},    	
	    myarrow/.style={<->, >=latex', shorten >=1pt, ultra thick},
	    myline/.style={-, =latex', shorten >=1pt, rounded corners, ultra thick},
	    mylabel/.style={text width=7em, text centered} 
	} 
	    
    \def\blockdist{1.6}
	\def\edgedist{2.5}

\begin{tikzpicture}
    \node (core) [naveqs] {Rdzeń ARM Cortex-A8};
    \node (memory) [naveqs, below=1mm of core] {Pamięć};    
    % Note the use of \path instead of \node at ... below. 
    \path (core.north east)+(\blockdist,-0.45) node (mi) [red] {MIII x2};
    \path (core.east)+(\blockdist,-0.3) node (uart) [red] {UART};
    \path (core.south east)+(\blockdist,-0.2) node (pru) [red] {PRU x2};

    \node (IMU) [below of=pru, text centered, text width=2cm] {Jednostka \\ PRU};
    \path (core.north)+(1.5, 0.45) node (INS) {AM335x};
    
	\node[mynodemini, right=6cm of core.east] (layer1) {Warstwa \\ fizyczna};
	\node[right=of layer1] (empty1) {};
	\node[mynodemini, right=6cm of memory.east] (layer2) {Warstwa \\ fizyczna};  	 	 		
	\node[right=of layer2] (empty2) {};	
	
	\draw[myarrow] (core.east) ++(3.55,0) -- (layer1);	
	\draw[myarrow] (memory.east) ++(3.55,0) -- (layer2);	
				
	\draw[myarrow] (layer1) -- (empty1);	
	\draw[myarrow] (layer2) -- (empty2);  
	  
    % Now it's time to draw the colored IMU and INS rectangles.
    % To draw them behind the blocks we use pgf layers. This way we  
    % can use the above block coordinates to place the backgrounds   
    \begin{pgfonlayer}{background}
        % Compute a few helper coordinates
        \path (core.north |- mi.east)+(5,1.2) node (a) {};        
        \path (INS.south -| memory.west)+(-0.5,-4.2) node (b) {};
        \path[fill=blue!15,rounded corners, draw=black!100, very thick]
            (a) rectangle (b);
            
        \path (mi.north west)+(-0.2,0.2) node (a) {};
        \path (IMU.south -| mi.east)+(+0.2,-0.2) node (b) {};
        \path[fill=red!15,rounded corners, draw=black!100, very thick]
            (a) rectangle (b);
    \end{pgfonlayer}
\end{tikzpicture}

\caption{Budowa układu FPGA z wbudowanym procesorem.}
\label{etherCAT:FPGA}
\end{figure} %

Jednym z~przykładów opisanego jednoukładowego rozwiązania~są mikrokontrolery z~rdzeniem ARM~Cortex-A8 z~rodziny Sitara AM335x produkowanymi przez amerykańską firmę Texas Instruments (Rysunek~\ref{etherCAT:Sitara}). Kluczowym elementem takiego scalaka jest programowalna jednostka czasu rzeczywistego (ang. programmable real-time unit, w~skrócie PRU). 

W~PRU zaimplementowana została warstwa MAC standardu EtherCAT. Dzięki temu odpowiada~ona bezpośrednio za~przetwarzanie przepływających przez nią telegramów, dekodowanie adresów urządzeń oraz wykonywanie zapisanych w datagramie komend. W~wyniku takiego rozwiązania, że~wszystkie założenia oraz cała funkcjonalność jest realizowana właśnie przy użyciu opisywanego PRU, zamontowany wewnątrz procesor może~być zastosowany do~realizacji bardziej zaawansowanych oraz złożonych zadań wynikających ze~specyfiki konkretnego sprzętu.
\begin{figure}[htbp]
 \centering
        \tikzstyle{background grid}=[draw, black!50,step=.25cm]
	\begin{tikzpicture}[node distance=1cm, auto]%, show background grid]
	\tikzset{
    	mynode/.style={rectangle,rounded corners,draw=black, fill=red!15 ,very thick, inner sep=1em, minimum size=2.5em, 		text centered, text width=2.5cm},
    	mynodemini/.style={rectangle,rounded corners,draw=black, fill=red!15,very thick, inner sep=.5em, text centered, text width=2.5cm},    	
	    myarrow/.style={<->, >=latex', shorten >=1pt, ultra thick},
	    myline/.style={-, =latex', shorten >=1pt, rounded corners, ultra thick},
	    mylabel/.style={text width=7em, text centered} 
	} 
	
	\node[mynode, fill=blue!15, fit={(asic_fpga) (proc.east)}] (mikro) {}; 
	
	\node[mynode] (asic_fpga) {Interfejs \\ EtherCAT};  
	\node[mylabel, left=1mm of asic_fpga] (proc) {Procesor};	
	
	\node[mynodemini, right=2cm of asic_fpga.north east] (layer1) {Warstwa \\ fizyczna};
	\node[right=of layer1] (empty1) {};
	\node[mynodemini, right=2cm of asic_fpga.south east] (layer2) {Warstwa \\ fizyczna};  	 	 		
	\node[right=of layer2] (empty2) {};	
	
	\draw[myarrow] (asic_fpga.north east) ++(0.5,0) -- (layer1);	
	\draw[myarrow] (asic_fpga.south east) ++(0.5,0) -- (layer2);	
				
	\draw[myarrow] (layer1) -- (empty1);	
	\draw[myarrow] (layer2) -- (empty2);	
 
\end{tikzpicture} 
\caption{Budowa mikrokontrolera Sitara AM335x wyposażonego w programowalną jednostkę czasu rzeczywistego.}
\label{etherCAT:Sitara}
\end{figure} %

\subsubsection{EtherCAT Technology Group}
\begin{figure}[!htb] 	\centering 	\includegraphics[width=0.3\textwidth]{images/logoETG} \caption{Logo EtherCAT Technology Group (http://www.ethercat.org).} \label{logoETG} \end{figure}

Standard EtherCAT został opracowany w~2003 roku przez Beckhoff Automation, niemiecką firmę z~branży automatyki przemysłowej. Następnie powołano organizację EtherCAT Technology Group (ETG), która zajęła~się standaryzacją tego protokołu. Stowarzyszenie~to obecnie zajmuje~się też organizowaniem szkoleń oraz popularyzacją tego standardu. 

Aktualnie w~skład ETG wchodzi ponad 2480 firm (dane na dzień 1~września~2013). Najważniejszym członkiem organizacji jest oczywiście firma BECKHOFF Automation. Pozostałe duże i~znane firmy wchodzące w~jej skład~to między innymi: ABB, Brother Industries, BMW Group, Częstochowa University of Technology, Epson, FANUC, Festo, GE Intelligent Platforms, Hitachi, Hochschule Ingolstadt, Mitsubishi, Microchip Technology, Mentor Graphics, Nikon, National Instruments, OLYMPUS, Panasonic, Rzeszów University of Technology, Red Bull Technology, Samsung Electronics, TRW Automotive, Volvo Group, Volkswagen oraz Xilinx.

Jak widać na powyższej liście w~skład organizacji wchodzą firmy z~bardzo wielu branż, a~nawet ośrodki naukowe. Autor pracy wybrał duże i~dobrze znane sobie firmy, aby pokazać jak wiele firm interesuje~się rozwojem przemysłowych protokołów komunikacyjnych. %
%\lstset{language=Pascal,
        basicstyle=\footnotesize\ttfamily,
        breaklines=true,
        tabsize=2,
        numbers=left,
        numberstyle=\tiny,
        numbersep=7pt,
        showspaces=false,
        keywordstyle=\color{Blue}\textbf,
        commentstyle=\color{Red}\emph,
        showstringspaces=false,
        stringstyle=\color{BurntOrange}
        }
\section{Oprogramowanie sterownika}
W~niniejszym rozdziale opisane zostało oprogramowanie sterujące modelem. W~kolejnych podrozdziałach zostanie przedstawiona specyfikacja zewnętrzna oraz wewnętrzna. 

Stworzone przez autora oprogramowanie wraz ze~wszystkimi funkcjami systemowymi zajmują w~sterowniku 47428~kB z dostępnych 524288~kB. Zajętość pamięci dostępnej w~sterowniku obrazuje zrzut ekranu wykonany w~środowisku Step~7, widoczny na Rysunku~\ref{memory}.
%\begin{figure}[!htb] \centering \includegraphics[width=0.6\textwidth]{obrazki/memory.PNG} \caption{Wykorzystanie pamięci sterownika} \label{memory} \end{figure}

\subsection{Specyfikacja zewnętrzna}
Specyfikacja zewnętrzna przedstawiona w dalszej części podrozdziału zawiera opis, jak korzystać z~oprogramowania wgranego do sterownika przez jego autora. Opisane zostało, jak ustawiać odpowiednie zmienne, aby~uzyskać żądany efekt.

Lista zmiennych wejściowych i wyjściowych wymieniana między sterownikiem a modelem została już opisana w~pierwszym rozdziale, w~Tablicach~\ref{in} oraz~\ref{out}. Pozostałe zmienne znajdują się w wewnętrznej pamięci sterownika.

Oprogramowanie może sterować modelem w~sposób automatyczny lub ręczny. Tryb automatyczny w~trybie obsługi magazynu zostanie opisany w podrozdziale 2.1.2. Tryb ręczny może być realizowany przy pomocy zadajnika podpiętego do sterownika lub przy pomocy przycisków umieszczonych na odpowiednim ekranie wizualizacji. W~trybie tym o~pracy robota decydujący jest stan przycisków. Dopuszczalne są wszystkie możliwe ruchy w~przedziale od wyłącznika krańcowego do wartości maksymalnej.

Sterowanie przy pomocy pilota podłączonego do sterownika odbywa się za pomocą 4~przycisków monostabilnych do załączania silników oraz 4~przełączników bistabilnych do wybierania kierunku. Podobne do sterowania pilotem jest sterowanie ręczne z~poziomu wizualizacji, polegające na odpowiednim modyfikowaniu bitów: M11.0~-~M11.7.

W automatycznym trybie pracy kluczową rolę odgrywają 4~zmienne: \emph{LiftEndPos}, \emph{GrabEndPos}, \emph{ArmEndPos} oraz \emph{RotateEndPos} typu INT, które wskazują na pozycje docelowe silników. Są one podawane do bloków FB1 odpowiadających kolejnym silnikom. O tym, czy wybrany silnik może się w~danym momencie poruszać czy nie decydują flagi odpowiednio ustawione przez blok FB9. 

Ważnym elementem jest kolejka obsługiwana w~trybie automatycznym. Indeksy przechowywane są w~przestrzeni od DB6.DBW0 do DB6.DBW202, a~związane z~nimi bezpośrednio zmienne typu bool - w przestrzeni od DB6.DBX202.0 do DB6.DBX216.0. Z kolejką tą związana jest dodatkowo zmienna \emph{CurrentIndex} określająca jej długość.

Kolejną przestrzenią adresową jest blok DB5, który jest reprezentacją w pamięci sterownika modelu magazynu i informacji o nim. Od adresu DB5.DBX0.0 znajduje się 26-elementowa tablica określająca zajętość poszczególnych komórek magazynu. Następnie od adresu DB5.DBW4 dostępna jest dwuwymiarowa tablica 26x3 przechowująca pozycję poszczególnych komórek magazynu. W~przestrzeni zaczynającej się pod adresem DB5.DBB160 znajduje się 26-elementowa tablica zmiennych typu DATE\_AND\_TIME przechowująca datę ostatniego dostępu do komórki. Dodatkowo w przestrzeni tej mamy dostępną pod adresem DB5.DBB368 zmienną przechowującą aktualną datę oraz godzinę w sterowniku, odświeżaną przy każdym przebiegu bloku OB1.

\subsection{Specyfikacja wewnętrzna}
Podrozdział specyfikacja wewnętrzna opisuje sposób rozwiązania przez autora kwestii sterowania modelem przy użyciu dostępnego na~stanowisku sterownika oraz poszczególnych trybów sterowania.
W~tworzeniu oprogramowania zostały wykorzystane następujące języki programowania:
\begin{itemize} 
\item język drabinkowy (Ladder), wykorzystany do stworzenia głównych elementów programu,
\item S7-SCL, który został zastosowany do korzystania z tablic. Niestety do korzystania z nich nie można zastosować języka LAD, ponieważ nie da się w~nim odwoływać do elementów tablicy przez indeksy będące zmiennymi, a~jedynie przez stałe. Po zapoznaniu się z~dokumentacją okazało się, że~taka możliwość istnieje w~języku STL, ale jest to metoda skomplikowana w~implementacji. Właśnie dlatego najlepszym i~najprostszym rozwiązaniem okazuję się S7-SCL, który jest kompilowany do kodu w~języku STL.
\end{itemize} 
Istotne fragmenty programu aplikacyjnego:
\begin{itemize} 
\item blok funkcyjny FB1 - blok sterujący pracą poszczególnych silników,
\item blok funkcyjny FB8 - blok zawierający implementację kolejki FIFO,
\item blok funkcyjny FB9 - blok analizujący stan modelu i ustawiający zmienne zezwalające na ruch silników,
\item blok funkcyjny FB10 - blok przetwarzający zadania z kolejki na odpowiednie dane (operuje na odpowiednich tablicach),
\item blok funkcyjny FB11 - blok zwracający indeks i zmienną bool najbliższego lub najdalszego elementu,
\item blok funkcyjny FB12 - blok zwracający indeks i zmienną bool najmłodszego lub najstarszego elementu.
\end{itemize} 
\indent
\indent Wszystkie wymienione bloki zostaną szczegółowo opisane w kolejnych podrozdziałach.
\vspace*{-9mm}
%dokładniejszy opis poszczególnych bloków w kolejnych podrozdziałach
\subsubsection{Blok FB1}
Najważniejszą częścią oprogramowania jest blok funkcyjny~FB1. Wszystkie zmienne wejściowe oraz wyjściowe dla tego bloku zostały zebrane w Tablicah~\ref{fb1datain}~oraz~\ref{fb1dataout}. Do~działania wykorzystuje~on zestaw danych wewnętrznych. Jako parametry wejściowe bloku, oprócz zmiennych z~modułu~I/O sterownika, podajemy typ aktualnego trybu, maksymalną dopuszczalną wartość wewnętrznego licznika, pozycję docelową w~trybie automatycznym oraz~zmienne \emph{Enable} i \emph{ResetSequence}. Ostatnie dwie zmienne odpowiadają za~to, aby~ruchy w~trybie automatycznym i resetowanie wykonywane były w~odpowiedniej kolejności. Na~wyjściu mamy tylko połączenia dla~zmiennych z~tablicy Symbols oraz~flagę osiągnięcia pozycji docelowej przez dany silnik.

\begin{table}[!htb]
\begin{center}
\begin{tabular}{|c|c|p{8.9cm}|}\hline
Nazwa & Typ & Opis  \\
zmiennej &  zmiennej &   \\\hline
StopSensor & Bool & Sygnał z odpowiedniej krańcówki \\\hline   
EngineCounter & Bool & Sygnał z impulsatora obrotów\\\hline   
CounterMax\_I & Int & Maksymalna wartość licznika typu Int \\\hline   
EndPosition & Int & Pozycja docelowa \\\hline   
Automat & Bool & Tryb automatyczny \\\hline   
ManualWorkDirection & Bool & Start w trybie ręcznym z                                   pilota \\\hline   
ManualWorkStart & Bool & Start w trybie ręcznym z                               wizualizacji \\\hline   
VisualWorkDirection & Bool & Kierunek w trybie ręcznym                                   z pilota \\\hline   
VisualWorkStart & Bool & Kierunek w trybie ręcznym                               z wizualizacji \\\hline   
CounterForState & Counter & Licznik wewnętrzny z aktualną pozycją \\\hline   
Enable & Bool & Sygnał zezwalający na                                                  inkrementację \\\hline   
ResetSequence & Bool & Zmienna pozwalająca na sekwencyjny reset \nobreak{modelu} \\\hline   
VisualPilot & Bool & Zmienna określająca tryb pracy manualnej \break z~pilota~lub wizualizacji \\\hline   
\end{tabular}
\end{center}
\vspace*{-6mm}
  \caption{Zmienne wejściowe do bloku FB1}
	\label{fb1datain}
\end{table}

\begin{table}[!htb]
\begin{center}
\begin{tabular}{|c|c| p{10cm} |}\hline
Nazwa & Typ & Opis  \\
zmiennej &  zmiennej &   \\\hline
EngineOnOff & Bool & Włączenie/wyłączenie                             silnika \\\hline   
EngineDirection & Bool & Kierunek pracy silnika \\\hline   
MinValue & Bool & Sygnał osiągnięcia                           minimum \\\hline   
MaxValue & Bool & Sygnał osiągnięcia                           maksimum \\\hline   
ReachPosition & Bool & Sygnał osiągnięcia                                pozycji docelowej \\\hline   
ResetFinish & Bool & Sygnał zakończenia                                resetu danego silnika \\\hline   
\end{tabular}
\end{center}
\vspace*{-6mm}
  \caption{Zmienne wyjściowe z bloku FB1}
	\label{fb1dataout}
\end{table}

Blok FB1 jest wywoływany w bloku OB1 z~odpowiednimi parametrami, niezależnie dla poszczególnych 4~silników. Przykładowe wywołanie znajduje się na Rysunku~\ref{ob1}. Na rysunku łatwo można zaobserwować, że do bloku podane są wszystkie niezbędne informacje dla danego silnika oraz 2~wspólne dla wszystkich silników sygnały informujące o~aktualnym trybie pracy modelu.
%\begin{figure}[!htb] 	\centering 	\includegraphics[width=0.58\textwidth]{obrazki/ob1.png} 	\caption{Przykładowe wywołanie FB1 w OB1 dla silnika Lift} \label{ob1} \end{figure} 

Istotnymi fragmentami bloku FB1 są 3~gałęzie programu: licznik wewnętrzny, decyzja o~kierunku oraz decyzja o~załączeniu.
Bardzo istotnym warunkiem wpływającym na pracę silnika jest jego położenie. Jeśli silnik osiąga wartość maksymalną lub minimalną to przerywa swoją pracę. Warunkiem koniecznym do określenia tej pozycji jest wykorzystanie licznika do zliczania impulsów z~modelu i~jego odpowiednia inkrementacja lub dekrementacja. Tak działający licznik przedstawia Rysunek~\ref{licznik}.
%\begin{figure}[!htb] 	\centering 	\includegraphics[width=0.5\textwidth]{obrazki/fb1licznik.png} 	\caption{Licznik wewnętrzny dla silnika w bloku FB1} \label{licznik} \end{figure} 

Decyzja dotycząca kierunku jest podejmowana zależnie od trybu pracy, co zaobserwować można na Rysunku~\ref{kierunek}. W~przypadku trybu automatycznego decydująca jest bieżąca pozycja oraz pozycja docelowa. Kierunek jest ustawiany tak, aby załączony silnik zmierzał do~pozycji docelowej. W~przypadku trybów manualnych decydujące są stany przycisków przy sterowaniu z~pilota lub zmienne przy sterowaniu z~wizualizacji. Jeżeli została podjęta decyzja o~załączeniu silnika to stan przycisku (zmiennej) jest wpisywany do zmiennej statycznej \emph{EngineDirectionManual}, która jest następnie zsumowana logicznie ze zmienną \emph{EngineDirectionAuto}, dając w~wyniku decyzję o~kierunku.
%\begin{figure}[!htb] 	\centering 	\includegraphics[width=0.66\textwidth]{obrazki/fb1kierunek.png} \caption{Fragment bloku FB1 wybierający kierunek pracy silnika} \label{kierunek} \end{figure} 

Część decydująca o~załączeniu silnika jest bardzo rozbudowana. Jest to przykład załączenia z~podtrzymaniem. Zależnie od trybu pracy sprawdzane są inne warunki. W~przypadku trybu automatycznego ważna jest zmienna \emph{Enable}, która decyduje o~załączeniu wybranego silnika, co pozwala na częściową pracę sekwencyjną. Silnik pracuje w~kierunku krańcówki aż do~jej osiągnięcia, a w~kierunku od krańcówki aż~do osiągnięcia swojego maksimum, chyba że~wcześniej zostanie osiągnięta pozycja docelowa. W~ręcznych trybach pracy głównym warunkiem decydującym o~załączeniu zależnie od kierunku są wartość maksymalna licznika lub krańcówka. Wspólnym dla obu trybów warunkiem pozwalającym zatrzymać pracę silnika jest ustawienie zmiennej \emph{EmergencyStop} na wartość \emph{true}. Tą rozbudowaną i skomplikowaną gałąź prezentuje Rysunek~\ref{onoff}.
%\begin{figure}[!htb] 	\centering 	\includegraphics[width=0.95\textwidth]{obrazki/fb1wlwyl.png} 	\caption{Fragment bloku FB1 decydujący o załączeniu silnika} \label{onoff} \end{figure} 

\subsubsection{Blok FB8}
W trybie automatycznym podczas obsługi magazynu praca odbywa się na zasadzie zadań do wykonania. Na zadanie takie składa się indeks komórki w magazynie oraz zmienna typu bool, która decyduje o ruchu chwytaka (zaciskanie lub zwalnianie). Indeks jest następnie przetwarzany w odpowiednim bloku na położenie komórki w modelu magazynu. 

Bardzo ważnym blokiem funkcyjnym jest blok zarządzający kolejką zadań do wykonania w trybie automatycznym. Na wejściu znajdują się: dodawany do kolejki indeks oraz zmienna decydująca o zamknięciu lub otwarciu chwytaka. Na wyjściu mamy zmienne takie same jak na wejściu, ale przepuszczone już przez kolejkę. Zmienne wejściowo-wyjściowe to 2~flagi żądania, odpowiednio: dodania do kolejki lub pobrania z niej elementu oraz 2~flagi potwierdzające wykonanie żądania. Dodatkowo w~bloku tym występuje zmienna wewnętrzna stanowiąca indeks w~tablicy. 
\newpage
\begin{lstlisting}[caption={FB8 - Zarządzanie kolejką}]
FUNCTION_BLOCK FB8

VAR_INPUT
    AddedBool: BOOL;
    AddedIndex: INT;
END_VAR     

VAR_OUTPUT 
    WarehouseBool: BOOL;
    WarehouseIndex: INT;
END_VAR  

VAR_IN_OUT
    Request: BOOL;
    Add: BOOL;
    RequestAck: BOOL;
    AddAck: BOOL;
END_VAR  

VAR
    Index: INT ;
END_VAR

BEGIN
IF Add = true & Queue.CurentIndex <= 100 THEN
    IF AddedIndex >= 0 AND AddedIndex < 26 THEN
        Queue.QueueIndex[Queue.CurentIndex] := AddedIndex;
        Queue.QueueBool[Queue.CurentIndex] := AddedBool;
        Queue.CurentIndex := Queue.CurentIndex + 1;
        Add := false;
        AddAck := true;
    ELSE
        Error.IndexError := 'Indeks spoza zakresu';    
        Error.IndexErrorPres := true;
    END_IF;
ELSIF Queue.CurentIndex >= 100 THEN   
    Error.QueueError := 'Przepelniona Kolejka';
    Error.QueueErrorPres := true;    
END_IF;

IF Request = true & Queue.CurentIndex > 0 THEN
    IF Queue.QueueIndex[0] <> -1 THEN        
        WarehouseIndex := Queue.QueueIndex[0];
        WarehouseBool := Queue.QueueBool[0];
            FOR Index := 0 TO Queue.CurentIndex DO
                Queue.QueueIndex[Index] := Queue.QueueIndex[Index+1];
                Queue.QueueBool[Index] := Queue.QueueBool[Index+1]; 
            END_FOR;        
        Queue.QueueIndex[Queue.CurentIndex] := 0;
        Queue.QueueBool[Queue.CurentIndex] := false;
        Queue.CurentIndex := Queue.CurentIndex - 1;
        Request := false;
        RequestAck := true;
    ELSE
        Error.IndexError := 'Bledny indeks';    
        Error.IndexErrorPres := true;            
    END_IF;    
ELSIF Queue.CurentIndex = 0 THEN
    Request := false;   
    RequestAck := false;     
    Error.QueueError := 'Kolejka jest pusta';    
    Error.QueueErrorPres := true;
END_IF;
END_FUNCTION_BLOCK
\end{lstlisting}
\subsubsection{Blok FB9}
Kolejnym istotnym blokiem jest blok decydujący o~zezwoleniu na pracę poszczególnych silników, pozwalając przez to wykonywać ruchy ramienia w~odpowiedniej kolejności. Blok na wyjściu posiada 4~zmienne stanowiące właściwe zezwolenie na ruch. Decyzja jest podejmowana na podstawie położenia silnika Arm oraz osiągnięcia przez poszczególne silniki swoich pozycji docelowych.
\begin{lstlisting}[caption={FB9 - Zezwolenie na ruch}]
FUNCTION_BLOCK FB9

VAR_INPUT
    ArmCounterl: COUNTER;
END_VAR     

VAR_OUTPUT
    ArmEn: BOOL;
    LiftEn: BOOL;
    RotateEn: BOOL;
    GrabEn: BOOL;            
END_VAR     

BEGIN
    IF AutoTest THEN
        ArmEn := true;
        LiftEn := true;
        RotateEn := true;
        GrabEn := true;
    ELSE     
        IF ArmData.CounterValue_I <= 25 THEN
            ArmEn := true;
            LiftEn := true;
            RotateEn := true;
            GrabEn := false;
        ELSE
            IF LiftPositionReach  AND RotatePositionReach THEN        
                ArmEn := true;   
                LiftEn := false;
                RotateEn := false;
                GrabEn := false;  
            ELSE
                IF ArmData.CounterValue_I > 25 THEN
                    ArmEn := true;
                    LiftEn := false;
                    RotateEn := false;
                    GrabEn := false;
                else
                    ArmEn := false;   
                    LiftEn := true;
                    RotateEn := true;
                    GrabEn := false;              
                END_IF;
            END_IF;    
        END_IF;        
        IF LiftPositionReach & RotatePositionReach & ArmPositionReach THEN
            ArmEn := false;   
            LiftEn := false;
            RotateEn := false;
            GrabEn := true;    
        END_IF;    
    END_IF;           
END_FUNCTION_BLOCK
\end{lstlisting}

\subsubsection{Blok FB10}
Blok funkcyjny FB10 zarządza operacjami wykonywanymi na tablicach związanych z~obsługiwanym magazynem. Na~wejściu mamy zmienne stanowiące wyjście z~kolejki oraz zmienną określającą, że~wykonane zostało ostatnie podzadanie uruchomione przez ten blok. Na~wyjściu mamy zmienne określające, do jakich pozycji mają dojechać silniki. 

Zmienne wejściowo-wyjściowe to żądanie następnego zadania z~kolejki oraz potwierdzenie obsłużenia żądania przez kolejkę. Zmienna tymczasowa jest to wartość zwrócona przez funkcję systemową odczytu bieżącej daty oraz godziny. Zmienna wewnętrzna \emph{DoneCount} przechowuje informację o tym, ile podzadań zostało wykonanych. 
\begin{lstlisting}[caption={FB10 - Operacje na tablicah}]
FUNCTION_BLOCK FB10

VAR_INPUT
    Done: BOOL;
    WarehouseIndex: INT;
    WarehouseBool: BOOL;
END_VAR 

VAR_OUTPUT
    ArmTo: INT;
    LiftTo: INT;
    RotateTo: INT;
    GrabTo: INT;            
END_VAR

VAR_IN_OUT
    NextRequest: BOOL;
    NextRequestACK: BOOL;
END_VAR  
 
VAR_TEMP
    SFC1_Ret_val: INT;    
END_VAR

VAR
    DoneCount: INT;
    LoadNext: BOOL;
END_VAR
    
BEGIN

IF NextRequestACK = true THEN
    NextRequest := false;
    NextRequestACK := false;
    DoneCount := 0;    
    ArmTo := 22;
    IF WarehouseBool THEN
        LiftTo := Warehouse.WarehousePosition[WarehouseIndex, 2];    
    ELSE 
        LiftTo := Warehouse.WarehousePosition[WarehouseIndex, 2]-3;
    END_IF;  
    RotateTo := Warehouse.WarehousePosition[WarehouseIndex, 3];
    LoadNext := true;
END_IF;

IF DoneCount = 1 THEN
    ArmTo := Warehouse.WarehousePosition[WarehouseIndex, 1];
    IF WarehouseBool THEN
        LiftTo := Warehouse.WarehousePosition[WarehouseIndex, 2];    
    ELSE 
        LiftTo := Warehouse.WarehousePosition[WarehouseIndex, 2]-3;
    END_IF;        
    RotateTo := Warehouse.WarehousePosition[WarehouseIndex, 3];
    GrabTo := BOOL_TO_INT(WarehouseBool) * 19;
END_IF;
    
IF DoneCount = 2 THEN
    Warehouse.WarehouseBool[WarehouseIndex] := NOT WarehouseBool;
    SFC1_Ret_val := SFC1(CDT := Warehouse.WarehouseDateTime[WarehouseIndex]);
    DoneCount := 0;
    IF Queue.CurentIndex > 0 THEN
        NextRequest := true;
        LoadNext := false;
    ELSE
        Error.QueueError := 'Kolejka jest pusta';  
    END_IF;
END_IF;

IF Done = true & LoadNext THEN
    DoneCount := DoneCount + 1;    
END_IF;       
END_FUNCTION_BLOCK
\end{lstlisting}
\subsubsection{Blok FB11}
Blok FB11 w wyniku działania zwraca indeks komórki w magazynie oraz powiązaną z~nim zmienną typu bool, które pozwalają wybrać najdalej lub najbliżej położony element. Wskazuje on pustą lub zajętą komórkę magazynu, zależnie od zmiennych wejściowych. Zależnie od wartości zmiennej \emph{FarNear} wybierana jest najdalej lub najbliżej położona komórka, natomiast zmienna \emph{BoolGrab} decyduje o~tym, czy szukamy wolnej czy zajętej komórki magazynu. 

W wyniku działania blok ustawia na wyjściu zmienne związane z~wybraną komórką magazynu oraz flagę żądania dodania uzyskanego wyniku do kolejki zadań. Zmienna \emph{MyEn} decyduje o tym, czy ten blok zostaje aktywowany czy nie. Zmienna index jest zmienną wewnętrzną wykorzystywaną w~pętlach FOR.
\begin{lstlisting}[caption={FB11 - Funkcja wybiera najdalszą lub najbliższą komórkę}]
FUNCTION_BLOCK FB11

VAR_INPUT
    FarNear: BOOL;
END_VAR 

VAR_OUTPUT
    IndexSel: INT;
    AddQueue: BOOL;
END_VAR
 
VAR_IN_OUT
    MyEn: BOOL;
    BoolGrab: BOOL;
END_VAR  

VAR
    Index: INT;
END_VAR    
    
BEGIN
IF MyEn = true THEN    
    IF FarNear = true & BoolGrab = true THEN    
        FOR Index := 1 TO 24 DO
            IF Warehouse.WarehouseBool[Index] = true THEN
                IndexSel := Index;
                EXIT;
            ELSE
                IndexSel := -1;                
            END_IF;
        END_FOR;        
    END_IF;
    
    IF FarNear = false & BoolGrab = true THEN
        FOR Index := 24 TO 1 BY -1 DO
            IF Warehouse.WarehouseBool[Index] = true THEN
                IndexSel := Index;
                EXIT;
            ELSE
                IndexSel := -1;                                
            END_IF;            
        END_FOR;
    END_IF;
    
    IF FarNear = true & BoolGrab = false THEN
        FOR Index := 1 TO 24 DO
            IF Warehouse.WarehouseBool[Index] = false THEN
                IndexSel := Index;
                EXIT;
            ELSE
                IndexSel := -1;                                                
            END_IF;            
        END_FOR;
    END_IF;
    
    IF FarNear = false & BoolGrab = false THEN
        FOR Index := 24 TO 1  BY -1 DO
            IF Warehouse.WarehouseBool[Index] = false THEN
                IndexSel := Index;
                EXIT;
            ELSE
                IndexSel := -1;                                                
            END_IF;            
        END_FOR;
    END_IF;   
    MyEn := false;
    AddQueue := true;
END_IF;       
END_FUNCTION_BLOCK
\end{lstlisting}
\subsubsection{Blok FB12}
Blok FB12 w wyniku działania zwraca indeks komórki w magazynie oraz powiązaną z~nim zmienną typu bool, które pozwalają wybrać najstarszy lub najmłodszy element tablicy. Wskazuje on zajętą komórkę magazynu. Zależnie od wartości zmiennej \emph{SmallestGreatest} wybierana jest najmłodsza lub najstarsza zajęta komórka. 

W wyniku działania blok ustawia na wyjściu zmienne związane z~wybraną komórką magazynu oraz flagę żądania dodania uzyskanego wyniku do kolejki zadań. Zmienna \emph{MyEn} decyduje o~tym, czy ten blok zostaje aktywowany czy nie. Zmienna index jest zmienną wewnętrzną wykorzystywaną w~pętlach FOR. \emph{IndexTemp} jest zmienną pomocniczą stanowiącą tymczasowy indeks przy wybieraniu elementu końcowego.
\begin{lstlisting}[caption={FB12 - Funkcja wybiera najmłodszą lub najstarszą zajętą komórkę}]
FUNCTION_BLOCK FB12

VAR_INPUT
    SmallestGreatest: BOOL;
END_VAR 

VAR_OUTPUT
    IndexSelected: INT;
    BoolSelected: BOOL;
    AddQueue: BOOL;    
END_VAR
 
VAR_IN_OUT
    MyEn: BOOL;
END_VAR  
 
VAR
    Index: INT;
END_VAR
    
VAR_TEMP
    IndexTemp: INT;
END_VAR 
       
BEGIN
IF MyEn = true THEN    
    
    IF SmallestGreatest = false THEN    
        FOR Index := 1 TO 24 DO        
            IF Warehouse.WarehouseBool[Index] = true AND FC28(DT1 := Warehouse.WarehouseDateTime[Index], DT2 := DT#1990-01-01-12:00:00.00) THEN
                IndexTemp := Index;
                EXIT;
            ELSE
                IndexTemp := -1;
            END_IF;
        END_FOR;  
        IF IndexTemp <> -1 THEN                   
            FOR Index := 1 TO 24 DO        
                IF Warehouse.WarehouseBool[Index] = true then
                    IF FC14(DT1 := Warehouse.WarehouseDateTime[Index], DT2 := Warehouse.WarehouseDateTime[IndexTemp]) THEN
                        IndexTemp := Index;            
                    END_IF;
                END_IF;
            END_FOR;        
        END_IF;           
    END_IF;
    
    IF SmallestGreatest = true THEN
        FOR Index := 1 TO 24 DO        
            IF Warehouse.WarehouseBool[Index] = true AND FC28(DT1 := Warehouse.WarehouseDateTime[Index], DT2 := DT#1990-01-01-12:00:00.00) THEN
                IndexTemp := Index;
                EXIT;
            ELSE
                IndexTemp := -1;
            END_IF;
        END_FOR;      
        IF IndexTemp <> -1 THEN                   
            FOR Index := 1 TO 24 DO
                IF Warehouse.WarehouseBool[index] = true then        
                    IF FC23(DT1 := Warehouse.WarehouseDateTime[Index], DT2 := Warehouse.WarehouseDateTime[IndexTemp]) THEN
                        IndexTemp := Index;
                    END_IF;            
                END_IF;            
            END_FOR;
        END_IF;            
    END_IF;
    MyEn := false;
    IndexSelected := IndexTemp;
    BoolSelected := true;
    AddQueue := true;
END_IF;    
END_FUNCTION_BLOCK
\end{lstlisting}
 %
%\lstset{language=VBScript,
        basicstyle=\footnotesize\ttfamily,
        breaklines=true,
        tabsize=2,
        numbers=left,
        numberstyle=\tiny,
        numbersep=7pt,
        showspaces=false,
        keywordstyle=\color{Blue}\textbf,
        commentstyle=\color{Red}\emph,
        showstringspaces=false,
        stringstyle=\color{BurntOrange}
        }
\section{Wizualizacja HMI}
Zaimplementowana przez autora wizualizacja ma na~celu zobrazowanie działania modelu oraz umożliwienie operatorowi wpływania na~jego działanie. Kolejne podrozdziały zawierają opis specyfikacji zewnętrznej oraz wewnętrznej. Część odnosząca się do specyfikacji zewnętrznej jest skróconą instrukcją obsługi użytkownika. Specyfikacja wewnętrzna jest opisem, jak zostały zrealizowane poszczególne elementy~i~w jaki sposób wizualizacja współpracuje ze~sterownikiem.

\subsection{Specyfikacja zewnętrzna}
Specyfikacja zewnętrzna przedstawiona w dalszej części podrozdziału stanowi skróconą instrukcję obsługi wizualizacji oraz opis możliwości oferowanych przez poszczególne ekrany.

Autor projektu wykorzystał w~swojej pracy szereg elementów dostępnych standardowo w~środowisku Simatic WinCC flexible. Podstawowymi elementami sterującymi są przyciski w~trybie tekstowym oraz przeźroczystym. Głównymi obiektami służącymi do prezentacji informacji są: pola tekstowe, pola wejściowo-wyjściowe oraz pola daty i~godziny. Dodatkowo celem uatrakcyjnienia wizualizacji wykorzystane zostały suwaki (ang. \emph{slider}), obrazki oraz zegarek. 

Obsługa tej części projektu jest realizowana za pomocą myszy i~klawiatury podłączonych do~komputera. Za~pomocą klawiatury wybieramy interesujący nas ekran lub wprowadzamy żądaną wartość pozycji docelowej na~ekranie testowania trybu automatycznego.

\subsection{Specyfikacja wewnętrzna}
Wizualizacja komunikuje się z~komputera klasy~PC ze~sterownikiem za~pośrednictwem protokołu Ethernet w~sieci lokalnej.
Odniesienia do~odpowiednich adresów w~pamięci sterownika dokonywane są za~pomocą nazw symbolicznych zdefiniowanych w~tablicy Tags. Do działania wizualizacja używa tylko jednej zmiennej wewnętrznej i~jest~to zmienna tablicowa \emph{MagazynDTEnable} z elementami typu bool. Elementy te odpowiadają za~wyświetlanie dat oraz godzin na~ekranie ze~stanem magazynu po~kliknięciu na~wybraną komórkę. Obsługa wyświetlania dat polega na tym, że po kliknięciu w wybrane pole ustawiana jest odpowiednia zmienna w~tej tablicy na~wartośc \emph{true}, a~po zwolnieniu klawisza myszki na wartość \emph{false}. Za zmiany te odpowiadają niewidzialne przyciski umieszczone na~tych polach.

Wizualizacja wpływa na pracę sterownika poprzez zmianę pojedynczych bitów za~pomocą umieszczonych na~ekranie przycisków. Wpływa ona również poprzez modyfikowanie wybranych zmiennych odpowiadających pozycjom docelowym lub poprzez dodawanie odpowiednich zadań do~kolejki. Bardziej zaawansowane operacje zostały zrealizowane za~pomocą skryptów napisanych w~języku VBScript, które są bardzo prostą i~szybką opcją wykonywania bardziej zaawansowanych czynności.  %
\section{Badania}
W~niniejszym rozdziale opisany został przebieg przeprowadzonych badań oraz analiza ich wyników. W~kolejnych podrozdziałach zostaną przedstawione kolejne różne eksperymenty. Dokładne otrzymane wyniki znajdują się na~dołączonej do pracy płycie~CD w~odpowiednich plikach, o~nazwach zgodnych z~podrozdziałem opisującym dane badanie i~jego wyniki.

%\subsection{Opóźnienia pojedynczego odcinka sieci}
%\subsection{Wpływ topologi na opóźnienia}
\subsection{Czas stabilizacji sieci po odłączeniu i~podłączeniu dowolnego urządzenia}
Badanie miało na~celu sprawdzenie czy zgodnie z~zapewnieniami producenta możliwe jest rozłączanie i~ponowne podłączanie urządzeń do~sieci ,,w locie'' bez konieczności żadnej dodatkowej ingerencji. Do przeprowadzenia pomiarów autor odłączał przewód ethernetowy i~podłączał go~ponownie do~gniazda. Oczywiście taka metoda może mieć wpływ na~dokładność dokonywanych pomiarów, ale zdaniem autora jest on stosunkowo niewielki, ze względu na fakt, że od~momentu odłączenia do wykrycia tego zajścia przez urządzenie nadzorujące pracę sieci mija pewien czas, który w optymistycznym przypadku może być krótszy niż sam proces ingerencji autora. Aby poprawić dokładność należałoby zastosować specjalnie stworzone samodzielnie urządzenie, co zostanie opisane w~perspektywach rozwoju niniejszej pracy. Wszystkie zmierzone czasy oraz treści komunikatów zostały odczytane w~środowisku TwinCAT System Manager, przykładowy został przedstawiony na Rysunku~\ref{reading_time}.

\begin{figure}[!htb] 	\centering 	\includegraphics[width=0.95\textwidth]{images/reading_time} \caption{Topologia stanowiska z odłączonym jednym węzłem} \label{reading_time} \end{figure}
\vspace{-5mm}
\subsubsection{Pojedyncze urządzenie}
Eksperyment miał na~celu sprawdzenie po upływie jakiego czasu oo~odłączenia i~ponownego podłączenia pojedynczego węzła sieci zaczyna on~znów funkcjonować poprawnie. Zaburzoną pracę sieci przedstawiono na~Rysunku~\ref{one_slave}, który został wygenerowany przy użyciu środowiska TwinCAT System Manager, które udostępnia możliwość podglądu topologii sieci w~czasie rzeczywistym. Do eksperymentu wybrany został węzeł końcowy w~topologii.
\begin{figure}[!htb] 	\centering 	\includegraphics[width=0.9\textwidth]{images/topologyCXerror} \caption{Topologia stanowiska z odłączonym jednym węzłem} \label{one_slave} \end{figure}

\begin{table}[!htb]
\begin{center}
\begin{tabular}{| c | c | c | c | c |}\hline
\textbf{Liczba} & \textbf{Wartość} & \textbf{Wartość} & \textbf{Wartość} & \textbf{Odchylenie} \\
\textbf{próbek} & \textbf{średnia} & \textbf{minimalna} & \textbf{maksymalna} & \textbf{standardowe} \\\hline\hline
20 & 2,715s & 2,644s & 3,184s & 0,131\\\hline
\end{tabular}
\end{center}
\vspace*{-6mm}
  \caption{Wyniki przeprowadzonego badania}
	\label{badania:wyniki:stabilizacja_jeden}
\end{table}

\noindent Wyniki przeprowadzonego eksperymentu zostały zebrane w Tabeli~\ref{badania:wyniki:stabilizacja_jeden}. Na podstawie zmierzonych wartości oraz po ich analizie statystycznej uzyskano wykres przedstawiony na Rysunku~\ref{badania:wykres:stabilizacja_jeden}
\begin{figure}[htbp]
 \centering
 \begin{tikzpicture}[x=0.5cm,y=10cm]
  \tikzstyle{background grid}=[draw, black!50,step=.25cm]
	\draw[-latex, thin, draw=gray] (0,2.5)--(20,2.5) node [right] {$x$};
	\draw[-latex, thin, draw=gray] (0,2.45)--(0,3.5) node [above] {$t[s]$};
	%\draw [dotted, gray, step=0.5cm] (0,0) grid (20,4);
 	\draw (0,2.5) node[left] {2,5};
	\draw[] (0,2.715)--(20,2.715) node[left=10cm] {2,715};
	\draw[thin, dotted] (0,2.644)--(20,2.644) node[below] {min};
	\draw[thin, dotted] (0,3.184)--(20,3.184) node[above] {max};
		
	\node at (1, 2.664) {\textbullet};
	\node at (2, 2.664) {\textbullet};
	\node at (3, 2.654) {\textbullet};
	\node at (4, 2.664) {\textbullet};
	\node at (5, 2.654) {\textbullet};
	\node at (6, 2.774) {\textbullet};
	\node at (7, 2.764) {\textbullet};
	\node at (8, 2.664) {\textbullet};		
	\node at (9, 2.654) {\textbullet};
	\node at (10, 2.664) {\textbullet};
	\node at (11, 2.654) {\textbullet};
	\node at (12, 2.664) {\textbullet};
	\node at (13, 2.654) {\textbullet};
	\node at (14, 2.644) {\textbullet};
	\node at (15, 2.664) {\textbullet};
	\node at (16, 2.664) {\textbullet};							
	\node at (17, 2.654) {\textbullet};
	\node at (18, 2.934) {\textbullet};
	\node at (19, 2.764) {\textbullet};
	\node at (20, 3.184) {\textbullet};
					
\end{tikzpicture}
\caption{Pomiary czasu ponownego podłączenia oraz obliczona wartość średnia}
\label{badania:wykres:stabilizacja_jeden}
\end{figure}

Analiza wyników przeprowadzonego badania potwierdza, że istnieje możliwość odłączenie i~ponowne podłączenie pojedynczego węzła sieci ,,w locie''. Czas potrzebny na~stabilizację połączenia po jego utracie jest zdaniem autora zadowalający. Obserwując uzyskany wykres można dojść do wniosku, że różnice pomiędzy kolejnymi iteracjami eksperymentu są stosunkowo małe, o czym świadczy odchylenie standardowe, którego wartość wynosi 0,131. Zdaniem autora głównym źródłem różnic jest zastosowana metoda pomiarowa.

\subsubsection{Wyspa z modułami I/O}
Eksperyment miał na~celu sprawdzenie po upływie jakiego czasu od~odłączeniu i~ponownego podłączenia zdalna wyspa z~dołączonymi kilkoma modułami wejścia/wyjścia zaczyna w~całości działać poprawnie tzn. wszystkie węzły nawiązują komunikację. Zaburzoną pracę sieci przedstawiono na~Rysunku~\ref{coupler}.
\begin{figure}[!htb] 	\centering 	\includegraphics[width=0.8\textwidth]{images/topologyCPerror} \caption{Topologia stanowiska z odłączoną wyspą} \label{coupler} \end{figure}
%1,5s wyspa i każdy kolejny moduł I/O z opóźnieniem 4ms

Wyniki przeprowadzonego eksperymentu zostały zebrane w Tabeli~\ref{badania:wyniki:stabilizacja_wyspa}.
\begin{table}[!htb]
\begin{center}
\begin{tabular}{| c | c | c | c | c |}\hline
\textbf{Liczba} & \textbf{Wartość} & \textbf{Wartość} & \textbf{Wartość} & \textbf{Odchylenie} \\
\textbf{próbek} & \textbf{średnia} & \textbf{minimalna} & \textbf{maksymalna} & \textbf{standardowe} \\\hline\hline
42 & 2,699s & 2,634s & 2,812s & 0,061s \\\hline
\end{tabular}
\end{center}
\vspace*{-6mm}
  \caption{Wyniki przeprowadzonego badania}
	\label{badania:wyniki:stabilizacja_wyspa}
\end{table}

Na podstawie zmierzonych wartości oraz po ich analizie statystycznej uzyskano wykres przedstawiony na Rysunku~\ref{badania:wykres:stabilizacja_wyspa}. Kolejnymi kolorami są oznaczone kolejne iteracje badania, a~punkty tego samego koloru oznaczają kolejne węzły zdalnej stacji wejść/wyjść.
\begin{figure}[htbp]
 \centering
 \begin{tikzpicture}[x=1cm,y=5cm]


 \draw[latex-latex, thin, draw=gray] (0,0)--(10,0) node [right] {$x$}; % l'axe des abscisses
 \draw[latex-latex, thin, draw=gray] (0,0)--(0,2) node [above] {$t[s]$}; % l'axe des ordonnées
 \draw[thick] (0,1.5)--(10,1.54); % l'axe des abscisses
  \draw[thick,red] (0,1.6)--(10,1.64); % l'axe des abscisses

    \foreach \i in {0,...,10}{% 
\foreach \Point in {(\i ,1.5+0.004*\i)}{
    \node at \Point {\textbullet}; } ;}     

    \foreach \i in {0,...,10}{% 
\foreach \Point in {(\i ,1.6+0.004*\i)}{
    \node[red] at \Point {\textbullet}; } ;}
    
% to ensure that the points are being properly centered:
\draw [dotted, gray] (0,0) grid (10,2);

\end{tikzpicture}
\caption{Pomiary czasu ponownego podłączenia oraz obliczona wartość średnia}
\label{badania:wykres:stabilizacja_wyspa}
\end{figure}

Spoglądając na uzyskane wyniki oraz wykres można zaobserwować, że zwiększenie liczby odłączanych i~podłączanych ponownie węzłów nie~wpłynęło na~możliwość dokonywania tej operacji w~czasie pracy sieci. Ciekawa zdaniem autora jest różnica czasu pomiędzy podłączeniem się pierwszego urządzenia z~zestawu, a~każdym kolejnym wynoszący 2 lub milisekundy. Tak więc liczba modułów wejścia/wyjścia ma~proporcjonalnie niewielki wpływ na czas potrzebny do ustabilizowania się całego zdalnego zestawu.

\subsubsection{Wszystkie węzły sieci}
\label{badania:cala_siec}
Eksperyment miał na~celu sprawdzenie po upływie jakiego czasu od~odłączeniu i~ponownego podłączenia wszystkich węzłów sieć wróci ona do~normalnego stanu. Zaburzoną pracę sieci przedstawiono na~Rysunku~\ref{topologyCPallerror}.
\begin{figure}[!htb] 	\centering 	\includegraphics[width=0.9\textwidth]{images/topologyCPallerror} \caption{Topologia stanowiska z odłączonymi wszystkimi węzłami} \label{topologyCPallerror} \end{figure}

\noindent Stan sieci bezpośrednio po ponownym podłączeniu kabla sieciowego i~w~fazie inicjalizacji ponownego połączenia przedstawiono na Rysunku~\ref{topologyCPallloading}. Można zaobserwować, że węzły połączone w~zdalną stację wejść/wyjść są już w~jednej z~kolejnych faz inicjalizacji, a~pozostałe jeszcze nie rozpoczęły.
\begin{figure}[!htb] 	\centering 	\includegraphics[width=0.9\textwidth]{images/topologyCPallloading} \caption{Topologia stanowiska w czasie ponownego podłączania węzłów} \label{topologyCPallloading} \end{figure}

\begin{table}[!htb]
\begin{center}
\begin{tabular}{| c | c | c | c | c |}\hline
\textbf{Liczba} & \textbf{Wartość} & \textbf{Wartość} & \textbf{Wartość} & \textbf{Odchylenie} \\
\textbf{próbek} & \textbf{średnia} & \textbf{minimalna} & \textbf{maksymalna} & \textbf{standardowe} \\\hline\hline
10 & 4,790s & 4,149s & 5,455s & 0,446s\\\hline
\end{tabular}
\end{center}
\vspace*{-6mm}
  \caption{Wyniki przeprowadzonego badania}
	\label{badania:wyniki:stabilizacja_siec}
\end{table}

\noindent Wyniki przeprowadzonego eksperymentu zostały zebrane w Tabeli~\ref{badania:wyniki:stabilizacja_jeden}. Na podstawie zmierzonych wartości oraz po ich analizie statystycznej uzyskano wykres przedstawiony na Rysunku~\ref{badania:wykres:stabilizacja_siec}
\begin{figure}[h]
 \centering
 \begin{tikzpicture}[x=1cm,y=5cm]
  \tikzstyle{background grid}=[draw, black!50,step=.25cm]
	\draw[-latex, thin, draw=gray] (0,4)--(10,4) node [right] {$x$};
	\draw[-latex, thin, draw=gray] (0,3.95)--(0,5.55) node [left] {$t[s]$};
	%\draw [dotted, gray, step=0.5cm] (0,0) grid (20,4);
 	\draw (0,4) node[left] {4};
	\draw[] (0,4.790)--(10,4.790) node[left=10cm] {4,790};
	\draw[thin, dotted] (0,4.149)--(10,4.149) node[below] {min};
	\draw[thin, dotted] (0,5.455)--(10,5.455) node[above] {max};
		
	\node at (1, 4.863) {\textbullet};
	\node at (2, 4.149) {\textbullet};
	\node at (3, 5.313) {\textbullet};
	\node at (4, 5.455) {\textbullet};
	\node at (5, 4.339) {\textbullet};
	\node at (6, 4.911) {\textbullet};
	\node at (7, 5.099) {\textbullet};
	\node at (8, 4.339) {\textbullet};		
	\node at (9, 4.967) {\textbullet};
	\node at (10, 4.463) {\textbullet};
					
\end{tikzpicture}
\caption{Pomiary czasu ponownego podłączenia oraz obliczona wartość średnia.}
\label{badania:wykres:stabilizacja_siec}
\end{figure}

Analiza wyników przeprowadzonego badania pozwala wysnuć wniosek, że możliwe jest całkowite rozłączenie i~ponowne podłączenie sieci ,,w locie''. Czas potrzebny na~stabilizację połączenia po jego utracie również w tym wypadku jest zdaniem autora zadowalający. Niestety jednak porównując uzyskane wielkości do~tych z~badania pojedynczego węzła sieci można zauważyć, że czas ten wzrósł dość znacząco i~w~przypadku jeszcze większej liczby węzłów przestanie on już być akceptowalny. Obserwując uzyskany wykres można dojść do wniosku, że różnice pomiędzy kolejnymi iteracjami eksperymentu są stosunkowo małe i~wpływ na~nie może mieć w~większości zastosowana metoda pomiarowa.

\clearpage
\subsubsection{Problemy}
W~czasie dziesiątek przeprowadzonych w~tym badaniu pomiarów autor doprowadził do sytuacji w~której sieć nie~powróciła już samoczynnie do~prawidłowego działania. Jest to sprzeczne z~zapewnieniami twórców standardu i~dlatego wszystkie przypadki zostaną tu zawarte,szczegółowo opisane i~przeanalizowane.

\begin{enumerate}
\item Treść wiadomości odczytanej ze środowiska była następująca: \textit{Device 2 (EtherCAT (v2.10 only)': 'INIT to PREOP' failed! Error: 'read slave count'. Communication Error '0x707 (1799)'.} \\[1mm]
%ADS, problem ze stanem urządzenia
Do błędu doszło w~momencie zaburzenia pracy całej sieci, tj. odłączenia wszystkich węzłów podrzędnych od węzła nadrzędnego jak w~badaniu~\ref{badania:cala_siec}. Stan sieci po~wystąpieniu błędu przedstawiono na Rysunku~\ref{err0x707}.
\begin{figure}[!htb] 	\centering 	\includegraphics[width=0.9\textwidth]{images/err0x707} \caption{Stan sieci po błędzie 0x707} \label{err0x707} \end{figure}

Na rysunku można zaobserwować, że urządzenia są podłączone do sieci, ale ich stan jest błędny (brak czerwony ramek wokół węzłów). Analizując stan stanowiska i~sieci w~środowisku zaobserwowano, że w~sieci nie są przesyłane dane.
Dalsza analiza wykazała, że węzeł AX-5203 jest w prawidłowym stanie inicjalizacji (0x0001), natomiast problematyczna okazała się zdalna wyspa, która ma nieprawidłowy stan 0x5C01, który według jednej z dokumentacji producenta oznacz błąd podczas resetowania stanu \cite{err0x707}.
Okazało się, ze problem ten można prosto rozwiązać bez potrzeby restartowania całego stanowiska poprzez wymuszenie zmiany stanu węzła nadrzędnego z~INIT na~OP (opcja dostępna z~poziomu środowiska), co skutkuje wysłaniem takiego samego żądania do wszystkich węzłów podrzędnych i sieć ponownie zaczyna funkcjonować. Wynik analizy znajduje odzwierciedlenie bezpośrednio w~treści wiadomości opisującej błąd.
\end{enumerate}

\subsection{Badania niewykonalne}

\subsubsection{Zbadanie innych topologii}

\subsubsection{Zbadanie opóźnień na poziomie transmisji pojedynczych ramek}
%Różne kable
%Długość kabla
%
%Połączyć do jednego sterownika oba napędy kolejno i zrobić coś na zasadzie inkrementacji i sprawdzić czy się przypadkiem nie rozjedzie
%
%Mamy opóźnienie na jednym odcinku
%
%Ewentualnie jeden kabel można zamienić na dłuższy i sprawdzić czy nie ma różnicy.
%
%Wymyślić jak sprawdzić czas ponownego włączenia do sieci. %
\section{Uruchamianie i testowanie}
W rozdziale zawarto podsumowanie przebiegu prac nad projektem. Opisane zostaną tu~problemy, które wystąpiły w~czasie realizacji projektu. Ponadto zawarto~tu opis przebiegu procesu testowania.

%\subsection{Przebieg testowania}
%W procesie weryfikacji poprawności działania projektu zastosowano testowanie wstępujące. 
%
%Głównym testerem był autor projektu więc większość testów przebiegała na zasadzie białej skrzynki (ang. \emph{white box}), bardzo często z~użyciem podglądu stanu w środowisku TwinCAT System Manager. Takie testowanie pozwala stosunkowo łatwo wyszukać źródło błędu i~je wyeliminować.
%
%Autor kilka razy przeprowadzał testy stosując metodę czarnej skrzynki (ang. \emph{black box}), nie~biorąc pod uwagę zależności wykonywanych czynności, od~realizowanego przez sterownik kodu. Kilkukrotnie w czasie realizacji projektu do testów zgłaszały się osoby trzecie, które były nim zaciekawione. Testy wykonane przez takie osoby są niezwykle cenne ze względu na dużą nieprzewidywalność oraz całkowitą niezależność działań od rozwiązań ze~względu na~brak ich znajomości.
%
%W~czasie realizacji autor stosował testowanie oparte na dwóch metodach analizy. Testowanie oprogramowania można wykonywać pod kątem analizy statycznej i~dynamicznej. Analiza statyczna polega na~sprawdzaniu kodu źródłowego i~znajdowaniu w~nim błędów bez uruchamiania sprawdzanego kodu. Ta~metoda była stosowana poza laboratorium, gdzie brak był dostępu do sterownika i~modelu. Podczas analizy dynamicznej oprogramowanie jest uruchamiane i~badane pod kątem ścieżki przebiegu i~czasu wykonywania. Ta metoda z~kolei była najważniejsza i~często wyniki tych testów były zaskakujące w~stosunku do~przeprowadzonych wcześniej z~zastosowaniem analizy statycznej.
%
%Ostatnim etapem testów były~te przeprowadzone w~obecności promotora oraz te wykonane przez niego. Ostatecznie oprogramowanie zostało zatwierdzone i~uznane za spełniające wszystkie wstępne założenia przedstawione w~podrozdziale 1.2.
%\newpage
\subsection{Napotkane problemy}
\label{subsec:problemy}
Podczas tworzenia projektu napotkane i~przeanalizowane zostały następujące problemy:
\begin{itemize}
\item Problem z automatycznym uruchamianiem stworzonego projektu PLC:\\[1mm]
Po utworzeniu oprogramowania sterownika~PLC~oraz po odpowiednim skonfigurowaniu go w oprogramowaniu TwinCAT System Manager tj. linkowaniu zmiennych programu do~odpowiednich fizycznych wejść oraz wyjść modelu oraz aktywowaniu tak przygotowanej konfiguracji, które z~kolei wymusza zresetowanie systemu nie~następuje uruchomienie projektu sterownika PLC. W początkowej fazie autor przełączał się na oprogramowanie TwinCAT PLC Control gdzie logował się do~sterownika i~ręcznie uruchamiał stworzony przez siebie kod. Niestety to~rozwiązanie na~dłuższą metę okazało~się niewystarczające i~czasochłonne. Okazało się, że~w~sterownikach firmy Beckhoff trzeba w~specjalny sposób przygotować oprogramowanie, które ma~być uruchamiane w~sposób automatyczny, tzn. trzeba utworzyć projekt, który jest bootowalny. Początkowo takie podejście wydało się autorowi bardzo dziwne, ale po~dłuższym zastanowieniu oraz kilku rozmowach z~bardziej doświadczonymi w~branży osobami okazało się, że~ma~ono swoje plusy. Przykładowo w~przypadku tworzenia oprogramowania w~fazie rozwojowej reset urządzenia pozwala przerwać całkowicie wykonywanie oprogramowania zawierającego błędy mające destrukcyjny wpływ na~model lub w~praktyce na~obiekt przemysłowy. Po~zastosowaniu nowej metody uruchamianie~i testowanie tworzonego oprogramowania stało~się zdecydowanie prostsze.

\item Problem ze zbyt wysokim napięciem napędu serwomechanizmów:\\[1mm]

\item Problem z~wykryciem jednego z~silników:\\[1mm]
aa

\item Utrata komunikacji z~jednym ze sterowników:\\[1mm]
Przy pewnej modyfikacji konfiguracji stanowiska typu~CX związanej z~próbą uruchomienia silników została utracona komunikacja z urządzeniem. Precyzując, przestało ono odpowiadać na~poprzednim adresie. Sytuacja była o~tyle dziwna, że system na urządzeniu działał całkowicie normalnie. Po~podpięciu zewnętrznego monitora okazało się, że w systemie operacyjnym karty sieciowe są skonfigurowane prawidłowo oraz udaje się nawiązać połączenie i~uzyskać żądany adres (konfiguracja adresów jest statyczna). Próby wykorzystania programu ping nie~dały początkowo żadnego efektu, ponieważ urządzenie nie odpowiadało. W licznych próbach i~pomysłach udało się ustalić, że urządzenie podczas uruchamiania (dokładnie podczas uruchamiania systemu operacyjnego) odpowiada na kilka zapytań (od 4 do 6 w kilku próbach), podobnie w momencie zamykania systemu. To~odkrycie zasugerowała autorowi, że~coś blokuje, przekonfigurowywuje lub wywłaszcza urządzenia. Pierwszy został przeanalizowany autostart systemu Windows lecz okazał się on pusty. Pojawił się pomysł przywrócenia urządzenia do ustawień fabrycznych lecz nie~udało się odnaleźć takiej możliwości. Problem udało się rozwiązać uruchamiając w~systemie operacyjnym standardowy menadżer plików (w~tym przypadku explorer), odnajdując stworzone pliki konfiguracji TwinCAT i~usuwając~je. ($\backslash Hard Disk\backslash TwinCAT\backslash Boot\backslash$)

\end{itemize}
\indent
\indent Wszystkie problemy zostały rozwiązane i~w~ostatecznej wersji oprogramowania nie wpływają one w~negatywny sposób na~pracę modelu.
 %
\section{Wnioski}
Protokół EtherCAT jest rozwiązaniem zdecydowanie bardzo nowoczesnym, zaawansowanym oraz pomysłowym. Pozwala bardzo dobrze wykorzystać wzrastające moce obliczeniowe sterowników~PLC, przy zachowaniu wszelkich rygorów czasowych.
Zdecydowanie ogromny wpływ na~tak szybki rozwój ma~fakt, że~technologia jest bardzo otwarta i~postawiona na~współpracę wszystkich zainteresowanych rozwojem stron. Olbrzymim plusem jest fakt wykorzystania standardowej struktury Ethernetu co upraszcza proces integracji w obrębie jednego systemu obu standardów.

Temat zdecydowanie nadaje~się do~pogłębiania i~dalszych badań. Autor dopuszcza taką możliwość w~ramach prac badawczych w~toku swoich studiów doktoranckich. Zarówno same sterowniki firmy Beckhoff oparte o koncepcję ,,soft PLC'' jak i sam badany protokół EtherCAT są na~tyle rozbudowane i rozwojowe, że zawsze znajdzie się jakiś aspekt do~przeanalizowania od strony teoretycznej i~eksperymentalnej. W momencie powstawania niniejszej pracy dwa bardzo interesujące kwestie wydają się autorowi ciekawą postawą do przeprowadzenia badań.

Po pierwsze przetestowanie elementów sieci EtherCAT pochodzących od innych producentów. Różne koncepcje ASIC itd.

IMPLEMENTACJA U INNYCH PRODUCENTÓW PORÓWNANIE - ASIC ARM< SINTARA ITD.


Drugim ciekawym rozwiązaniem i pomysłem na~które autor natrafił w~sieci w~czasie analizy tematu, rozwiązywania problemów oraz pisania niniejszej pracy jest EtherLab. Jest to technologia łącząca sprzęt i oprogramowanie w celach testowych oraz do~sterowania procesów przemysłowych. Jest~to niejako technika zbudowana z~dobrze znanych i~niezawodnych elementów.
EtherLab pracuje jako działający w czasie rzeczywistym moduł jądra otwartego systemu Linux, który komunikuje się z urządzeniami peryferyjnymi poprzez protokół EtherCAT. Rozwiązanie jest całkowicie darmowe i otwarte co na~pewno jest jego olbrzymią zaletą. Można zdecydować się na pobranie sobie wszystkich komponentów i~ich samodzielne uruchomienie lub zakup gotowego preinstalowanego zestawu startowego bezpośrednio od twórców. 
Oprogramowanie całego zestawu może zostać wygenerowane przy użyciu Simulinka/RTW lub napisane ręcznie w C. Następnie tak przygotowane jest uruchamiane w środowisku kontrolującym proces (jądro Linuksa oraz moduł czasu rzeczywistego) komunikującym się z ,,obiektem przemysłowym'' poprzez EtherCAT. Dodatkowo można rozszerzyć możliwości całego zestawu poprzez Ethernet TCP/IP dołączając interfejs użytkownika (ang. Frontend) w wersji dla Linuksa lub Windowsa albo jeden z innych dodatkowych serwisów. Przykładowe serwisy to:
\begin{itemize}
\item Raportowanie poprzez SMS,
\item Zdalne usługi: Internet, ISDN, DSL,
\item Usługi sieciowe: Web, DHCP, Drukowanie,
\item Logowanie danych (ang. data logging).
\end{itemize}
Jeżeli autor będzie miał taką możliwość to~na~pewno chętnie przyjrzy się tej koncepcji ze~względu na swoją sympatię do~systemu Linux oraz wszystkich rozwiązań go wykorzystujących. Ciekawe wydają~się badania wydajności takiego rozwiązania oraz porównanie ich z drogimi rozwiązaniami komercyjnymi.
 %
\section{Bibliografia}
Literatura, która została wykorzystana przez autora w czasie powstawania projektu, którą opisuje niniejsza dokumentacja.

\begin{thebibliography}{99}
%\begin{enumerate}
%\item 
\bibitem{plc1} 
Jerzy Kasprzyk: 
\emph{"Programowanie sterowników przemysłowych"},
Wydawnictwa Naukowo-Techniczne WNT, 
Warszawa, 
2007      

\bibitem{plc2} 
\emph{"Programowalne sterowniki PLC w systemach sterowania przemysłowego"}, 
Politechnika Radomska, 
Radom,
2001

\bibitem{plc4} 
Andrzej Maczyński:
\emph{„Sterowniki programowalne PLC. Budowa systemu i podstawy programowania”},
Astor, 
Kraków,
2001 

\bibitem{plc5} 
Zbigniew Seta: 
\emph{„Wprowadzenie do zagadnień sterowania. Wykorzystanie programowalnych sterowników logicznych PLC.”},
MIKOM Wydawnictwo, 
Warszawa,
2002 

\bibitem{plc6} 
Janusz Kwaśniewski: 
\emph{„Programowalne sterowniki przemysłowe w systemach sterowania”}, 
Wyd. AGH, 
Kraków,
1999

\bibitem{step2} 
Dokumentacja producenta: 
\emph{„SIMATIC Working with STEP 7 - Getting Started Edition”}, 
marzec 2006.

\bibitem{kurs1} 
Materiały szkoleniowe:
„SIMATIC S7 - Kurs podstawowy”

\end{thebibliography}

 %

\section{Spis rysunków, tablic i kodów źródłowych}
\subsection{Spis rysunków}
%\listoffigures
\renewcommand*\numberline[1]{Rysunek\,#1:\indent}
\listoffigures
\subsection{Spis tablic}
\renewcommand*\numberline[1]{Tablica\,#1:\indent}
\listoftables
\subsection{Spis kodów źródłowych}
\renewcommand*\numberline[1]{Kod źródłowy\,#1:\indent}
\lstlistoflistings

\section{Załączniki}
\begin{itemize}
\item Oświadczenie o autorstwie,
\item Płyta CD, na której znajdują się:
\begin{itemize}
\item Kod oprogramowania wewnętrznego TwinCAT PLC Control,
\item Pliki projektu TwinCAT System Manager,
\item Kod wizualizacji oraz pliki projektu !?,
\item Plik wykonywalny wizualizacji !?
\item \LaTeX-owe pliki źródłowe pracy magisterskiej.
\end{itemize}
\end{itemize}
 %

\end{document}
