\section{Wnioski}
Tworzenie rozbudowanego oprogramowania do obsługi robota 3D pracującego w~magazynie wysokiego składowania wymaga rozwiązania wielu problemów programistycznych. Autor w pierwszej kolejności zapoznał się z~kursami programowania w~środowisku Step~7 oraz WinCC flexible \cite{kurs1,kurs2,kurs3} dostępnymi w~sieci.

W~procesie tworzenia oprogramowania autor zapoznał się z~wieloma zagadnieniami typowymi dla sterowników firmy Beckhoff. Przykładowo, 

Wykonanie pojedynczego cyklu sterownika stworzonego przez autora oprogramowania w czasie testowania wynosiło od 1 do 3~ms. Czas ten jest zadowalający biorąc pod uwagę, że sterownik bez funkcji użytkownika oraz z~pustym blokiem OB1 ma czas wykonania cyklu równy 1~ms oraz to, że oprogramowanie przynajmniej zdaniem autora jest rozbudowane.

Wizualizacja stanu robota stanowi bardzo atrakcyjną formę korzystania z~urządzenia, która jest jednocześnie wyjątkowo przystępna dla osób nie znających zagadnień związanych z programowaniem sterowników przemysłowych. Wizualizacja będzie pełniła ważną rolę w~prowadzeniu zajęć laboratoryjnych związanych ze~sterownikami firmy Beckhoff