\section{Wnioski}
Protokół EtherCAT jest rozwiązaniem zdecydowanie bardzo nowoczesnym, zaawansowanym oraz pomysłowym. Pozwala bardzo dobrze wykorzystać wzrastające moce obliczeniowe sterowników~PLC, przy zachowaniu wszelkich rygorów czasowych.
Zdecydowanie ogromny wpływ na~tak szybki rozwój ma~fakt, że~technologia jest bardzo otwarta i~postawiona na~współpracę wszystkich zainteresowanych rozwojem stron. Olbrzymim plusem jest fakt wykorzystania standardowej struktury Ethernetu co upraszcza proces integracji w obrębie jednego systemu obu standardów.

Temat zdecydowanie nadaje~się do~pogłębiania i~dalszych badań. Autor dopuszcza taką możliwość w~ramach prac badawczych w~toku swoich studiów doktoranckich. Zarówno same sterowniki firmy Beckhoff oparte o koncepcję ,,soft PLC'' jak i sam badany protokół EtherCAT są na~tyle rozbudowane i rozwojowe, że zawsze znajdzie się jakiś aspekt do~przeanalizowania od strony teoretycznej i~eksperymentalnej. W momencie powstawania niniejszej pracy dwa bardzo interesujące kwestie wydają się autorowi ciekawą postawą do przeprowadzenia badań.

Po pierwsze przetestowanie elementów sieci EtherCAT pochodzących od innych producentów. Różne koncepcje ASIC itd.

IMPLEMENTACJA U INNYCH PRODUCENTÓW PORÓWNANIE - ASIC ARM< SINTARA ITD.


Drugim ciekawym rozwiązaniem i pomysłem na~które autor natrafił w~sieci w~czasie analizy tematu, rozwiązywania problemów oraz pisania niniejszej pracy jest EtherLab. Jest to technologia łącząca sprzęt i oprogramowanie w celach testowych oraz do~sterowania procesów przemysłowych. Jest~to niejako technika zbudowana z~dobrze znanych i~niezawodnych elementów.
EtherLab pracuje jako działający w czasie rzeczywistym moduł jądra otwartego systemu Linux, który komunikuje się z urządzeniami peryferyjnymi poprzez protokół EtherCAT. Rozwiązanie jest całkowicie darmowe i otwarte co na~pewno jest jego olbrzymią zaletą. Można zdecydować się na pobranie sobie wszystkich komponentów i~ich samodzielne uruchomienie lub zakup gotowego preinstalowanego zestawu startowego bezpośrednio od twórców. 
Oprogramowanie całego zestawu może zostać wygenerowane przy użyciu Simulinka/RTW lub napisane ręcznie w C. Następnie tak przygotowane jest uruchamiane w środowisku kontrolującym proces (jądro Linuksa oraz moduł czasu rzeczywistego) komunikującym się z ,,obiektem przemysłowym'' poprzez EtherCAT. Dodatkowo można rozszerzyć możliwości całego zestawu poprzez Ethernet TCP/IP dołączając interfejs użytkownika (ang. Frontend) w wersji dla Linuksa lub Windowsa albo jeden z innych dodatkowych serwisów. Przykładowe serwisy to:
\begin{itemize}
\item Raportowanie poprzez SMS,
\item Zdalne usługi: Internet, ISDN, DSL,
\item Usługi sieciowe: Web, DHCP, Drukowanie,
\item Logowanie danych (ang. data logging).
\end{itemize}
Jeżeli autor będzie miał taką możliwość to~na~pewno chętnie przyjrzy się tej koncepcji ze~względu na swoją sympatię do~systemu Linux oraz wszystkich rozwiązań go wykorzystujących. Ciekawe wydają~się badania wydajności takiego rozwiązania oraz porównanie ich z drogimi rozwiązaniami komercyjnymi.
