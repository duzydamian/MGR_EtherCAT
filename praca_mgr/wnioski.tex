\section{Wnioski}
Protokół EtherCAT jest rozwiązaniem zdecydowanie bardzo nowoczesnym, zaawansowanym oraz pomysłowym. Pozwala bardzo dobrze wykorzystać wzrastające moce obliczeniowe sterowników~PLC, przy zachowaniu wszelkich rygorów czasowych.
Zdecydowanie ogromny wpływ na~tak szybki rozwój ma~fakt, że~technologia jest bardzo otwarta i~postawiona na~współpracę wszystkich zainteresowanych rozwojem stron. Olbrzymim plusem jest fakt wykorzystania standardowej struktury Ethernetu co upraszcza proces integracji w obrębie jednego systemu obu standardów.

Temat zdecydowanie nadaje~się do~pogłębiania i~dalszych badań. Autor dopuszcza taką możliwość w~ramach prac badawczych w~toku swoich studiów doktoranckich. Zarówno same sterowniki firmy Beckhoff oparte o koncepcję ,,soft PLC'' jak i sam badany protokół EtherCAT są na~tyle rozbudowane i rozwojowe, że zawsze znajdzie się jakiś aspekt do~przeanalizowania od strony teoretycznej i~eksperymentalnej. W momencie powstawania niniejszej pracy trzy bardzo interesujące kwestie wydają się autorowi ciekawą postawą do przeprowadzenia badań.

Po pierwsze przetestowanie elementów sieci EtherCAT pochodzących od innych producentów. 
Jak już zostało to opisane w rozdziale zawierającym analizę tematu węzły EtherCAT mogą być realizowane na wiele sposobów zależnie od wymagań funkcjonalnych. Autor bardzo chętnie podjąłby badania mające na celu zaprojektowanie własnego układu cyfrowych wejść/wyjść i porównanie takiego rozwiązania z dostępnymi na rynku, aby wyznaczyć stosunek jakości do kosztów. W dalszym etapie zdecydowanie warto~by przyjrzeć się bardziej elastycznym i rozbudowanym rozwiązaniom pozwalającym na realizację bardziej zaawansowanych węzłów.
Interesująca wydaje się możliwość konstruowania zupełnie od podstaw węzłów o dowolnej funkcjonalności. Dzięki temu można by spróbować zbudować węzeł eksperymentalny mający za zadanie tylko monitorowanie przepływających przez niego ramek i~ich ewentualne gromadzenie. Być może taka analiza krążących w sieci ramek pozwoliłaby wykryć jakieś interesujące właściwości lub zachowania charakterystyczne dla omawianego protokołu.

Zdaniem autora najbardziej interesujące pod względem badawczym są właśnie jednostki wyposażone w~procesor dedykowany do~realizacji bardziej złożonych zadań niezależnie od działania protokołu. Ciekawym rozwiązaniem jest oferowany przez firmę Texas Instruments mikrokontroler z rdzeniem ARM Cortex-A8 z rodziny Sitara AM335x. Być może rozwiązania firm wchodzących w~skład ETG okażą się lepsze lub gorsze pod względem szybkości w porównaniu do pierwszych twórców protokołu.

Drugim ciekawym rozwiązaniem i pomysłem na~które autor natrafił w~sieci w~czasie analizy tematu, rozwiązywania problemów oraz pisania niniejszej pracy jest EtherLab. Jest to technologia łącząca sprzęt i oprogramowanie w celach testowych oraz do~sterowania procesów przemysłowych. Jest~to niejako technika zbudowana z~dobrze znanych i~niezawodnych elementów.
EtherLab pracuje jako działający w czasie rzeczywistym moduł jądra otwartego systemu Linux, który komunikuje się z urządzeniami peryferyjnymi poprzez protokół EtherCAT. Rozwiązanie jest całkowicie darmowe i otwarte co na~pewno jest jego olbrzymią zaletą. Można zdecydować się na pobranie sobie wszystkich komponentów i~ich samodzielne uruchomienie lub zakup gotowego preinstalowanego zestawu startowego bezpośrednio od twórców. 
Oprogramowanie całego zestawu może zostać wygenerowane przy użyciu Simulinka/RTW lub napisane ręcznie w C. Następnie tak przygotowane jest uruchamiane w środowisku kontrolującym proces (jądro Linuksa oraz moduł czasu rzeczywistego) komunikującym się z ,,obiektem przemysłowym'' poprzez EtherCAT. Dodatkowo można rozszerzyć możliwości całego zestawu poprzez Ethernet TCP/IP dołączając interfejs użytkownika (ang. Frontend) w wersji dla Linuksa lub Windowsa albo jeden z innych dodatkowych serwisów. Przykładowe serwisy to:
\begin{itemize}
\item Raportowanie poprzez SMS,
\item Zdalne usługi: Internet, ISDN, DSL,
\item Usługi sieciowe: Web, DHCP, Drukowanie,
\item Logowanie danych (ang. data logging).
\end{itemize}
Jeżeli autor będzie miał taką możliwość to~na~pewno chętnie przyjrzy się tej koncepcji ze~względu na swoją sympatię do~systemu Linux oraz wszystkich rozwiązań go wykorzystujących. Ciekawe wydają~się badania wydajności takiego rozwiązania oraz porównanie ich z drogimi rozwiązaniami komercyjnymi.

Trzecim w~kolejności pomysłem, który powstał na~etapie realizacji pracy dyplomowej oraz zainspirowany po~części realizacją przez kolegów z~roku projektu z przedmiotu Projektowanie Przemysłowych Systemów Komputerowych mającego na~celu umożliwienie podłączenia do~sterownika Beckhoff bazy danych. Studenci zaproponowali i~zrealizowali rozwiązanie bazujące na RS-232. Zdaniem autora protokół EtherCAT nadawałby się do~tego celu bardzo dobrze głównie ze~względu na prędkość działania, ale również ze względu na możliwość bezpośredniej modyfikacji danych zapisywanych do~bazy przez dowolny węzeł sieci. Pomysł nie~jest w swoich podstawach badawczy, a~bardziej praktyczny. Autor dostrzega jednak możliwość przygotowania stanowiska eksperymentalnego na~którym można~by przeprowadzić badania wydajności takiego rozwiązania.
