\section{Wnioski}
Protokół EtherCAT jest rozwiązaniem zdecydowanie bardzo nowoczesnym, zaawansowanym oraz pomysłowym. 
Jest on stosunkowo bardzo młody w~porównaniu z~innymi znanymi autorowi protokołami komunikacyjnymi stosowanymi powszechnie w~przemyślę takimi jak:
\begin{itemize}
\item Modbus -- wprowadzony na rynek przez firmę Modicon w 1979 roku,
\item Profibus (\textbf{Pro}cess \textbf{Fi}eld \textbf{Bus}) -- wprowadzony przez Niemiecki departament edukacji i badań (niem. Bundesministerium für Bildung und Forschung, w~skrócie BMBF) w roku 1989,
\item CAN (Controller Area Network) -- wprowadzony przez Robert Bosch GmbH w roku 1989
\item EGD (Ethernet Global Data ) -- wprowadzony przez firmy GE~Fanuc Automation oraz GE~Drive Systems w~roku 1998
\end{itemize}

Fakt, że protokół jest tak młody wpływa bardzo pozytywnie na~liczbę materiałów dostępnych w~internecie. Niestety jest też drobna wada, a~mianowicie informacje są często nie~kompletne i~trzeba doszukiwać się pożądanych szczegółów w~różnych miejscach. Na szczęście po~zapoznaniu się z~kilkoma stronami oraz artykułami można uzyskać zupełnie przyzwoity zakres wiedzy o~tym protokole.

Pozwala bardzo dobrze wykorzystać wzrastające moce obliczeniowe sterowników~PLC, przy zachowaniu wszelkich rygorów czasowych. Zdecydowanie ogromny wpływ na~tak szybki rozwój ma~fakt, że~technologia jest bardzo otwarta i~postawiona na~współpracę wszystkich zainteresowanych rozwojem stron. Olbrzymim plusem jest fakt wykorzystania standardowej struktury Ethernetu co upraszcza proces integracji w~obrębie jednego systemu obu standardów.

Badany protokół jest bardzo rozbudowany i~skomplikowany. Jego zrozumienie wymaga czasu oraz zapoznania się z~dużą ilością stosownych dokumentacji. Mnogość możliwości i~konfiguracji będąca niewątpliwą zaletą tego standardu może na~początku przerażać, ale niestety takie są koszty tak rozbudowanej i~zaawansowanej funkcjonalności. Autor sam był początkowo zaskoczony tak wysokim stopniem zaawansowania protokołu przemysłowego, w~porównaniu przykładowo do~poznanego dobrze wcześniej protokołu Modbus.

Prawdopodobnie po zakończeniu projektu posłuży on jako podstawa do stworzenia i~przeprowadzenia ćwiczeń laboratoryjnych prowadzonych w~ramach działalności dydaktycznej prowadzonej przez pracowników Zespołu Przemysłowych Zastosowań Informatyki. Można z~wykorzystaniem dołączonych konfiguracji, projektów oraz dokumentacji przygotować proste laboratorium z~podstaw programowania sterowników firmy Beckhoff, bardziej zaawansowane programowanie ze~sterowaniem silnikami lub najbardziej zaawansowane laboratorium związane z działaniem protokołu EtherCAT.

\subsection{Perspektywy dalszych badań}
Temat zdecydowanie nadaje~się do~pogłębiania i~prowadzenia dalszych badań. Autor dopuszcza taką możliwość w~ramach prac badawczych w~toku swoich studiów doktoranckich. Zarówno same sterowniki firmy Beckhoff oparte o koncepcję ,,soft PLC'' jak i sam badany protokół EtherCAT są na~tyle rozbudowane i rozwojowe, że zawsze znajdzie się jakiś aspekt do~przeanalizowania od strony teoretycznej i~eksperymentalnej. W momencie powstawania niniejszej pracy trzy bardzo interesujące kwestie wydają się autorowi ciekawą postawą do przeprowadzenia badań.

Po pierwsze przetestowanie elementów sieci EtherCAT pochodzących od innych producentów. 
Jak już zostało to opisane w rozdziale zawierającym analizę tematu węzły EtherCAT mogą być realizowane na wiele sposobów zależnie od wymagań funkcjonalnych. Autor bardzo chętnie podjąłby badania mające na celu zaprojektowanie własnego układu cyfrowych wejść/wyjść i porównanie takiego rozwiązania z dostępnymi na rynku, aby wyznaczyć stosunek jakości do kosztów. W dalszym etapie zdecydowanie warto~by przyjrzeć się bardziej elastycznym i rozbudowanym rozwiązaniom pozwalającym na realizację bardziej zaawansowanych węzłów.
Interesująca wydaje się możliwość konstruowania zupełnie od podstaw węzłów o dowolnej funkcjonalności. Dzięki temu można by spróbować zbudować węzeł eksperymentalny mający za zadanie tylko monitorowanie przepływających przez niego ramek i~ich ewentualne gromadzenie. Być może taka analiza krążących w sieci ramek pozwoliłaby wykryć jakieś interesujące właściwości lub zachowania charakterystyczne dla omawianego protokołu.

Zdaniem autora najbardziej interesujące pod względem badawczym są właśnie jednostki wyposażone w~procesor dedykowany do~realizacji bardziej złożonych zadań niezależnie od działania protokołu. Ciekawym rozwiązaniem jest oferowany przez firmę Texas Instruments mikrokontroler z rdzeniem ARM Cortex-A8 z rodziny Sitara AM335x. Być może rozwiązania firm wchodzących w~skład ETG okażą się lepsze lub gorsze pod względem szybkości w porównaniu do pierwszych twórców protokołu.

Drugim ciekawym rozwiązaniem i pomysłem na~które autor natrafił w~sieci w~czasie analizy tematu, rozwiązywania problemów oraz pisania niniejszej pracy jest EtherLab. Jest to technologia łącząca sprzęt i oprogramowanie w celach testowych oraz do~sterowania procesów przemysłowych. Jest~to niejako technika zbudowana z~dobrze znanych i~niezawodnych elementów.
EtherLab pracuje jako działający w czasie rzeczywistym moduł jądra otwartego systemu Linux, który komunikuje się z urządzeniami peryferyjnymi poprzez protokół EtherCAT. Rozwiązanie jest całkowicie darmowe i otwarte co na~pewno jest jego olbrzymią zaletą. Można zdecydować się na pobranie sobie wszystkich komponentów i~ich samodzielne uruchomienie lub zakup gotowego preinstalowanego zestawu startowego bezpośrednio od twórców. 
Oprogramowanie całego zestawu może zostać wygenerowane przy użyciu Simulinka/RTW lub napisane ręcznie w C. Następnie tak przygotowane jest uruchamiane w środowisku kontrolującym proces (jądro Linuksa oraz moduł czasu rzeczywistego) komunikującym się z ,,obiektem przemysłowym'' poprzez EtherCAT. Dodatkowo można rozszerzyć możliwości całego zestawu poprzez Ethernet TCP/IP dołączając interfejs użytkownika (ang. \textit{Frontend}) w wersji dla Linuksa lub Windowsa albo jeden z innych dodatkowych serwisów. Przykładowe serwisy to:
\begin{itemize}
\item Raportowanie poprzez SMS,
\item Zdalne usługi: Internet, ISDN, DSL,
\item Usługi sieciowe: Web, DHCP, Drukowanie,
\item Logowanie danych (ang. \textit{data logging}).
\end{itemize}
Jeżeli autor będzie miał taką możliwość to~na~pewno chętnie przyjrzy się tej koncepcji ze~względu na swoją sympatię do~systemu Linux oraz wszystkich rozwiązań go wykorzystujących. Ciekawe wydają~się badania wydajności takiego rozwiązania oraz porównanie ich z drogimi rozwiązaniami komercyjnymi.

Trzecim w~kolejności pomysłem, który powstał na~etapie realizacji pracy dyplomowej oraz zainspirowany po~części realizacją przez kolegów z~roku projektu z przedmiotu Projektowanie Przemysłowych Systemów Komputerowych mającego na~celu umożliwienie podłączenia do~sterownika Beckhoff bazy danych. Studenci zaproponowali i~zrealizowali rozwiązanie bazujące na RS-232. Zdaniem autora protokół EtherCAT nadawałby się do~tego celu bardzo dobrze głównie ze~względu na prędkość działania, ale również ze względu na możliwość bezpośredniej modyfikacji danych zapisywanych do~bazy przez dowolny węzeł sieci. Pomysł nie~jest w swoich podstawach badawczy, a~bardziej praktyczny. Autor dostrzega jednak możliwość przygotowania stanowiska eksperymentalnego na~którym można~by przeprowadzić badania wydajności takiego rozwiązania.

Kolejnym pomysłem jest przebadania protokołu z wykorzystaniem stworzonego na naszym wydziale i~opisanego w~literaturze analizatora sieci czasu rzeczywistego opartych na Ethernecie \cite{projekt_FPGA}. Można~by to~urządzenie zastosować do zewnętrznej analizy protokołu EtherCAT,a nie wewnątrz działającego systemu jak to miało miejsce w~przypadku badań opisanych niniejszym dokumentem. Możliwe byłoby porównanie opóźnień wprowadzanych przez węzły typu EtherCAT z~tymi wprowadzonymi przez omawiany analizator.

Następnym pomysłem do ewentualnego przebadania w przyszłości znalezionym w sieci jest możliwość uruchomienia kontrolera urządzenia podrzędnego EtherCAT na układzie FPGA firmy Xilinx lub Altera \cite{FPGA_Xilinx, FPGA_Altera}. Rozwiązanie tego typu można swobodnie przetestować i~zaimplementować dzięki dostarczanym przez firmę Beckhoff zestawom ewaluacyjnym wyposażonym w~wymienione układy. Przykładowe dwa zestawy to EL9830 wyposażony w układ Altery (EP3C25) oraz EL9840 z~Xilinxem (XC3S1200).


Teoretycznie dzięki wykorzystaniu w protokole EtherCAT standardu Ethernet możliwe jest sterowanie urządzeniami z~poziomu zwykłego komputera klasy PC z~wykorzystaniem modelu TCP/IP. Ciekawym wydaje się na ile skomplikowane okaże się zbudowanie ramki do celów typowo testowych. Autor ma~świadomość, że rozwiązanie takie nie~nadaje się raczej do profesjonalnego stosowania w~przemyśle, ale wydaje się być interesującą alternatywą do~testowania działania urządzeń wykonawczych bez konieczności zaprzęgania sterownika PLC. Dodatkowym interesującym aspektem wykorzystania takiego rozwiązania wydaje się lepsze poznanie i~zrozumienie zasad działania protokołu. Autor w~czasie realizacji jednego z~projektów semestralnych prowadzonego w~ramach przedmiotu Projektowanie Przemysłowych Systemów Komputerowych i~swojej działalności w Kole Naukowym Przemysłowych Zastosowań Informatyki „Industrum” implementował samodzielnie oprogramowanie do~komunikacji w~sieci ELAN. Działanie to~pozwoliło zdecydowanie lepiej zrozumieć zasady działania tego dość prostego, ale rozbudowanego interfejsu.

Tak jak już zostało napisane w~rozdziale opisującym badania do zwiększenia precyzji badań mówiących o~czasie ponownego podłączenia odłączonego i~podłączonego węzła sieci przydałoby się zaprojektować i wykonać urządzenie pozwalające w~precyzyjny sposób wpływać na ciągłość linii transmisyjnej. Urządzenie takie można~by oprzeć o~układ mikroprocesorowy do kontroli oraz tranzystory do przerywania ciągłości magistrali. Pozwoliło~by to~w~bardzo precyzyjny sposób określić czas ponownego podłączenia z~pominięciem błędu pomiarowego wprowadzanego przez człowieka. Takie urządzenie pozwoliłoby również przebadać inne możliwe scenariusze. Przykładowo można~by spróbować wygenerować przerwę trwającą równowartość czasu potrzebnego na transmisję pojedynczego znaku (ewentualnie kilku znaków). Pozwoliłoby to zaobserwować zachowanie protokołu w momencie wystąpienia błędów transmisji. Ewentualnie wywołując odpowiednio dużo błędów można sprawdzić stabilność pracy protokołu w~sytuacji problemów transmisyjnych.
Dzięki stworzeniu omówionego urządzenia dałoby się również przebadać inne protokoły komunikacyjne.
