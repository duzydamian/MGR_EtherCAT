\section{Wstęp}
%\subsection{Geneza}
Tematem projektu, którego dotyczy ta praca jest: „Projekt i~realizacja stanowiska laboratoryjnego do badania zależności czasowych w~sieci EtherCAT". Zagadnienia związane z~tworzeniem oprogramowania dla sterowników przemysłowych są dla autora niezwykle interesujące, a zrealizowany projekt miał na~celu dalsze pogłębienie jego wiedzy z tego zakresu. Wyboru tego konkretnego tematu autor dokonał, ponieważ protokół EtherCAT jest jeszcze nowością i według wielu źródeł stanowi przyszłość branży informatyki przemysłowej \cite{art1, art2}, a~praca nad tym tematem wydaje się być pomocna i~wartościowa w przyszłej pracy zawodowej lub na~ewentualnym dalszym etapie kształcenia.

\subsection{Stanowisko laboratoryjne}
Na potrzeby realizacji projektu wykorzystano dwa różne istniejące stanowiska laboratoryjne, które składały się~z~elementów opisanych w Tabeli~\ref{stanowiska}.
\begin{table}[!htb]
\begin{center}
\begin{tabular}{| p{0.51\textwidth} || p{0.51\textwidth} |}\hline
Stanowisko typu CP (Rysunek~\ref{stanowisko:cp}) & Stanowisko typu CX (Rysunek~\ref{stanowisko:cx})  \\\hline
\begin{enumerate}
\item 2~silniki AM3021-0C00-0000,
\item Wyspa EK1100 z~zestawem modułów~IO:
\begin{itemize}
\item Terminal sieci EtherCAT EK1100,
\item 2-kanałowy moduł wyjść analogowych EL4132,
\item 4-kanałowy moduł wejść cyfrowych EL1004,
\item 2 4-kanałowe moduły wyjść cyfrowych EL2004,
\item 2-kanałowy moduł wejść analogowych EL3102,
\end{itemize}
\item Napęd serwomechnizmów AX5203 (2~osiowy napęd),
\item Komputer przemysłowy C6925,
\item Zasilacz.
\end{enumerate}
&
\begin{enumerate}
\item 2~silniki AM3021-0C00-0000,
\item Zestaw modułów IO:
\begin{itemize}
\item 2-kanałowy moduł wyjść analogowych EL4132,
\item 4-kanałowy moduł wejść cyfrowych EL1004,
\item 2 4-kanałowe moduły wyjść cyfrowych EL2004,
\item 2-kanałowy moduł wejść analogowych EL3102,
\end{itemize}
\item Terminal sieci EtherCAT EK1100,
\item Napęd serwomechnizmów AX5203 (2~osiowy napęd),
\item Modułowy komputer przemysłowy CX1020:
\begin{itemize}
\item Interfejs USB/DVI CX1020-N010 ,
\item Ethernet CX1020-N000,
\item CPU CX1020-0113,
\item Zasilacz CPU i magistrali I/O CX1100-0004,
\end{itemize}
\item Zasilacz.
\end{enumerate}
\\\hline                                            
\end{tabular}
\end{center}
\vspace*{-6mm}
  \caption{Dostępne stanowiska laboratoryjne}
	\label{stanowiska}
\end{table}

\begin{figure}[htbp]
 \centering
        \tikzstyle{background grid}=[draw, black!50,step=.25cm]
	\begin{tikzpicture}[node distance=1cm, auto, show background grid]
	\tikzset{
    	mynode/.style={rectangle,rounded corners,draw=black, top color=white, bottom color=yellow!50,very thick, inner sep=1em, minimum size=3em, 		text centered, text width=3cm},
	    myarrow/.style={->, >=latex', shorten >=1pt, thick},
	    myline/.style={-, =latex', shorten >=1pt, rounded corners, ultra thick},
	    mylabel/.style={text width=7em, text centered} 
	} 
	\node[mynode] (plc) {Komputer \\ przemysłowy C6925};  
	\node [left=of plc] (laptop) {\includegraphics[width=3cm]{images/laptop}};
	\node[mynode, right=of plc] (zasilacz) {Zasilacz};
	\node[below=3cm of plc] (dummy) {}; 
	\node[mynode, left=of dummy] (io) {Wyspa EK1100 z~zestawem modułów IO};  
	\node[mynode, right=of dummy] (naped) {Napęd \\ serwomechnizmów AX5203};
	\node[below=3cm of naped] (dummy2) {}; 
	\node[mynode, left=of dummy2, text width=4cm](silnik1){Silnik \\ AM3021-0C00-0000};
	\node[mynode, right=of dummy2, text width=4cm](silnik2){Silnik \\ AM3021-0C00-0000};
	
	\draw[myline,blue] (laptop.east) -- ++(-1, 0) -- (plc.west);
	
	\draw[myline,black] (zasilacz.west) -- (plc.east);
	\draw[myline,black] (zasilacz.south) -- (io.north);
	\draw[myline,black] (zasilacz.south) -- (naped.north);	
	
	\draw[myline,yellow] (plc.south) -- (io.north);	
	\draw[myline,yellow] (io.east) -- (naped.west);	

	\draw[myline, green, bend right=10] (naped.south) to (silnik1.north);
	\draw[myline, orange, bend left=10] (naped.south) to (silnik1.north);	
	\draw[myline, green, bend right=10] (naped.south) to (silnik2.north);
	\draw[myline, orange, bend left=10] (naped.south) to (silnik2.north);	
	%\draw[<->, >=latex', shorten >=2pt, shorten <=2pt, bend right=45, thick, dashed] 
    %(io.south) to node[auto, swap] {Competition}(naped.south); 
    
    \draw [yellow, line width=6] (6,-2) -- (6.5,-2); \node at (7.5,-2) {EtherCAT};
    \draw [blue, line width=6] (6,-2.5) -- (6.5,-2.5); \node at (7.5,-2.5) {Ethernet};
    \draw [black, line width=6] (6,-3) -- (6.5,-3); \node at (7.5,-3) {Zasilanie};
    \draw [green, line width=6] (6,-3.5) -- (6.25,-3.5); \draw [orange, line width=6] (6.25,-3.5) -- (6.5,-3.5); 
    \node [text width=2cm] at (7.75,-3.75) {Sterowanie silnikiem};    
\end{tikzpicture} 
\caption{Schemat stanowiska typu CP}
\label{stanowisko:cp}
\end{figure} %
\begin{figure}[htbp]
 \centering
        \tikzstyle{background grid}=[draw, black!50,step=.5cm]
	\begin{tikzpicture}[node distance=1cm, auto]%, show background grid]
	\tikzset{
    	mynode/.style={rectangle,rounded corners,draw=black, top color=white, bottom color=yellow!50,very thick, inner sep=1em, minimum size=3em, 		text centered, text width=3cm},
    	mynodemini/.style={rectangle,rounded corners,draw=black, top color=white, bottom color=yellow!50,very thick, inner sep=.5em, text centered},
	    myarrow/.style={->, >=latex', shorten >=1pt, thick},
	    myline/.style={-, =latex', shorten >=1pt, rounded corners, ultra thick},
	    mylabel/.style={text width=7em, text centered} 
	} 
	\node[mynode] (plc) {Modułowy komputer przemysłowy CX1020};  
	\node[mynode, above=of plc] (plc1) {CX1020-N010 \\ DVI/USB};
	\node[mynode, left=of plc1] (plc2) {CX1020-N000 \\ LAN};
	\node[mynode, right=of plc1] (plc3) {CX1020-0113 \\ CPU};
 	\node [fit=(plc1) (plc2) (plc3)] (fit) {}; 
 	\draw [decorate,decoration={brace,amplitude=10pt}, line width=1pt] (fit.south east) -- (fit.south west);
 		
	\node [left=of plc] (laptop) {\includegraphics[width=3cm]{images/laptop}};
	\node[below=2cm of plc] (dummy) {}; 
	\node[mynode, left=of dummy] (io) {Zestaw  \\modułów~IO}; 
 	\node[mynodemini, left=of io] (io2) {EL1004};
	\node[mynodemini, above=2mm of io2] (io1) {EL4132};  	
 	\node[mynodemini, above=2mm of io1] (io0) {CX1100}; 	 	
 	\node[mynodemini, below=2mm of io2] (io3) {EL2004}; 	 	
 	\node[mynodemini, below=2mm of io3] (io4) {EL2004};
 	\node[mynodemini, below=2mm of io4] (io5) {EL3102};
 	\node [fit=(io0) (io1) (io2) (io3) (io4) (io5)] (fit2) {};  
 	%\draw [decorate, xshift=-20pt,line width=4pt] (fit.south east) -- (fit.north east);
	\draw [decorate,decoration={brace, mirror,amplitude=10pt}, line width=1pt] (fit2.south east) -- (fit2.north east);
 	 	 	 	 	 	
	\node[mynode, below=5mm of io, text width=1.6cm] (ek) {Terminal EK1100};  
	
	\node[mynode, right=of dummy] (naped) {Napęd \\ serwomechnizmów AX5203};
	\node[below=2cm of naped] (dummy2) {}; 
	\node[mynodemini, left=2mm of dummy2, text width=4cm](silnik1){Silnik \\ AM3021-0C00-0000};
	\node[mynodemini, right=2mm of dummy2, text width=4cm](silnik2){Silnik \\ AM3021-0C40-0000};

	\draw[myline,black,dotted] (fit.south) ++(0, -0.4) -- (plc.north);	
	\draw[myline,black,dotted] (fit2.east) ++(0.4, 0) -- (io.west);
	\draw[myline,blue] (laptop.east) -- ++(-1, 0) -- (plc.west);
	
	\draw[myline,purple] (plc.south) -- (io.north);	
	\draw[myline,purple] (io.south) -- (ek.north);
	\draw[myline,purple] (io0.south) -- (io1.north);	
	\draw[myline,purple] (io1.south) -- (io2.north);
	\draw[myline,purple] (io2.south) -- (io3.north);		
	\draw[myline,purple] (io3.south) -- (io4.north);
	\draw[myline,purple] (io4.south) -- (io5.north);
				
	\draw[myline,yellow] (ek.east) -- (naped.west);	

	\draw[myline, green, bend right=10] (naped.south) to (silnik1.north);
	\draw[myline, orange, bend left=10] (naped.south) to (silnik1.north);	
	\draw[myline, green, bend right=10] (naped.south) to (silnik2.north);
	\draw[myline, orange, bend left=10] (naped.south) to (silnik2.north);	
	%\draw[<->, >=latex', shorten >=2pt, shorten <=2pt, bend right=45, thick, dashed] 
    %(io.south) to node[auto, swap] {Competition}(naped.south); 
    
    \draw [purple, line width=6] (6,-1) -- (6.5,-1); \node[text width=2cm] at (7.65,-1.3) {EtherCAT (E-bus)};    
    \draw [yellow, line width=6] (6,-2) -- (6.5,-2); \node[text width=2cm] at (7.65,-2.3) {EtherCAT (skrętka)};
    \draw [blue, line width=6] (6,-3) -- (6.5,-3); \node at (7.5,-3) {Ethernet};
    \draw [black, line width=6] (6,-3.5) -- (6.5,-3.5); \node at (7.5,-3.5) {Zasilanie};
    \draw [green, line width=6] (6,-4) -- (6.25,-4); \draw [orange, line width=6] (6.25,-4) -- (6.5,-4); 
    \node [text width=2cm] at (7.75,-4.25) {Sterowanie silnikiem};    
\end{tikzpicture} 
\caption{Schemat stanowiska typu CX.}
\label{stanowisko:cx}
\end{figure}
 %

\subsubsection{Sterownik PLC}
~~~~~~~~~~~~~~~~~~~~~~~~~~~~~~~~~~~~~~~~~~~~~~~~~~~~~~~~~~~~~~~~~~~~~~~~~~~~~~~~~~~~~~~~~~~~~
soft plc
hard plc
opisać zasadę działania plc w beckhoffie
~~~~~~~~~~~~~~~~~~~~~~~~~~~~~~~~~~~~~~~~~~~~~~~~~~~~~~~~~~~~~~~~~~~~~~~~~~~~~~~~~~~~~~~~~~~~~
W~realizacji wykorzystane zostały stanowiska firmy Beckhoff wyposażone w jednostki centralne pracujące pod kontrolą Windowsa CE (Microsoft Windows Compact Edition). Na~takiej jednostce uruchamiane są programy do~sterowania z~poziomu komputera (ang. Soft PLC). Jest to rozwiązanie alternatywne dla~klasyczny sterowników swobodnie programowalnych w~postaci dedykowanego urządzenia (ang. Hard PLC).
Taki ,,sterownik PLC w komputerze PC'' wykorzystuje standardowe języki programowania sterowników~PLC (zgodność z~normą IEC~61131-3) do~tworzenia logiki sterującej takiej jak:
\begin{itemize}
\item IL -- \textbf{I}nstruction \textbf{L}ist to~tekstowy język programowania składający się z~serii instrukcji, z~których każda zaczyna~się z~nowej linii i~zawiera operator z~jednym lub więcej argumentem (zależnie od~funkcji),

\item LD -- \textbf{L}adder \textbf{D}iagram jest graficznym językiem programowania, który swoją struktura przypomina obwód elektryczny. Doskonały do~łączenia POUs (Progam Organization Units). LD~składa~się z~sieci cewek i~styków ograniczonej przez linie prądowe. Linia z~lewej strony przekazuje wartość logiczną TRUE, z~tej strony zaczyna~się też wykonywać linia pozioma.

\item FBD -- \textbf{F}unction \textbf{B}lock \textbf{D}iagram jest graficznym językiem programowania przypominającym sieć, której elementy to~struktury reprezentujące funkcje logiczne bądź wyrażenia arytmetyczne, wywołania bloków funkcyjnych~itp.

\item SFC -- \textbf{S}equential \textbf{F}unction \textbf{C}hart to~graficzny język programowania, w~którym łatwo jest ukazać chronologię wykonywania przez program różnych procesów.

\item ST -- \textbf{S}truktured \textbf{T}ext jest tekstowym językiem programowania, złożonym z~serii instrukcji takich jak If..then lub For...do.

\item CFC -- \textbf{C}ontinuous \textbf{F}unction \textbf{C}hart jest graficznym językiem programowania, który w~przeciwieństwie do~FBD nie~działa w sieci, a~w~luźno poło onej strukturze, co~pozwala na np.~stworzenie sprzężenia zwrotnego.
\end{itemize}

\indent
\indent Stanowiska podłączone są do sieci lokalnej Ethernet w~laboratorium, więc komunikacja z~nimi odbywa się tak samo jak z~każdym innym urządzeniem sieciowym. Podstawy programowania i korzystania ze sterowników autor poznał zapoznając się z odpowiednią literaturą \cite{plc1,plc2,plc4,plc5,plc6} oraz uczęszczając w toku studiów na zajęcia obowiązkowe oraz specjalizacyjne.
Konfigurację sterowników wraz z modułami przedstawiają Rysunki~\ref{conf:cp}~oraz~\ref{conf:cx}.
\begin{figure}[!htb] 	\centering 	\includegraphics[width=0.75\textwidth]{images/confCP} \caption{Konfiguracja stanowiska typu CP} \label{conf:cp} \end{figure}
\begin{figure}[!htb] 	\centering 	\includegraphics[width=0.75\textwidth]{images/confCX} \caption{Konfiguracja stanowiska typu CX} \label{conf:cx} \end{figure}

\subsubsection{Komputer}
Projekt w całości był realizowany na laptopie autora, podłączanym do~sieci w~laboratorium. Na~komputerze uruchomiana była maszyna wirtualna. Na~jednej zainstalowane było środowisko TwinCAT do~programowania sterownika oraz do~tworzenia i~uruchamiania wizualizacji. Wizualizacje tworzone w~środowisku TwinCAT można uruchomić bezpośrednio na~komputerze wyposażonym w~odpowiednie oprogramowanie lub na~urządzeniu docelowym po~podpięciu do~niego monitora (o~ile urządzenie docelowe posiada wyjście DVI lub odpowiedni interfejs systemowy w~postaci odrębnego modułu).

\subsection{Analiza tematu}
Analiza tematu polegała przede wszystkim na zapoznaniu się z~narzędziami programistycznymi do~tworzenia oprogramowania sterownika oraz wizualizacji.
W~wyniku analizy autor poznał podstawy obsługi środowiska TwinCAT oraz jego elementów składowych a w szczególności:
\begin{itemize}
\item TwinCAT System Manager - centralne narzędzie konfiguracyjne,
\item TwinCAT PLC - narzędzie do tworzenia programów,
\item TwinCAT NC/CNC - grupa narzędzi do sterowania osiami w różnych trybach.
\end{itemize} 

\subsection{EtherCAT}
zasada działania
topologia
źródła opóźnień
wady/zalety
itd.
\begin{figure}[htbp]
 \centering
        \tikzstyle{background grid}=[draw, black!50,step=.25cm]
	\begin{tikzpicture}[node distance=1cm, auto]%, show background grid]
	\tikzset{
    	mynode/.style={rectangle,rounded corners,draw=black, top color=white, bottom color=orange!50,very thick, inner sep=0.6em, minimum size=2.5em, 		text centered, text width=2.5cm},
    	mynodemini/.style={rectangle,rounded corners,draw=black, top color=white, bottom color=orange!50,very thick, inner sep=.5em, text centered},    	
	    myarrow/.style={->, >=latex', shorten >=1pt, ultra thick},
	    myline/.style={-, =latex', shorten >=1pt, rounded corners, ultra thick},
	    mylabel/.style={text width=7em, text centered} 
	} 
	\node[mynode] (master) {Węzeł nadrzędny \\ (ang. \textit{master})};  
	\node[mynode, below right=of master] (slave1) {Węzeł podrzędny 1 \\ (ang. \textit{slave})};
	\node[mynode, right=of slave1] (slave2) {Węzeł podrzędny 2 \\ (ang. \textit{slave})};
	\node[mynode, right=of slave2] (slaven) {Węzeł podrzędny n \\ (ang. \textit{slave})};  	 	 		
	
	\draw[myarrow,black] (master.350) -| (slave1.135);
	\draw[myarrow,black] (slave1.45) -- ++(0,1.5) -| (slave2.135);
	\draw[myarrow,black,dotted] (slave2.45) -- ++(0,1.5) -| (slaven.135);	
	\draw[myarrow,black] (slaven.45) |- (master.20);	
 
\end{tikzpicture} 
\caption{Przykładowa topologia sieci}
\label{etherCAT:topologia}
\end{figure} %
\begin{figure}[htbp]
 \centering
        \tikzstyle{background grid}=[draw, black!50,step=.25cm]
	\begin{tikzpicture}[node distance=2mm, auto]%, show background grid]
	\tikzset{
    	mynode/.style={rectangle,rounded corners,draw=black, top color=white, very thick, inner sep=4mm, 		text centered,font=\footnotesize},
    	mynodemini/.style={rectangle,rounded corners,draw=black, top color=white, thick, inner sep=2mm, text centered,font=\scriptsize},    	
	    myarrow/.style={->, >=latex', shorten >=1pt, ultra thick},
	    myline/.style={-, =latex', shorten >=1pt, rounded corners, ultra thick},
	    mylabel/.style={text centered, font=\scriptsize\bfseries} 
	} 
	\node[bottom color=gray!50, mynode] (ethhdr) {Ethernet header};  
	\node[bottom color=gray!50, mynodemini, below=of ethhdr.205] (da) {DA};
	\node[mylabel, below=1mm of da] (das) {6B};	
	\node[bottom color=gray!50, mynodemini, right=of da] (sa) {SA};  
	\node[mylabel, below=1mm of sa] (sas) {6B};	
	\node[bottom color=gray!50, mynodemini, right=of sa] (typ) {Typ};	
	\node[mylabel, below=1mm of typ] (typs) {2/4B};	
	\node[mylabel, below=of sas] (ethhdradr) {EtherType 0x88A4};
	
	\node[bottom color=yellow!50, mynode, right=of ethhdr, text width=2cm] (ecat) {EtherCAT};
	\node[bottom color=yellow!50, mynodemini, below=of ecat, text width=2.4cm] (ecathdr) {EtherCAT header};
	\node[mylabel, below=1mm of ecathdr] (ecathdrs) {2B};
	
	\node[bottom color=yellow!50, mynode, right=of ecat, text width=6.3cm] (ecatt) {EtherCAT telegram};
	\node[bottom color=yellow!50, mynodemini, below=of ecatt.193] (ecatd1) {Datagram 1};  	 
	\node[mylabel, below=1mm of ecatd1] (ecatd1s) {(10+n+2)B};
	\node[bottom color=yellow!50, mynodemini, right=of ecatd1] (ecatd2) {Datagram 2};
	\node[mylabel, below=1mm of ecatd2] (ecatd2s) {(10+m+2)B};  	 	 		
	\node[bottom color=yellow!50, mynodemini, right=0.9cm of ecatd2] (ecatdn) {Datagram n};
	\node[mylabel, below=1mm of ecatdn] (ecatdns) {(10+k+2)B};
	\node [fit=(ecatd1s) (ecatdns) (ecatdns)] (fit) {};  
 	%\draw [decorate, xshift=-20pt,line width=4pt] (fit.south east) -- (fit.north east);
	\draw [decorate,decoration={brace,amplitude=10pt}, line width=1pt] (fit.south east) ++(-0.3,0.3) -- ++(-6.9,0) (fit.south west);		
	\node[mylabel, below=1mm of fit] (fits) {44--1498B};
			
	\node[bottom color=gray!50, mynode, right=of ecatt] (eth) {Ethernet};
	\node[bottom color=gray!50, mynodemini, below=of eth.222] (pad) {Pad.};
	\node[mylabel, below=1mm of pad] (pads) {0--32B};
	\node[bottom color=gray!50, mynodemini, right=of pad] (fcs) {FCS}; 	 
	\node[mylabel, below=1mm of fcs] (fcss) {4B}; 		
 
	\draw[myline,black,dotted] (ecatd2) -- (ecatdn); 	
	
	\node[mylabel, below=of ethhdradr] (dal) {DA -- Destination Address};
	\node[mylabel, right=of dal] (sal) {SA -- Source Address};
	\node[mylabel, right=of sal] (padl) {Pad. -- Payload};
	\node[mylabel, right=of padl] (fcsl) {FCS -- Frame Check Sequance (CRC)};			
	
\end{tikzpicture} 
\caption{Ramka w transmisji EtherCAT i jej podział na datagramy}
\label{etherCAT:ramka}
\end{figure} %
\begin{figure}[htbp]
 \centering
        \tikzstyle{background grid}=[draw, black!50,step=.25cm]
	\begin{tikzpicture}[node distance=2mm, auto]%, show background grid]
	\tikzset{
    	mynode/.style={rectangle,rounded corners,draw=black, top color=white, very thick, inner sep=4mm, 		text centered,font=\footnotesize},
    	mynodemini/.style={rectangle,rounded corners,draw=black, top color=white, thick, inner sep=2mm, text centered,font=\scriptsize},    	
	    myarrow/.style={->, >=latex', shorten >=1pt, ultra thick},
	    myline/.style={-, =latex', shorten >=1pt, rounded corners, ultra thick},
	    mylabel/.style={text centered, font=\scriptsize\bfseries} 
	} 
	\node[bottom color=gray!50, mynode] (datagram) {Datagram};  
	\node[bottom color=gray!50, mynode, below=1cm of datagram] (part1) {Dane};   		
	\node[bottom color=gray!50, mynode, left=of part1] (part2) {Nagłówek};   		
	\node[bottom color=gray!50, mynode, right=of part1] (part3) {Licznik};   				
 
	\draw[myline,black,dotted] (datagram.south west) -- (part2.north west); 	
	\draw[myline,black,dotted] (datagram.south east) -- (part3.north east); 
\end{tikzpicture} 
\caption{Budowa datagramu}
\label{etherCAT:ramka}
\end{figure} %

\subsection{Założenia}
Oprogramowanie dla dostępnego stanowiska laboratoryjnego powinno zostać stworzone przy użyciu środowiska TwinCAT. Funkcjonalności robota wchodzące w~skład projektu, to:
\begin{itemize}
\item sterowanie ręczne z~pilota podłączonego bezpośrednio do~sterownika,
\item sterowanie ręczne z~wizualizacji,
\item sterowanie automatyczne, 
\item wizualizacja stanu stanowiska.
\end{itemize}
\indent
\indent Powyżej zostały wymienione założenia podstawowe, jednak autor nie wyklucza zrealizowania dodatkowych zadań, które nie zostały zamieszczone w~pierwotnej koncepcji realizacji projektu.

\subsection{Plan pracy}
Realizacja projektu została podzielona na następujące etapy:
\begin{itemize}
\item Zapoznanie się ze sterownikami Beckhoff oraz oprogramowaniem TwinCAT,
\item Zapoznanie się z dostępnymi serwonapędami Beckhoff,
\item Projekt i realizacja stanowiska,
\item Przygotowanie stanowiska do współpracy z systemem wizualizacji,
\item Testowanie i uruchamianie,
\item Przedstawienie projektu i~ewentualne korekty.
\end{itemize}
\indent
\indent Powyższy plan pracy stanowił dla autora wyznacznik kolejnych działań. Jednak powszechnie wiadomo, że w~praktyce poszczególne punkty są~wymienne i~wpływają na siebie wzajemnie.
