\section{Wstęp}
Tematem projektu, którego dotyczy ta praca jest: „Projekt i~realizacja stanowiska laboratoryjnego do badania zależności czasowych w~sieci EtherCAT". Zagadnienia związane z~tworzeniem oprogramowania dla sterowników przemysłowych są dla autora niezwykle interesujące, a zrealizowany projekt miał na~celu dalsze pogłębienie jego wiedzy z~tego zakresu. Wyboru tego konkretnego tematu autor dokonał, ponieważ protokół EtherCAT jest jeszcze nowością i~według wielu źródeł stanowi przyszłość branży informatyki przemysłowej \cite{art1_etherCAT, art2_etherCAT}, a~praca nad tym tematem wydaje się być pomocna i~wartościowa w~przyszłej pracy zawodowej lub na~ewentualnym dalszym etapie kształcenia.

\subsection{Geneza}
Informatyka przemysłowa to dziedzina wiedzy z~pogranicza nauk informatycznych oraz szeroko pojętych nauk o~technologiach przemysłowych łącząca te~dwa zakresy wiedzy. Informatyka przemysłowa to~dział, który jest niezwykle pomocny, zwłaszcza przy analizach danych, ale~również podczas tworzenia funkcjonalnych i~wydajnych maszyn dla nowoczesnych gałęzi gospodarki. Specjaliści z~zakresu informatyki przemysłowej zajmują się takimi zagadnieniami jak monitorowanie produkcji przemysłowej czy też analizowaniem wybranych danych, które później służą do~porównań. Niezwykle ważne jest wykorzystanie tych danych, które dostarczą istotnych informacji dla każdego przemysłu, a~także sterowanie produkcji przemysłowej, która pomaga w~dynamicznym rozwoju każdej gałęzi w~gospodarce. 
Przemysł to~szczególna gałąź gospodarki, która aby się rozwijać potrzebuje nowoczesnych rozwiązań. Jedną z~nich jest informatyka, która pozwala na~udoskonalanie wyprodukowanych już maszyn przemysłowych, a~także tych, które dopiero są w~trakcie rozwoju technologicznego. Informatyka to aspekt, który zwraca uwagę na~nowoczesne zastosowanie komputerów w~celach przemysłowych. Wielu informatyków współpracuje z~firmami działającymi w~przemyśle, gdyż współpraca opłaca się obydwu stronom.

Systemy komputerowe realizują bardzo odpowiedzialne zadania, a~współczesna technologia stawia coraz poważniejsze wymagania związane przede wszystkim z~gwarantowanym i~nieprzekraczalnym czasem realizacji pojedynczego cyklu sterowania bądź regulacji, co wymaga wprowadzenia rozwiązań gwarantujących determinizm czasowy. Determinizm czasowy oznacza, że wyniki działania systemu komputerowego pojawiają się w~określonym i~skończonym czasie.

System czasu rzeczywistego (ang. \textit{Real-Time System}, w~skrócie \textit{RTS}) według jednej z~definicji jest to~system komputerowy, w~którym obliczenia są prowadzone równolegle z~przebiegiem zewnętrznego procesu w~otoczeniu systemu (w~sterowanym obiekcie). Mają one na celu nadzorowanie, sterowanie oraz terminowe reagowanie na~zdarzenia zachodzące w~tym procesie. Uproszczony schemat działania przedstawiono na~Rysunku~\ref{wstep:rts}.
\begin{figure}[htbp]
 \centering
        \tikzstyle{background grid}=[draw, black!50,step=.25cm]
	\begin{tikzpicture}[node distance=1cm, auto]%, show background grid]
	\tikzset{
    	mynode/.style={rectangle,rounded corners,draw=black, fill=white!15,very thick, inner sep=1.2em, minimum size=2.5em, 		text centered, text width=2.5cm},
	    myarrow/.style={->, >=latex', shorten >=1pt, ultra thick},
	    myline/.style={-, =latex', shorten >=1pt, rounded corners, ultra thick},
	    mylabel/.style={text width=7em, text centered} 
	} 
	\node[mynode] (device2) {Obiekt (otoczenie)};  
	\node[mynode, fill=white!15, left=5cm of device2] (device1) {System \\ czasu \\ rzeczywistego};
	\node[right=0.1cm of device2] (loop) {};
	
	\draw[myarrow] ([yshift=6mm]device2.west) -- ([yshift=6mm]device1.east) node [midway,yshift=3.5mm,fill=white] {Zdarzenia};		
		\draw[myarrow] ([yshift=0mm]device2.west) -- ([yshift=0mm]device1.east) node [midway,yshift=3.5mm,fill=white] {Stan systemu};
	\draw[myarrow] ([yshift=-6mm]device1.east) -- ([yshift=-6mm]device2.west) node [midway,yshift=-3.5mm,fill=white] {Odpowiedzi};		
 
\end{tikzpicture} 
\caption{Schemat działania systemu czasu rzeczywistego.}
\label{wstep:rts}
\end{figure} %

System czasu rzeczywistego jest takim systemem, który do poprawnego działania musi spełniać następujące warunki:
\begin{enumerate}
\item Warunki logiczne -- odpowiedź systemu musi być zawsze prawidłowa przy uwzględnieniu stanu systemu, tzn. odpowiedź zależy od stanu otoczenia i~występujących zdarzeń oraz jest zgodna z~założeniami technologicznymi,
\item Warunki czasowe -- odpowiedź systemu musi nadejść we właściwym czasie (przed przekroczeniem czasu granicznego dla danego systemu).
\end{enumerate}

Dla określenia podziału systemów czasu rzeczywistego przydatne jest pojęcie funkcji zysku. Jest to funkcja zależna głównie od czasu i~dokonuje odwzorowania czasu wykonania zadania na korzyść wynikającą z~jego wykonania. Wartość korzyści niekoniecznie jest wielkością wymiarowaną. Źródłem występujących ograniczeń czasowych są~zazwyczaj zjawiska fizyczne zachodzące w~świecie rzeczywistym kontrolowanym przez system. Zadanie jest uznawane za~zrealizowane jako poprawne, jeśli w~chwili jego zakończenia wartość funkcji zysku jest większa od zera.

Założenia dla wykresów funkcji zysku:
\begin{itemize}
\item $t_0$ -- jest to moment w~którym zadanie zostało zlecone oraz rozpoczęło się jego wykonywanie,
\item $t_T$ -- jest to graniczny moment w~którym przetwarzanie może zostać zakończone,
\item w~przypadku jak na Rysunku~\ref{wstęp:funkcja_zysku:3} funkcja z(t) jest funkcją ciągłą w~punktach $t_1$ i~$t_T$,
\item dla wartości $t \leq t_0$ funkcji zysku nie wyznacza się, gdyż jest to~sytuacja nie~możliwa, tzn. obliczany byłby zysk z~zakończenia zadania przed jego rozpoczęciem, co~nie~może się zdarzyć i~nie ma logicznego sensu.
\end{itemize}

W tego typu systemach przekształcanie danych wymienianych między nim, a środowiskiem zewnętrznym zachodzi w~deterministycznie określonym czasie. W~opisach stosuje się pojęcie terminu (ang. \textit{deadline}), oznaczające najdłuższy dopuszczalny czas reakcji systemu na~wystąpienie zdarzenia. Szybkość działania systemu czasu rzeczywistego nie~jest tak naprawdę bardzo ważna, istotne jest, aby spełnione były bezwarunkowo przyjęte ograniczenia czasowe.
\clearpage

\tikzset{
%Define standard arrow tip
>=stealth',
%Define style for different line styles
help lines/.style={dashed, thick},
axis/.style={<->},
important line/.style={thick},
connection/.style={thick, dotted},
}
\newcommand\A{\ensuremath{\mathcal{A}}}
\newcommand\B{\ensuremath{\mathcal{B}}}

\begin{figure}[htbp]
 \centering
	\subfloat[System typu ,,hard'']{
		\label{wstęp:funkcja_zysku:1}	
		\begin{tikzpicture}

		% horizontal axis
		\draw[->] (0,0) -- (5,0) node[anchor=north] {$t$};
		% labels
		\draw	(1.1,0) node[anchor=north east] {$t_0$}
				(3,0) node[anchor=north east] {$t_T$};
		
		% vertical axis
		\draw[->] (0,-1) -- (0,3) node[anchor=east] {$z(t)$};
		% labels
		\draw	(0.1,1.9) node[anchor=south east] {$u$}
				(0,-1.2) node[anchor=south east] {$-\infty$};
						
		% ciągłe
		\draw[ultra thick] (1,2) -- (3,2);
		
		% Przerywane
		\draw[thick,dashed] (-0.2,2) -- (1,2);
		\draw[thick,dashed] (1,-0.2) -- (1,2);
		\draw[ultra thick,dashed, <-] (3,-0.9) -- (3,2);		
		
		\draw (8,1.5) node {
		$z(t)=\left\{
			\begin{array}{c l}     
			    u & t_0<t<t_T\\
		       -\infty & t \geq t_T
			\end{array}\right.$ };
					
		\end{tikzpicture}	
	}                
	
	\subfloat[System typu ,,firm'']{ 
		\label{wstęp:funkcja_zysku:2}
		\begin{tikzpicture}

		% horizontal axis
		\draw[->] (-0.2,0) -- (5,0) node[anchor=north] {$t$};
		% labels
		\draw	(1.1,0) node[anchor=north east] {$t_0$}
				(3,0) node[anchor=north east] {$t_T$};
		
		% vertical axis
		\draw[->] (0,-0.5) -- (0,3) node[anchor=east] {$z(t)$};
		% labels
		\draw	(0.1,1.9) node[anchor=south east] {$u$}
				(0,-0.1) node[anchor=south east] {$0$};
						
		% ciągłe
		\draw[ultra thick] (1,2) -- (3,2);
		\draw[ultra thick, ->] (3,0) -- (5,0);		
		
		% Przerywane
		\draw[thick,dashed] (-0.2,2) -- (1,2);
		\draw[thick,dashed] (1,-0.2) -- (1,2);
		\draw[ultra thick,dashed] (3,0) -- (3,2);		
		
		\draw (8,1.5) node {
		$z(t)=\left\{
			\begin{array}{c l}     
			    u & t_0<t<t_T\\
		        0 & t \geq t_T
			\end{array}\right.$ };
			
		\end{tikzpicture}		
	}
	
	\subfloat[System typu ,,soft'']{
		\label{wstęp:funkcja_zysku:3}	
		\begin{tikzpicture}

		% horizontal axis
		\draw[->] (-0.2,0) -- (5,0) node[anchor=north] {$t$};
		% labels
		\draw	(1.1,0) node[anchor=north east] {$t_0$}
				(2.1,0) node[anchor=north east] {$t_1$}
				(3,0) node[anchor=north] {$t_T$};
		
		% vertical axis
		\draw[->] (0,-0.5) -- (0,3) node[anchor=east] {$z(t)$};
		% labels
		\draw	(0.1,1.9) node[anchor=south east] {$u$}
				(0,-0.1) node[anchor=south east] {$0$};
						
		% ciągłe
		\draw[ultra thick] (1,2) -- (2,2);
		\draw[ultra thick] (2,2) -- (3,0);		
		\draw[ultra thick, ->] (3,0) -- (5,0);		
		
		% Przerywane
		\draw[thick,dashed] (-0.2,2) -- (1,2);
		\draw[thick,dashed] (1,-0.2) -- (1,2);
		\draw[thick,dashed] (2,-0.2) -- (2,2);		

		\draw (8,1.5) node {
		$z(t)=\left\{
			\begin{array}{c l}     
			    u & t_0<t<t_1\\
			    u \frac{t_T-t}{t_T-t_1} & t_1<t<t_T\\
		        0 & t \geq t_T
			\end{array}\right.$ };
			
		\end{tikzpicture}
	}
\caption{Funkcja zysku dla systemów czasu rzeczywistego}
\label{wstep:funkcja_zysku}
\end{figure} %
\vspace{-0.5cm}
Jak wynika z~przedstawionych na Rysunku~\ref{wstep:funkcja_zysku} teoretycznych funkcji zysku rozróżniamy trzy typy systemów czasu rzeczywistego:
\begin{itemize}
\item systemy o~ostrych ograniczeniach czasowych (ang. \textit{hard real-time}) -- gdy przekroczenie terminu powoduje poważne, a~nawet katastrofalne skutki, jak np.~zagrożenie życia lub~zdrowia ludzi, uszkodzenie lub~zniszczenie urządzeń, przy czym nie~jest istotna wielkość przekroczenia terminu a~jedynie sam fakt jego przekroczenia. Wynika z~tego, ze informacja wygenerowana przez system po~przekroczeniu czasu granicznego $t_T$ oznacza poważny błąd systemu,

\item systemy o~mocnych ograniczeniach czasowych (ang. \textit{firm real-time}) -- gdy fakt przekroczenia terminu powoduje całkowitą nieprzydatność wypracowanego przez system wyniku, jednakże nie oznacza to zagrożenia dla ludzi lub sprzętu; pojęcie to stosowane jest głównie w~opisie teoretycznym baz danych czasu rzeczywistego,

\item systemy o~miękkich lub~łagodnych ograniczeniach czasowych (ang. \textit{soft real-time}) -- gdy przekroczenie pewnego czasu powoduje negatywne skutki tym poważniejsze, im~bardziej ten czas został przekroczony; w~tym przypadku przez ,,negatywne skutki'' rozumie się spadek funkcji zysku aż~do~osiągnięcia wartości zero.
\end{itemize}

Idealnymi przykładami praktycznymi z~życia codziennego wyjaśniającymi wprowadzony podział oraz funkcję zysku~są:
\begin{itemize}
\item W~przypadku systemów typu  ,,hard real-time'' jest to reaktor jądrowy. Zakładając, że system nadzorujący pracę takiego urządzenia nie~zdążyłby wystarczająco szybko zareagować na~podnoszenie się temperatury rdzenia i~doszłoby do~jego przegrzania lub nawet do~ częściowego stopienia. Jak pokazał przykład awarii reaktora jądrowego w~Czarnobylu tego typu zjawiska mają tragiczne i~wieloletnie skutki. Jest to~idealny przykład, że~nie pojawienie się odpowiedzi systemu na~czas może przynieść poważne konsekwencje i~nie~ma znaczenia o ile czas graniczny został przekroczony,
\item W~przypadku systemów typu  ,,soft real-time'' jest to bankomat. Jak wiadomo bankomat musi zareagować na~żądania zgłaszane przez klienta w~skończonym czasie. Jeżeli czas ten będzie zbyt długi, tzn. będzie przekraczał czas graniczny to~negatywnym skutkiem będzie spadające zadowolenie klienta wynikające z~jakości świadczonej usługi. Wiadomo, że~nie~wolno ignorować tego faktu, bo~niezadowolenie klienta może spowodować, że~nie~skorzysta on~więcej z~danej sieci bankomatów. Na~tym przykładzie dobrze widać, że przekroczenie czasu granicznego nie~jest tragiczne w~skutkach, ale powoduje spadek wartości funkcji zysku, którą jest w~tym przypadku zadowolenie klienta.
\end{itemize}

Systemy czasu rzeczywistego najczęściej pracują pod kontrolą specjalnych systemów operacyjnych spełniających reżimy czasu rzeczywistego. System operacyjny czasu rzeczywistego (ang. \textit{Real-Time Operating System}, w~skrócie \textit{RTOS}) jest to system operacyjny, który został opracowany tak, aby~spełniać wymagania narzucone na~czas wykonania zadanych operacji. Podstawowym wymaganiem dla systemu typu RTOS jest określenie najdłuższego możliwego czasu, po którym urządzenie wypracuje odpowiedź na wystąpienie zdarzenia (określenie przypadku pesymistycznego). 
%Ze względu na~kryterium ograniczeń czasowych systemy te~dzielimy na:
%\begin{itemize}
%\item Twarde (ang. \textit{Hard RTOS}) -- są to systemy, dla których znany jest najdłuższy i~nie przekraczalny gwarantowany czas odpowiedzi,
%\item Miękkie (ang. \textit{Soft RTOS}) -- są to systemy, które starają się odpowiedzieć w~możliwie najkrótszym czasie, ale niestety nie jest znany najdłuższy możliwy czas odpowiedzi.
%\end{itemize}

Zaawansowanym problemem występującym w~systemach operacyjnych tego typu jest dobór algorytmu szeregowania oraz przydziału czasu. W~systemie czasu czasu rzeczywistego ze względu na ograniczone zasoby trzeba wybrać, który z~procesów będzie miał w~danym momencie przydzielony procesor oraz jak~długo przydział taki powinien trwać, aby wszystkie wykonywane procesy spełniły określone dla nich ograniczenia czasowe.

%Na rynku dostępnych jest aktualnie wiele systemów operacyjnych czasu rzeczywistego od różnych producentów lub otwartych, których przykłady przedstawia Tabela~\ref{rtos}.
%\begin{table}[!htb]
%\begin{center}
%\begin{tabular}{| p{0.03\textwidth} | p{0.17\textwidth} | p{0.3\textwidth} | p{0.4\textwidth} |}\hline
%\textbf{Nr} & \textbf{Nazwa} & \textbf{Producent} & \textbf{Platformy} \\\hline\hline
%1 & Solaris & Sun Microsystems & Sparc \\\hline
%2 & Windows CE & Microsoft & ARM, MIPS, PowerPC, SH, x86, Strong ARM, NEC, VR4111 \\\hline
%3 & QNX Neutrino & QNX Software Systems & MIPS, MPC8xx, PowerPC, SH, ARM, Strong~RM \\\hline
%4 & RT Linux & Open Source & ARM, PowerPC, x86, SH3, MIPS \\\hline
%5 & LynxOS & LynuxWorks & 68K, MIPS, MPC8xx, PowerPC, x86, Sparc \\\hline
%6 & VxWorks & Wind River Systems & 68K, i869, ARM, MIPS, PowerPC, x86, SH, SPARC \\\hline
%7 & eCOS & Open Source & ARM, MIPS, MPC8xx, PowerPC, Sparc \\\hline
%\end{tabular}
%\end{center}
%\vspace*{-6mm}
%  \caption{Systemy czasu rzeczywistego}
%	\label{rtos}
%\end{table}

Kolejnym niezbędnym elementem, oprócz układów wejściowych i~wyjściowych oraz kontrolowanych urządzeń (np. robotów) są deterministyczne sieci transmisyjne.
W~sieciach przemysłowych dane przesyłane są~często, ale~są~one krótkie (mają niewielki rozmiar), w~przeciwieństwie do~sieci ogólnego przeznaczenia, gdzie przesyła się dane rzadko~i~w~dużej ilości. Dane mają charakter np. wartości
pomiaru, rozkazów start/stop, alarmu, przekazania wartości zadanej do~zrealizowania przez regulator itd. Bardzo często informacja przesyłana w~tych sieciach ma rozmiar pojedynczego bitu lub słowa.

W~sieciach, w~których zagwarantowany jest determinizm czasowy transmisji, okres od wysłania pakietu danych od~nadawcy do~jego odbioru po~stronie odbiorcy jest z~góry znany i~pozostaje niezmienny. Wówczas komunikacja między węzłami sieci może być realizowana w~czasie rzeczywistym, łatwiejsza jest też ich synchronizacja. Obie te kwestie są kluczowe dla sprawnego działania systemów sterowania. Dodatkowo sieci takie muszą zapewniać wszelkiego rodzaju mechanizmy kontroli poprawności przesyłu danych oraz zabezpieczać przez ewentualną ich utratą. Kolejnym wymaganiem stawianym przed sieciami tego typu jest odporność na~zakłócenia zewnętrzne, w~tym szczególnie zakłócenia elektromagnetyczne. Rola komunikacji jest krytyczna z~punktu widzenia technologii, ekonomii i~bezpieczeństwa prowadzenia procesu. \clearpage
Podsumowanie najważniejszych wymagań stawianych przemysłowym sieciom informatycznym:
\begin{itemize}
\item Niezawodność,
\item Przewidywalność procesu komunikacji:
\begin{itemize}
\item Relacja pomiędzy węzłami w~sieci,
\item Praca w~czasie rzeczywistym,
\end{itemize}
\item Efektywność w~przekazywaniu krótkich wiadomości,
\item Standaryzacja interfejsów, łatwość podłączania dużej liczby urządzeń,
\item Możliwość podłączenia do zewnętrznych sieci,
\item Łatwość lokalizowania usterek.
\end{itemize}

\vspace{5mm}
Aby sieć mogła być deterministyczna musi stosować specyficznym sposób transmisji danych.
Istnieją trzy modele transmisji gwarantujące spełnienie wymogów determinizmu czasowego \cite{kwiecien,gaj}:
\begin{itemize}
\item Master--Slave,
\item Token,
\item Producent-Dystrybutor-Konsument, w~skrócie PDK
\end{itemize}

Model Master--Slave -- bazuje na wymianie informacji pomiędzy węzłem nadrzędnym (ang. \textit{Master}), a węzłami podrzędnymi (ang. \textit{Slave}). Stacja master jest odpowiedzialna za kontrolowanie ruchu sieciowego.Wymiany odbywają się cyklicznie zgodnie ze~scenariuszem. Wymiany zawsze odbywają się pomiędzy węzłami Master i Slave, w~przypadku wymiany danych między węzłami podrzędnymi, ruch odbywa się pośrednio przez węzeł nadrzędny.

Model Token -- wszystkie stacje w sieci są równorzędne. Prawo do nadawania ma w~danym momencie tylko jeden abonent posiadający żeton (ang. \textit{Token}). Żeton jest wymieniany między kolejnymi węzłami. Stacja, która otrzyma żeton ma prawo nadawać przez określony czas, po czym bezwzględni musi przekazać go dalej.

Model PDK -- istnieją trzy rodzaje stacji. Zgodnie z~nawą dane są przekazywane od producenta poprzez dystrybutora, aż do konsumenta.

\vspace{5mm}
Rozwinięcie pojawi się tutaj
\vspace{5mm}

Zdaniem autora EtherCAT jest modelem hybrydowym.

\clearpage
Aby system składający się z~wielu komponentów był systemem czasu rzeczywistego, konieczne jest spełnianie wymogów przez każdy z~elementów składowych. W~przypadku systemów informatycznych oznacza to, że~zarówno sprzęt, system operacyjny, jak i~oprogramowanie aplikacyjne muszą gwarantować dotrzymanie zdefiniowanych ograniczeń czasowych.

Z~powyższego opisu wynika, że systemy czasu rzeczywistego są~bardzo ważnym elementem nie~tylko w~przemyśle, ale również w~życiu codziennym. Przemysłowe sieci komputerowe są~bardzo istotnym elementem tych systemów, dlatego właśnie uzasadnione jest, aby przebadać i~poznać je~dokładnie oraz ustalić jakie zależności czasowe w~nich występują.
Sieć EtherCAT będąca tematem niniejszej pracy spełnia większość wymagania stawiane nowoczesnym protokołom pracującym w~warunkach przemysłowych, dlatego autor uważa, że~warto przyjrzeć mu się bliżej i~przeprowadzić zaplanowane badania oraz znaleźć ewentualne perspektywy na~kolejne prace badawcze w~przyszłości.
\clearpage
\subsection{Stanowisko laboratoryjne}
Na potrzeby realizacji projektu wykorzystano dwa różne istniejące stanowiska laboratoryjne, które składały się~z~elementów opisanych w~Tablicy~\ref{stanowiska}.
\begin{table}[!htb]
\begin{center}
\begin{tabular}{| p{0.5\textwidth} | p{0.5\textwidth} |}\hline
Stanowisko typu CP (Rysunek~\ref{stanowisko:cp}) & Stanowisko typu CX (Rysunek~\ref{stanowisko:cx})  \\
Adres IP: 157.158.57.121 & Adres IP: 157.158.57.47  \\\hline
\begin{enumerate}[leftmargin=7mm]
\setlength{\itemsep}{5pt}
\setlength{\parskip}{0pt}
\setlength{\parsep}{0pt}
\item Serwomechanizm AM3021-0C00-0000.
\item Serwomechanizm AM3021-0C40-0000.
\item Wyspa EK1100 z~zestawem modułów~IO:
\begin{itemize}[leftmargin=3mm]
\setlength{\itemsep}{3pt}
\setlength{\parskip}{0pt}
\setlength{\parsep}{0pt}
\item Terminal sieci EtherCAT EK1100.
\item 2-kanałowy moduł wyjść analogowych EL4132,
\item 4-kanałowy moduł wejść cyfrowych EL1004,
\item 2 4-kanałowe moduły wyjść cyfrowych EL2004,
\item 2-kanałowy moduł wejść analogowych EL3102.
\end{itemize}
\item Napęd serwomechnizmów AX5203 (2~osiowy).
\item Komputer przemysłowy C6925.
\item Zasilacz.
\end{enumerate}
&
\begin{enumerate}[leftmargin=7mm]
\setlength{\itemsep}{5pt}
\setlength{\parskip}{0pt}
\setlength{\parsep}{0pt}
\item Serwomechanizm AM3021-0C00-0000.
\item Serwomechanizm AM3021-0C40-0000.
\item Zestaw modułów IO:
\begin{itemize}[leftmargin=3mm]
\setlength{\itemsep}{3pt}
\setlength{\parskip}{0pt}
\setlength{\parsep}{0pt}
\item 2-kanałowy moduł wyjść analogowych EL4132,
\item 4-kanałowy moduł wejść cyfrowych EL1004,
\item 2 4-kanałowe moduły wyjść cyfrowych EL2004,
\item 2-kanałowy moduł wejść analogowych EL3102.
\end{itemize}
\item Terminal sieci EtherCAT EK1100.
\item Napęd serwomechnizmów AX5203 (2~osiowy).
\item Modułowy komputer przemysłowy CX1020:
\begin{itemize}[leftmargin=3mm]
\setlength{\itemsep}{3pt}
\setlength{\parskip}{0pt}
\setlength{\parsep}{0pt}
\item Interfejs USB/DVI CX1020-N010 ,
\item Ethernet CX1020-N000,
\item CPU CX1020-0113,
\item Zasilacz CPU i~magistrali I/O CX1100.
\end{itemize}
\item Zasilacz.
\end{enumerate}
\\\hline                                            
\end{tabular}
\end{center}
\vspace*{-6mm}
  \caption{Dostępne stanowiska laboratoryjne}
	\label{stanowiska}
\end{table}

\begin{figure}[htbp]
 \centering
        \tikzstyle{background grid}=[draw, black!50,step=.25cm]
	\begin{tikzpicture}[node distance=1cm, auto, show background grid]
	\tikzset{
    	mynode/.style={rectangle,rounded corners,draw=black, top color=white, bottom color=yellow!50,very thick, inner sep=1em, minimum size=3em, 		text centered, text width=3cm},
	    myarrow/.style={->, >=latex', shorten >=1pt, thick},
	    myline/.style={-, =latex', shorten >=1pt, rounded corners, ultra thick},
	    mylabel/.style={text width=7em, text centered} 
	} 
	\node[mynode] (plc) {Komputer \\ przemysłowy C6925};  
	\node [left=of plc] (laptop) {\includegraphics[width=3cm]{images/laptop}};
	\node[mynode, right=of plc] (zasilacz) {Zasilacz};
	\node[below=3cm of plc] (dummy) {}; 
	\node[mynode, left=of dummy] (io) {Wyspa EK1100 z~zestawem modułów IO};  
	\node[mynode, right=of dummy] (naped) {Napęd \\ serwomechnizmów AX5203};
	\node[below=3cm of naped] (dummy2) {}; 
	\node[mynode, left=of dummy2, text width=4cm](silnik1){Silnik \\ AM3021-0C00-0000};
	\node[mynode, right=of dummy2, text width=4cm](silnik2){Silnik \\ AM3021-0C00-0000};
	
	\draw[myline,blue] (laptop.east) -- ++(-1, 0) -- (plc.west);
	
	\draw[myline,black] (zasilacz.west) -- (plc.east);
	\draw[myline,black] (zasilacz.south) -- (io.north);
	\draw[myline,black] (zasilacz.south) -- (naped.north);	
	
	\draw[myline,yellow] (plc.south) -- (io.north);	
	\draw[myline,yellow] (io.east) -- (naped.west);	

	\draw[myline, green, bend right=10] (naped.south) to (silnik1.north);
	\draw[myline, orange, bend left=10] (naped.south) to (silnik1.north);	
	\draw[myline, green, bend right=10] (naped.south) to (silnik2.north);
	\draw[myline, orange, bend left=10] (naped.south) to (silnik2.north);	
	%\draw[<->, >=latex', shorten >=2pt, shorten <=2pt, bend right=45, thick, dashed] 
    %(io.south) to node[auto, swap] {Competition}(naped.south); 
    
    \draw [yellow, line width=6] (6,-2) -- (6.5,-2); \node at (7.5,-2) {EtherCAT};
    \draw [blue, line width=6] (6,-2.5) -- (6.5,-2.5); \node at (7.5,-2.5) {Ethernet};
    \draw [black, line width=6] (6,-3) -- (6.5,-3); \node at (7.5,-3) {Zasilanie};
    \draw [green, line width=6] (6,-3.5) -- (6.25,-3.5); \draw [orange, line width=6] (6.25,-3.5) -- (6.5,-3.5); 
    \node [text width=2cm] at (7.75,-3.75) {Sterowanie silnikiem};    
\end{tikzpicture} 
\caption{Schemat stanowiska typu CP}
\label{stanowisko:cp}
\end{figure} %
\begin{figure}[htbp]
 \centering
        \tikzstyle{background grid}=[draw, black!50,step=.5cm]
	\begin{tikzpicture}[node distance=1cm, auto]%, show background grid]
	\tikzset{
    	mynode/.style={rectangle,rounded corners,draw=black, top color=white, bottom color=yellow!50,very thick, inner sep=1em, minimum size=3em, 		text centered, text width=3cm},
    	mynodemini/.style={rectangle,rounded corners,draw=black, top color=white, bottom color=yellow!50,very thick, inner sep=.5em, text centered},
	    myarrow/.style={->, >=latex', shorten >=1pt, thick},
	    myline/.style={-, =latex', shorten >=1pt, rounded corners, ultra thick},
	    mylabel/.style={text width=7em, text centered} 
	} 
	\node[mynode] (plc) {Modułowy komputer przemysłowy CX1020};  
	\node[mynode, above=of plc] (plc1) {CX1020-N010 \\ DVI/USB};
	\node[mynode, left=of plc1] (plc2) {CX1020-N000 \\ LAN};
	\node[mynode, right=of plc1] (plc3) {CX1020-0113 \\ CPU};
 	\node [fit=(plc1) (plc2) (plc3)] (fit) {}; 
 	\draw [decorate,decoration={brace,amplitude=10pt}, line width=1pt] (fit.south east) -- (fit.south west);
 		
	\node [left=of plc] (laptop) {\includegraphics[width=3cm]{images/laptop}};
	\node[below=2cm of plc] (dummy) {}; 
	\node[mynode, left=of dummy] (io) {Zestaw  \\modułów~IO}; 
 	\node[mynodemini, left=of io] (io2) {EL1004};
	\node[mynodemini, above=2mm of io2] (io1) {EL4132};  	
 	\node[mynodemini, above=2mm of io1] (io0) {CX1100}; 	 	
 	\node[mynodemini, below=2mm of io2] (io3) {EL2004}; 	 	
 	\node[mynodemini, below=2mm of io3] (io4) {EL2004};
 	\node[mynodemini, below=2mm of io4] (io5) {EL3102};
 	\node [fit=(io0) (io1) (io2) (io3) (io4) (io5)] (fit2) {};  
 	%\draw [decorate, xshift=-20pt,line width=4pt] (fit.south east) -- (fit.north east);
	\draw [decorate,decoration={brace, mirror,amplitude=10pt}, line width=1pt] (fit2.south east) -- (fit2.north east);
 	 	 	 	 	 	
	\node[mynode, below=5mm of io, text width=1.6cm] (ek) {Terminal EK1100};  
	
	\node[mynode, right=of dummy] (naped) {Napęd \\ serwomechnizmów AX5203};
	\node[below=2cm of naped] (dummy2) {}; 
	\node[mynodemini, left=2mm of dummy2, text width=4cm](silnik1){Silnik \\ AM3021-0C00-0000};
	\node[mynodemini, right=2mm of dummy2, text width=4cm](silnik2){Silnik \\ AM3021-0C40-0000};

	\draw[myline,black,dotted] (fit.south) ++(0, -0.4) -- (plc.north);	
	\draw[myline,black,dotted] (fit2.east) ++(0.4, 0) -- (io.west);
	\draw[myline,blue] (laptop.east) -- ++(-1, 0) -- (plc.west);
	
	\draw[myline,purple] (plc.south) -- (io.north);	
	\draw[myline,purple] (io.south) -- (ek.north);
	\draw[myline,purple] (io0.south) -- (io1.north);	
	\draw[myline,purple] (io1.south) -- (io2.north);
	\draw[myline,purple] (io2.south) -- (io3.north);		
	\draw[myline,purple] (io3.south) -- (io4.north);
	\draw[myline,purple] (io4.south) -- (io5.north);
				
	\draw[myline,yellow] (ek.east) -- (naped.west);	

	\draw[myline, green, bend right=10] (naped.south) to (silnik1.north);
	\draw[myline, orange, bend left=10] (naped.south) to (silnik1.north);	
	\draw[myline, green, bend right=10] (naped.south) to (silnik2.north);
	\draw[myline, orange, bend left=10] (naped.south) to (silnik2.north);	
	%\draw[<->, >=latex', shorten >=2pt, shorten <=2pt, bend right=45, thick, dashed] 
    %(io.south) to node[auto, swap] {Competition}(naped.south); 
    
    \draw [purple, line width=6] (6,-1) -- (6.5,-1); \node[text width=2cm] at (7.65,-1.3) {EtherCAT (E-bus)};    
    \draw [yellow, line width=6] (6,-2) -- (6.5,-2); \node[text width=2cm] at (7.65,-2.3) {EtherCAT (skrętka)};
    \draw [blue, line width=6] (6,-3) -- (6.5,-3); \node at (7.5,-3) {Ethernet};
    \draw [black, line width=6] (6,-3.5) -- (6.5,-3.5); \node at (7.5,-3.5) {Zasilanie};
    \draw [green, line width=6] (6,-4) -- (6.25,-4); \draw [orange, line width=6] (6.25,-4) -- (6.5,-4); 
    \node [text width=2cm] at (7.75,-4.25) {Sterowanie silnikiem};    
\end{tikzpicture} 
\caption{Schemat stanowiska typu CX.}
\label{stanowisko:cx}
\end{figure}
 %
\vspace{-0.7cm}
\subsubsection{Sterownik PLC}
W~realizacji wykorzystane zostały stanowiska firmy Beckhoff wyposażone w~jednostki centralne pracujące pod kontrolą systemu Windows CE (Microsoft Windows Compact Edition). Na~jednostce takiej uruchamiane są programy do~sterowania z~poziomu komputera (ang. \textit{Soft PLC}). Jest to rozwiązanie alternatywne dla~klasyczny sterowników swobodnie programowalnych w~postaci dedykowanego urządzenia (ang.~\textit{Hard~PLC}), nazywanych przez niektórych prawdziwymi (ang.~\textit{True~PLC}).
Koncepcja~ta powstała i~jest rozwijana, ponieważ te~klasyczne sterowniki posiadają zbyt małe możliwości obliczeniowe oraz szybkość działania jednostki centralnej. W~tradycyjnych rozwiązaniach niestety zwiększanie tych możliwości (ilość dostępnej pamięci oraz szybkości działania) powoduje bardzo szybki wzrost ceny gotowego urządzenia.
Niezbędnym elementem konfiguracji zestawu, który zostaje przekształcony w~,,soft PLC'' jest karta komunikacyjna umożliwiająca połączenie sterownika z~modułami sygnałowymi i~wykonawczymi na~obiekcie z~wykorzystaniem sieci przemysłowej.
Przykładowe zastosowanie programu do~sterowania z~poziomu komputera zostało szczegółowo opisane w~\cite{art_softPLC}.
Tego typu rozwiązanie~ma następujące zalety:
\begin{itemize}
\item duże zwiększenie możliwości obliczeniowych przy stosunkowo niewielkim wzroście kosztów,
\item możliwość integracji PLC i~systemu SCADA na~jednym urządzeniu (podobnie jak w~panelach operatorskich ze~zintegrowanymi sterownikami~PLC),
\item możliwość zastosowania istniejącej infrastruktury na~obiekcie w~przypadku przebudowy; należy jedynie podmienić istniejący sterownik typu ,,hard'' na~jednostkę wyposażoną w~odpowiedni moduł komunikacyjny,
\item teoretycznie możliwość zastosowania istniejącego oprogramowania z~jednostki ,,hard PLC'', po modyfikacji ewentualnych różnic między systemami.
\end{itemize}

Taki ,,sterownik PLC w~komputerze PC'' wykorzystuje standardowe języki programowania sterowników~PLC (zgodność z~normą IEC~61131-3) do~tworzenia logiki sterującej takiej jak:
\begin{itemize}
\item IL -- \textbf{I}nstruction \textbf{L}ist to~tekstowy język programowania składający się z~serii instrukcji, z~których każda zaczyna~się z~nowej linii i~zawiera operator z~jednym lub więcej argumentem (zależnie od~funkcji),

\item LD -- \textbf{L}adder \textbf{D}iagram jest graficznym językiem programowania, który swoją strukturą przypomina obwód elektryczny. Doskonały do~łączenia POUs (Progam Organization Units). LD~składa~się z~sieci cewek i~styków ograniczonej przez linie prądowe. Linia z~lewej strony przekazuje wartość logiczną TRUE, z~tej strony zaczyna~się też wykonywać linia pozioma,

\item FBD -- \textbf{F}unction \textbf{B}lock \textbf{D}iagram jest graficznym językiem programowania przypominającym sieć, której elementy to~struktury reprezentujące funkcje logiczne bądź wyrażenia arytmetyczne, wywołania bloków funkcyjnych~itp,

\item SFC -- \textbf{S}equential \textbf{F}unction \textbf{C}hart to~graficzny język programowania, w~którym łatwo jest ukazać chronologię wykonywania przez program różnych procesów,

\item ST -- \textbf{S}tructured \textbf{T}ext jest tekstowym językiem programowania, złożonym z~serii instrukcji takich jak If..then lub For...do,

\item CFC -- \textbf{C}ontinuous \textbf{F}unction \textbf{C}hart jest graficznym językiem programowania, który w~przeciwieństwie do~FBD nie~działa w~sieci, a~w~luźno położonej strukturze, co~pozwala na np.~stworzenie sprzężenia zwrotnego.
\end{itemize}

\indent
\indent Stanowiska podłączone są do sieci lokalnej Ethernet w~laboratorium, więc komunikacja z~nimi odbywa się tak samo jak z~każdym innym urządzeniem sieciowym. Podstawy programowania i~korzystania ze sterowników autor poznał zapoznając się z~odpowiednią literaturą \cite{plc1,plc2,plc4,plc5,plc6} oraz uczęszczając w~toku studiów na zajęcia obowiązkowe oraz specjalizacyjne.

\subsubsection{Komputer}
Projekt w~całości był realizowany na laptopie autora, podłączanym do~sieci w~laboratorium. Na~komputerze uruchomiana była maszyna wirtualna. Na~jednej zainstalowane było środowisko TwinCAT do~programowania sterownika oraz do~tworzenia i~uruchamiania wizualizacji. Wizualizacje tworzone w~środowisku TwinCAT można uruchomić bezpośrednio na~komputerze wyposażonym w~odpowiednie oprogramowanie lub na~urządzeniu docelowym po~podpięciu do~niego monitora (o~ile urządzenie docelowe posiada wyjście DVI lub odpowiedni interfejs systemowy w~postaci odrębnego modułu).

\subsection{Plan pracy}
Realizacja projektu została podzielona na następujące etapy:
\begin{itemize}
\item Zapoznanie się ze sterownikami Beckhoff oraz oprogramowaniem TwinCAT,
\item Zapoznanie się z~dostępnymi serwonapędami Beckhoff,
\item Projekt i~realizacja stanowiska,
\item Skonfigurowanie stanowiska do współpracy z~systemem wizualizacji,
\item Testowanie i~uruchamianie,
\item Prezentacja projektu i~ewentualne korekty.
\end{itemize}
\indent
\indent Powyższy plan pracy stanowił dla autora wyznacznik kolejnych działań. W~praktyce poszczególne punkty są~wymienne i~wpływają na siebie wzajemnie.