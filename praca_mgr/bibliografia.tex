\section{Bibliografia}
Literatura, która została wykorzystana przez autora w czasie powstawania projektu, którą opisuje niniejsza dokumentacja.

\begin{thebibliography}{99}
\bibitem{kwiecien} 
Andrzej Kwiecień:
\emph{,,Analiza przepływu informacji w~komputerowych sieciach przemysłowych.''},
ZN Pol. Śl. s. Studia Informaitca z. 22, Gliwice 2002.

\bibitem{gaj} 
Piotr Gaj
\emph{,,Zastosowanie protokołu TCP/IP do transmisji informacji dla potrzeb przemysłowych systemów kontrolno-nadzorczych.''},
Gliwice,
2004.

\bibitem{rts} 
Zakład Systemów Zasilania:
\emph{,,Rdzeń modułowego system czasu rzeczywistego do profesjonalnych aplikacji hybrydowych systemów zasilania i systemów automatycznego nadzoru.''},
Warszawa,
grudzień 2007. 
 
\bibitem{plc1} 
Jerzy Kasprzyk: 
\emph{,,Programowanie sterowników przemysłowych''},
Wydawnictwa Naukowo-Techniczne WNT, 
Warszawa, 
2007.      

\bibitem{plc2} 
\emph{"Programowalne sterowniki PLC w systemach sterowania przemysłowego"}, 
Politechnika Radomska, 
Radom,
2001.

\bibitem{plc4} 
Andrzej Maczyński:
\emph{„Sterowniki programowalne PLC. Budowa systemu i podstawy programowania”},
Astor, 
Kraków,
2001. 

\bibitem{plc5} 
Zbigniew Seta: 
\emph{„Wprowadzenie do zagadnień sterowania. Wykorzystanie programowalnych sterowników logicznych PLC.”},
MIKOM Wydawnictwo, 
Warszawa,
2002. 

\bibitem{plc6} 
Janusz Kwaśniewski: 
\emph{„Programowalne sterowniki przemysłowe w systemach sterowania”}, 
Wyd. AGH, 
Kraków,
1999.

%\bibitem{beck1} 
%Dokumentacja producenta: 
%\emph{„Servo Drive AX5000”}, 
%28 wrzesień 2012.
%
\bibitem{kurs1} 
Materiały szkoleniowe: EtherCAT Technology Group:
\emph{,,EtherCAT for Factory Networking.EtherCAT Automation Protocol (EAP).''},
lipiec 2010.

\bibitem{kurs2} 
Materiały szkoleniowe Beckhoff:
\emph{,,Podstawy obsługi programu TwinCAT System Manager Część 1.''},
Warszawa,
2009.

%
%\bibitem{kurs3} 
%Materiały szkoleniowe:
%\emph{„Podstawy obsługi programów: TwinCAT System Manager i TwinCAT PLC Control”},
%15 grudzień 2006.

\bibitem{art1_etherCAT} 
Andrzej Gawryluk:
\emph{„EtherCAT – to nie takie trudne. Ethernet jako sieć real-time”},
Elektronika Praktyczna, 2/2010.

\bibitem{art2_etherCAT} 
Maneesh Soni:
\emph{,,EtherCAT w praktyce. Zastosowanie mikrokontrolera Sitara do implementacji EtherCAT.''},
Elektronika Praktyczna, 7/2012.

\bibitem{art3_etherCAT} 
Michał Gosk:
\emph{„Szybkość, niezawodność, doskonała synchronizacja. EtherCAT - system przyszłości.”},
Magazyn Sensor, 1/2013, kwiecień.

\bibitem{art4_etherCAT} 
Monika Jaworowska:
\emph{,,Determinizm czasowy transmisji w~Ethernecie przemysłowym.''},
Portal branżowy dla Automatyków -- AutomatykaB2B, 
[online],
8 listopada 2012,
[dostęp 28 sierpnia 2013].
Dostępny w Internecie: \\
http://automatykab2b.pl/tematmiesiaca/5061-determinizm-czasowy-transmisji-w-ethernecie-przemyslowym

\bibitem{art5_etherCAT} 
Monika Jaworowska:
\emph{,,Sieci przemysłowe w standardzie EtherCat.''},
Portal branżowy dla Automatyków -- AutomatykaB2B, 
[online],
20 czerwca 2012,
[dostęp 26 sierpnia 2013].
Dostępny w Internecie: \\
http://elektronikab2b.pl/technika/16954-sieci-przemyslowe-w-standardzie-ethercat

\bibitem{art6_etherCAT} 
Beckhoff Automation:
\emph{,,EtherCAT -- przyszłość sieci przemysłowych.''},
Portal branżowy dla Automatyków -- AutomatykaB2B, 
[online],
28 czerwca 2011,
[dostęp 22 sierpnia 2013].
Dostępny w Internecie: \\
http://automatykab2b.pl/prezentacja-artykul/3898-ethercat---przyszlosc-sieci-przemyslowych

\bibitem{art7_etherCAT} 
Beckhoff Automation:
\emph{,,EtherCAT i światłowody -- szybko i skutecznie.''},
Portal branżowy dla Automatyków -- AutomatykaB2B, 
[online],
2 listopada 2011,
[dostęp 22 sierpnia 2013].
Dostępny w Internecie: \\
http://automatykab2b.pl/prezentacja-artykul/4159-ethercat-i-swiatlowody---szybko-i-skutecznie

\bibitem{art8_etherCAT} 
Beckhoff Automation:
\emph{,,Nowa, szybka sieć EtherCAT.''},
Portal branżowy dla Automatyków -- AutomatykaB2B, 
[online],
2 listopada 2011,
[dostęp 22 sierpnia 2013].
Dostępny w Internecie: \\
http://automatykab2b.pl/prezentacja-artykul/4159-ethercat-i-swiatlowody---szybko-i-skutecznie

\bibitem{art9_etherCAT} 
Real Time Automation:
\emph{,,EtherCAT Open Real-Time Ethernet Network.''},
Portal Real Time Automation, 
[online],
2009,
[dostęp 2 września 2013].
Dostępny w Internecie: \\
http://www.rtaautomation.com/ethercat/

\bibitem{ETG_doc}
EtherCAT Technology Group:
\emph{,,EtherCAT -- the Ethernet Fieldbus.''},
Oficjalna strona internetowa EtherCAT Technology Group,
[online],
[dostęp 30 sierpnia 2013].
Dostępny w Internecie: \\
http://www.ethercat.org/en/technology.html

\bibitem{art_softPLC}
Krzysztof Oprzędkiewicz:
\emph{,,Realizacja predyktora Smitha na platformie sprzętowo-programowej ,,soft PLC'' bazującej na komputerze klasy PC''},
Automatyka~/~Akademia Górniczo-Hutnicza im. Stanisława Staszica w Krakowie,
2006,
t. 10, z. 2, s. 177--186,
ISSN 1429-3447.

\bibitem{projekt_FPGA}
Rafał Cupek, Piotr Piękoś, Marcin Poczobut, Adam Ziebinski:
\emph{,,FPGA based ,,Intelligent Tap'' device for real-time Ethernet network monitoring''},
Computer Networks, Communications in Computer and Information Science, Springer-Verlag Berlin Heidelberg,
2010,
t. 79, s. 58-66,
ISBN 978-3-642-13860-7.

\bibitem{FPGA_Xilinx}
Dokumentacja producenta: 
\emph{,,ET1815 / ET1817 EtherCAT Slave Controller IP Core for Xilinx FPGAs IP Core Release 2.02a''}, 
Wersja 2.2.1,
1 wrzesień 2008.

\bibitem{FPGA_Altera}
Dokumentacja producenta: 
\emph{,,ET1810 / ET1812 EtherCAT Slave Controller IP Core for Altera FPGAs IP Core Release 2.4.0''}, 
Wersja 1.0
15 marca 2011.

\bibitem{silniki}
Dokumentacja producenta: 
\emph{,,AX5000 – Motion control for high dynamic positioning''},
29 października 2007.

\bibitem{err0x707}
Dokumentacja producenta: 
\emph{,,AX5805 TwinSAFE drive option card for the AX5000 servo drive''},
11 marca 2013

\bibitem{etherlab}
Ingenieurgemeinschaft IgH:
\emph{,,EtherLab Technology Description.''},
Esse,
luty 2012.

\bibitem{ETG_etherlab}
EtherCAT Technology Group:
\emph{,,EtherCAT Master for Linux as part of EtherLab®.''},
Oficjalna strona internetowa EtherCAT Technology Group,
[online],
[dostęp 10 września 2013].
Dostępny w Internecie: \\
http://www.ethercat.org/en/products/263FCDB7639F4DDFB3587C18F46BA289.htm

\bibitem{ieee}
IEEE 1588-2002: IEEE Standard for a Precision Clock Synchronization Protocol for Networked Measurement and Control Systems.

%\bibitem{} 
%\emph{,,''},

\end{thebibliography}