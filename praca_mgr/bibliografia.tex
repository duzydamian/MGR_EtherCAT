\section{Bibliografia}
Literatura, która została wykorzystana przez autora w czasie powstawania projektu, którą opisuje niniejsza dokumentacja.

\begin{thebibliography}{99}
\bibitem{plc1} 
Jerzy Kasprzyk: 
\emph{"Programowanie sterowników przemysłowych"},
Wydawnictwa Naukowo-Techniczne WNT, 
Warszawa, 
2007.      

\bibitem{plc2} 
\emph{"Programowalne sterowniki PLC w systemach sterowania przemysłowego"}, 
Politechnika Radomska, 
Radom,
2001.

\bibitem{plc4} 
Andrzej Maczyński:
\emph{„Sterowniki programowalne PLC. Budowa systemu i podstawy programowania”},
Astor, 
Kraków,
2001. 

\bibitem{plc5} 
Zbigniew Seta: 
\emph{„Wprowadzenie do zagadnień sterowania. Wykorzystanie programowalnych sterowników logicznych PLC.”},
MIKOM Wydawnictwo, 
Warszawa,
2002. 

\bibitem{plc6} 
Janusz Kwaśniewski: 
\emph{„Programowalne sterowniki przemysłowe w systemach sterowania”}, 
Wyd. AGH, 
Kraków,
1999.

%\bibitem{beck1} 
%Dokumentacja producenta: 
%\emph{„Servo Drive AX5000”}, 
%28 wrzesień 2012.
%
%\bibitem{kurs1} 
%Materiały szkoleniowe: Rudolf W. Meier:
%\emph{„AX5000 – Motion control for high dynamic positioning”},
%luty 2007.
%
%\bibitem{kurs2} 
%Materiały szkoleniowe:
%\emph{„Pierwsze kroki w TwinCAT System Manager i TwinCAT PLC Control”},
%29 październik 2007.
%
%\bibitem{kurs3} 
%Materiały szkoleniowe:
%\emph{„Podstawy obsługi programów: TwinCAT System Manager i TwinCAT PLC Control”},
%15 grudzień 2006.

\bibitem{art1_etherCAT} 
Andrzej Gawryluk:
\emph{„EtherCAT – to nie takie trudne. Ethernet jako sieć real-time”},
Elektronika Praktyczna, 2/2010.

\bibitem{art2_etherCAT} 
Michał Gosk:
\emph{„Szybkość, niezawodność, doskonała synchronizacja. EtherCAT - system przyszłości.”},
Magazyn Sensor, 1/2013, kwiecień.

\bibitem{art_softPLC}
Krzysztof Oprzędkiewicz:
\emph{,,Realizacja predyktora Smitha na platformie sprzętowo-programowej ,,soft PLC'' bazującej na komputerze klasy PC''},
Automatyka~/~Akademia Górniczo-Hutnicza im. Stanisława Staszica w Krakowie,
2006,
t. 10, z. 2, s. 177--186,
ISSN 1429-3447.

\bibitem{projekt_FPGA}
Rafał Cupek, Piotr Piękoś, Marcin Poczobut, Adam Ziebinski:
\emph{,,FPGA based ,,Intelligent Tap'' device for real-time Ethernet network monitoring''},
Computer Networks, Communications in Computer and Information Science, Springer-Verlag Berlin Heidelberg,
2010,
t. 79, s. 58-66,
ISBN 978-3-642-13860-7.

\bibitem{FPGA_Xilinx}
Dokumentacja producenta: 
\emph{,,ET1815 / ET1817 EtherCAT Slave Controller IP Core for Xilinx FPGAs IP Core Release 2.02a”}, 
Wersja 2.2.1,
1 wrzesień 2008.

\bibitem{FPGA_Altera}
Dokumentacja producenta: 
\emph{„ET1810 / ET1812 EtherCAT Slave Controller IP Core for Altera FPGAs IP Core Release 2.4.0”}, 
Wersja 1.0
15 marca 2011.

\end{thebibliography}