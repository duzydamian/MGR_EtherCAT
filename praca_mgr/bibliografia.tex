\section{Bibliografia}
Literatura, która została wykorzystana przez autora w czasie powstawania projektu, którą opisuje niniejsza dokumentacja.

\begin{thebibliography}{99}
%\begin{enumerate}
%\item 
\bibitem{plc1} 
Jerzy Kasprzyk: 
\emph{"Programowanie sterowników przemysłowych"},
Wydawnictwa Naukowo-Techniczne WNT, 
Warszawa, 
2007.      

\bibitem{plc2} 
\emph{"Programowalne sterowniki PLC w systemach sterowania przemysłowego"}, 
Politechnika Radomska, 
Radom,
2001.

\bibitem{plc4} 
Andrzej Maczyński:
\emph{„Sterowniki programowalne PLC. Budowa systemu i podstawy programowania”},
Astor, 
Kraków,
2001. 

\bibitem{plc5} 
Zbigniew Seta: 
\emph{„Wprowadzenie do zagadnień sterowania. Wykorzystanie programowalnych sterowników logicznych PLC.”},
MIKOM Wydawnictwo, 
Warszawa,
2002. 

\bibitem{plc6} 
Janusz Kwaśniewski: 
\emph{„Programowalne sterowniki przemysłowe w systemach sterowania”}, 
Wyd. AGH, 
Kraków,
1999.

\bibitem{beck1} 
Dokumentacja producenta: 
\emph{„Servo Drive AX5000”}, 
28 wrzesień 2012.

\bibitem{kurs1} 
Materiały szkoleniowe: Rudolf W. Meier:
\emph{„AX5000 – Motion control for high dynamic positioning”},
luty 2007.

\bibitem{kurs2} 
Materiały szkoleniowe:
\emph{„Pierwsze kroki w TwinCAT System Manager i TwinCAT PLC Control”},
29 październik 2007.

\bibitem{kurs3} 
Materiały szkoleniowe:
\emph{„Podstawy obsługi programów: TwinCAT System Manager i TwinCAT PLC Control”},
15 grudzień 2006.

\bibitem{art1} 
Andrzej Gawryluk:
\emph{„EtherCAT – to nie takie trudne. Ethernet jako sieć real-time”},
Elektronika Praktyczna, 2/2010.

\bibitem{art2} 
Michał Gosk:
\emph{„Szybkość, niezawodność, doskonała synchronizacja. EtherCAT - system przyszłości.”},
Magazyn Sensor, 1/2013, kwiecień.
\end{thebibliography}

