\section{Bibliografia}
Literatura, która została wykorzystana przez autora w czasie powstawania projektu, którą opisuje niniejsza dokumentacja.

\begin{thebibliography}{99}
%\begin{enumerate}
%\item 
\bibitem{plc1} 
Jerzy Kasprzyk: 
\emph{"Programowanie sterowników przemysłowych"},
Wydawnictwa Naukowo-Techniczne WNT, 
Warszawa, 
2007      

\bibitem{plc2} 
\emph{"Programowalne sterowniki PLC w systemach sterowania przemysłowego"}, 
Politechnika Radomska, 
Radom,
2001

\bibitem{plc4} 
Andrzej Maczyński:
\emph{„Sterowniki programowalne PLC. Budowa systemu i podstawy programowania”},
Astor, 
Kraków,
2001 

\bibitem{plc5} 
Zbigniew Seta: 
\emph{„Wprowadzenie do zagadnień sterowania. Wykorzystanie programowalnych sterowników logicznych PLC.”},
MIKOM Wydawnictwo, 
Warszawa,
2002 

\bibitem{plc6} 
Janusz Kwaśniewski: 
\emph{„Programowalne sterowniki przemysłowe w systemach sterowania”}, 
Wyd. AGH, 
Kraków,
1999

\bibitem{step1} 
SIMATIC S7 - 
\emph{„Podstawy programowania w STEP 7”}, 
Wydanie 3, 
Warszawa, 
2010

\bibitem{step2} 
Dokumentacja producenta: 
\emph{„SIMATIC Working with STEP 7 - Getting Started Edition”}, 
marzec 2006.

\bibitem{step3} 
Dokumentacja producenta:
\emph{„SIMATIC Programming with STEP 7 V5.3 - Manual~Edition”}, 
styczeń 2004

\bibitem{scl1} 
Dokumentacja producenta: 
\emph{„SIMATIC S7-SCL V5.3 for S7-300/400 - Getting Started Release”}, 
styczeń 2005

\bibitem{scl2} 
Dokumentacja producenta: 
\emph{„SIMATIC S7-SCL V5.3 for S7-300/400 - Manual Edi- \hspace*{1mm} tion”}, 
styczeń 2005

\bibitem{scl3} 
Dokumentacja producenta: 
\emph{„SIMATIC Structured Control Language for S7-300/400 \hspace*{0.5mm} Programming - Manual”}

\bibitem{robot1} 
Dokumentacja robota - FischerTechnik

\bibitem{robot2} 
Instrukcja Laboratoryjna Robota Fischertechnik

\bibitem{kurs1} 
Materiały szkoleniowe:
„SIMATIC S7 - Kurs podstawowy”

\bibitem{kurs2} 
Materiały szkoleniowe:
„SIMATIC S7 - Kurs zaawansowany”

\bibitem{kurs3} 
Materiały szkoleniowe:
„WinCC - Kurs podstawowy”

\end{thebibliography}

