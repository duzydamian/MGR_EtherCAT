\begin{figure}[htbp]
 \centering
        \tikzstyle{background grid}=[draw, black!50,step=.25cm]
	\begin{tikzpicture}[node distance=1cm, auto, show background grid]
	\tikzset{
    	mynode/.style={rectangle,rounded corners,draw=black, top color=white, bottom color=yellow!50,very thick, inner sep=1em, minimum size=3em, 		text centered, text width=3cm},
	    myarrow/.style={->, >=latex', shorten >=1pt, thick},
	    myline/.style={-, =latex', shorten >=1pt, rounded corners, ultra thick},
	    mylabel/.style={text width=7em, text centered} 
	} 
	\node[mynode] (plc) {Komputer \\ przemysłowy C6925};  
	\node [left=of plc] (laptop) {\includegraphics[width=3cm]{images/laptop}};
	\node[mynode, right=of plc] (zasilacz) {Zasilacz};
	\node[below=3cm of plc] (dummy) {}; 
	\node[mynode, left=of dummy] (io) {Wyspa EK1100 z~zestawem modułów IO};  
	\node[mynode, right=of dummy] (naped) {Napęd \\ serwomechnizmów AX5203};
	\node[below=3cm of naped] (dummy2) {}; 
	\node[mynode, left=of dummy2, text width=4cm](silnik1){Silnik \\ AM3021-0C00-0000};
	\node[mynode, right=of dummy2, text width=4cm](silnik2){Silnik \\ AM3021-0C00-0000};
	
	\draw[myline,blue] (laptop.east) -- ++(-1, 0) -- (plc.west);
	
	\draw[myline,black] (zasilacz.west) -- (plc.east);
	\draw[myline,black] (zasilacz.south) -- (io.north);
	\draw[myline,black] (zasilacz.south) -- (naped.north);	
	
	\draw[myline,yellow] (plc.south) -- (io.north);	
	\draw[myline,yellow] (io.east) -- (naped.west);	

	\draw[myline, green, bend right=10] (naped.south) to (silnik1.north);
	\draw[myline, orange, bend left=10] (naped.south) to (silnik1.north);	
	\draw[myline, green, bend right=10] (naped.south) to (silnik2.north);
	\draw[myline, orange, bend left=10] (naped.south) to (silnik2.north);	
	%\draw[<->, >=latex', shorten >=2pt, shorten <=2pt, bend right=45, thick, dashed] 
    %(io.south) to node[auto, swap] {Competition}(naped.south); 
    
    \draw [yellow, line width=6] (6,-2) -- (6.5,-2); \node at (7.5,-2) {EtherCAT};
    \draw [blue, line width=6] (6,-2.5) -- (6.5,-2.5); \node at (7.5,-2.5) {Ethernet};
    \draw [black, line width=6] (6,-3) -- (6.5,-3); \node at (7.5,-3) {Zasilanie};
    \draw [green, line width=6] (6,-3.5) -- (6.25,-3.5); \draw [orange, line width=6] (6.25,-3.5) -- (6.5,-3.5); 
    \node [text width=2cm] at (7.75,-3.75) {Sterowanie silnikiem};    
\end{tikzpicture} 
\caption{Schemat stanowiska typu CP}
\label{stanowisko:cp}
\end{figure}