Głównym celem pracy było zaprojektowanie i~zrealizowanie stanowiska laboratoryjnego do~badania zależności czasowych w protokole EtherCAT. Wykorzystać należało dostępny na~uczelni w~laboratorium sprzęt i~przedstawić na nim obrazowo, że w protokole tym występują opóźnienia. Dla ich zobrazowania należało wykorzystać dostępne na stanowiskach silniki oraz przygotować przy użyciu dostępnych narzędzi system wizualizacji.

Prawdopodobnie po zakończeniu projektu posłuży on jako podstawa do stworzenia i~przeprowadzenia ćwiczeń laboratoryjnych prowadzonych w~ramach działalności dydaktycznej prowadzonej przez pracowników Zespołu Przemysłowych Zastosowań Informatyki.
%
%Model robota posiada 4~silniki umożliwiajace mu poruszania się w 3~osiach oraz poruszanie chwytakiem. W ramach pracy powstał kod sterowania pojedynczym silnikiem, który po gruntownym testowaniu podlegał dalszemu rozwojowi. Wszystkie silniki poprzez odpowiednie wysterowanie przez sterownik mogą pracować w trybach: automatycznym, ręcznym z dołączonego do sterownika zadajnika sygnałów dyskretnych oraz ręcznym z poziomu stworzonej wizualizacji. Tak przygotowane oprogramowanie zostało wykorzystane do stworzenia kodu umożliwiającego obsługę zbudowanego przez autora modelu magazynu z wykorzystaniem kolejki zadań do realizacji. Każde zadanie składa się z indeksu komórki w magazynie oraz zmiennej mówiącej o ruchu chwytaka (zabranie lub upuszczenie przedmiotu). Wszystkie operacje wykonywane na magazynie są wykonywane z wykorzystaniem zaimplementowanej kolejki FIFO od momentu kliknięcia przycisku w wizualizacji aż do jej opróżnienia.
%
%Podczas realizacji zostały rozwiązane problemy, z których najważniejszym zdaniem autora była bezwładność silników. W jej wyniku silnik po wyłączeniu dalej poruszał się przez nieokreślony czas aż do samoistnego zatrzymania. Wyeliminowanie negatywnych efektów tego zjawiska pozwoliło stworzyć w pełni funkcjonalne oprogramowanie spełniające stawiane przed nim zadania.
%W~procesie tworzenia oprogramowania autor poruszył i wykorzystał do realizacji wiele zagadnień typowych dla sterowników firmy Siemens. Przykładowo bloki wywoływane podczas startu sterownika (OB100 / OB101 / OB102) z~czego na sterowniku S7-300 dostępnym w~laboratorium dostępny jest tylko blok OB100, czyli tzw. gorący restart (ang. \emph{warm restart}) wywoływany przy każdym przejściu ze~stanu STOP do RUN/RUN-P oraz podczas uruchomienia po zaniku zasilania. Kolejne charakterystyczne bloki to~te~wywoływane jako cykliczne przerwania (OB30~do~OB38). Tutaj niestety również występuje ograniczenie i~dostępny jest jedynie blok OB35, wywoływany domyślnie co 100~ms.

Na końcu dokument zawiera podsumowanie oraz wnioski wyciągnięte z przeprowadzonych badań. Przedstawione zostały również liczne możliwe do~zrealizowania i~przeprowadzenia w przyszłości eksperymenty. Pozwolą one prawdopodobnie na~pogłębienie wiedzy o~protokole jak również sprawdzenie opóźnień czasowych występujących w protokole w~inny sposób.
\clearpage