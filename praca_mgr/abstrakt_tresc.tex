Głównym celem pracy było zaprojektowanie i~zrealizowanie stanowiska laboratoryjnego do~zbadania zależności czasowych w protokole EtherCAT. Należało wykorzystać dostępny na~uczelni w~laboratorium sprzęt i~przedstawić na nim obrazowo, że w protokole tym występują opóźnienia. 
%Dla ich pokazania należało wykorzystać dostępne na stanowiskach silniki oraz przygotować przy użyciu dostępnych narzędzi system wizualizacji.
Podstawowym zadaniem było stworzenie konfiguracji, oprogramowania sterownika swobodnie programowalnego oraz wizualizacji wykorzystując dostępne silniki i~pozostałe elementy stanowiska do pokazania zależności czasowych w protokole EtherCAT.

W~ramach pracy oprócz projektu i wykonania stanowiska, wykonano badania dotyczące opóźnień w protokole EtherCAT i~ zaproponowano rodzaje eksperymentów badawczych możliwych do zrealizowania na stanowisku.
Po zakończeniu projektu posłuży on jako podstawa do stworzenia i~przeprowadzenia ćwiczeń laboratoryjnych prowadzonych w~ramach działalności dydaktycznej pracowników Zespołu Przemysłowych Zastosowań Informatyki.

Określono i objaśniono dziedzinę, z~której wywodzi się temat. Praca zdaniem autora stanowi dobrą podstawę do poznania i~zrozumienia zasad działania protokołu EtherCAT. Opisane zostały różne metody realizacji urządzeń pracujących w~tym standardzie. Przeprowadzone badania udowodniły, że protokół jest bardzo nowoczesny i~spełnia wszystkie wymagania stawiane protokołom przemysłowym.

W końcowej części dokument zawiera podsumowanie oraz wnioski wyciągnięte z przeprowadzonych badań. Podczas realizacji zostały rozwiązane problemy charakterystyczne dla produktów firmy Beckhoff, z których najważniejszym zdaniem autora było skonfigurowanie i~poprawne uruchomienie silników. Przedstawione zostały również liczne możliwe do~zrealizowania i~przeprowadzenia w przyszłości eksperymenty. Pozwolą one na~pogłębienie wiedzy o~protokole, jak również sprawdzenie opóźnień czasowych występujących w protokole w~inny sposób.
\clearpage