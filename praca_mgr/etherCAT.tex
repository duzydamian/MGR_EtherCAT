\subsection{EtherCAT}
%zasada działania
%topologia
%źródła opóźnień
%wady/zalety
%itd.
EtherCAT jest nowoczesnym protokołem sieciowym przeznaczonym do stosowania w~aplikacjach przemysłowych, szczególnie takich, które wymagają działania całego systemu w~czasie rzeczywistym. Nazwa standardu jest skrótem od hasła: „Ethernet for Control Automation Technology”. W~zakresie warstwy fizycznej bazuje na~Ethernecie. Dodatkowo zaimplementowano w!nim mechanizmy w zakresie organizacji transmisji danych pozwalające na!ominięcie głównych ograniczeń sieci Ethernet. Dzięki temu EtherCAT jest obecnie jednym z~popularniejszych oraz szybciej rozwijających się protokołów komunikacyjnych w przemyśle.

\subsubsection{Determinizm, przepustowość}

\subsubsection{Przetwarzanie ,,w locie''}
Zamiast dzielić czas transmisji na dane priorytetowe i mniej ważne lub nadawać priorytety wszystkim pakietom, skorzystano z faktu, że większość danych i rozkazów transmitowanych przez urządzenia przemysłowe wymagające krótkich czasów reakcji jest bardzo mała. W praktyce rozmiar porcji informacji wymienianych jednorazowo przez pojedyncze przyrządy podłączone do sieci jest nawet mniejszy niż minimalny rozmiar ramki stosowanej w klasycznym Ethernecie. Korzystając z tego faktu, twórcy standardu EtherCAT potraktowali ramkę ethernetową jak pewnego rodzaju pamięć RAM, do której zapisywane i z której odczytywane są dowolne informacje w oparciu o adresy lokalizujące pożądane dane w tej pamięci.
Kolejne ramki nadawane są przez urządzenie pełniące rolę kontrolera i przesyłane do najbliższego urządzenia podrzędnego. Urządzenie to ma przydzielony adres, który jednoznacznie identyfikuje pewien fragment ramki (zwany datagramem), dzięki czemu może odczytać przeznaczone dla siebie dane. W przypadku gdy urządzenie podrzędne samo chce zainicjować komunikację, zapisuje pod odpowiednim adresem w odebranej ramce informacje przeznaczone dla innego urządzenia podłączonego do sieci EtherCAT. Niezależnie od tego, czy urządzenie coś zapisało, czy też tylko odczytywało fragment ramki, przesyła ją dalej do kolejnego z urządzeń. Duża szybkość tej metody transmisji wynika m.in. z tego, że nie ma potrzeby dekodowania całej ramki danych ani enkapsulacji w protokoły TCP/IP danych zapisywanych. Pozwala to błyskawicznie przekazywać ramki pomiędzy urządzeniami.

(ang. Pooling Timeslicing), (ang. Broadcast Master/Slave)

Wydajność tego rozwiązania jest dosyć duża, choć zależy od liczby elementów podłączonych do sieci. W praktyce czas opóźnienia nie wzrasta powyżej 1 ms i zazwyczaj jest znacznie (kilku- lub kilkunastokrotnie) krótszy. Czas synchronizacji nie przekracza 1 $\mu s$.

\subsubsection{Transmisja i synchronizacja w sieciach EtherCAT}

Jako medium komunikacyjne w~sieciach EtherCAT można wykorzystać kable miedziane (100Base-TX), światłowody (100Base-FX) lub łącze E-bus w~technologii LVDS. Te~ostatnie wprowadzono ze~względu na~to, że~transmisja w~sieciach EtherCAT często jest realizowana na~krótkich dystansach - E-bus sprawdza się w~realizacji łączności na~odległość do~około 10 m. Kable miedziane sprawdzają~się na~większych odległościach nieprzekraczających 100 metrów, natomiast użyteczna długość światłowodów może dochodzić aż do 20 kilometrów.

\subsubsection{Telegram EtherCAT}
Jako pokazano na Rysunku~\ref{etherCAT:ramka}, telegram EtherCAT jest inkapsulowany w  ramce Ethernet i  zawiera jeden lub więcej datagramów EtherCAT dostarczanych do urządzeń slave. 

Każdy datagram EtherCAT jest komendą, która zawiera zgodnie z Rysunkiem~\ref{etherCAT:datagram} nagłówek, dane i licznik roboczy. Nagłówek i dane są używane do wyspecyfikowania operacji, które muszą być wykonane przez slave, natomiast licznik roboczy jest aktualizowany przez slave informując mastera, że slave odebrał i przetwarza komendę.

\subsubsection{Synchronizacja}


\subsubsection{EtherCAT Technology Group}
\begin{figure}[!htb] 	\centering 	\includegraphics[width=0.3\textwidth]{images/logoETG} \caption{Logo EtherCAT Technology Group (http://www.ethercat.org).} \label{logoETG} \end{figure}

Standard EtherCAT został opracowany w~2003 roku przez Beckhoff Automation, niemiecką firmę z~branży automatyki przemysłowej. Następnie powołano organizację EtherCAT Technology Group (ETG), która zajęła~się standaryzacją tego protokołu. Stowarzyszenie~to obecnie zajmuje~się też organizowaniem szkoleń oraz popularyzacją tego standardu. 

Aktualnie w~skład ETG wchodzi ponad 2480 firm (dane na dzień 1~września~2013). Najważniejszym członkiem organizacji jest oczywiście firma BECKHOFF Automation. Pozostałe duże i~znane firmy wchodzące w~jej skład~to między innymi: ABB, Brother Industries, BMW Group, Częstochowa University of Technology, Epson, FANUC, Festo, GE Intelligent Platforms, Hitachi, Hochschule Ingolstadt, Mitsubishi, Microchip Technology, Mentor Graphics, Nikon, National Instruments, OLYMPUS, Panasonic, Rzeszów University of Technology, Red Bull Technology, Samsung Electronics, TRW Automotive, Volvo Group, Volkswagen oraz Xilinx.

Jak widać na powyższej liście w~skład organizacji wchodzą firmy z~bardzo wielu branż, a~nawet ośrodki naukowe. Autor pracy wybrał duże i~dobrze znane sobie firmy, aby pokazać jak wiele firm interesuje~się rozwojem przemysłowych protokołów komunikacyjnych.

\subsubsection{}
\subsubsection{}
\begin{figure}[htbp]
 \centering
        \tikzstyle{background grid}=[draw, black!50,step=.25cm]
	\begin{tikzpicture}[node distance=1cm, auto]%, show background grid]
	\tikzset{
    	mynode/.style={rectangle,rounded corners,draw=black, top color=white, bottom color=orange!50,very thick, inner sep=0.6em, minimum size=2.5em, 		text centered, text width=2.5cm},
    	mynodemini/.style={rectangle,rounded corners,draw=black, top color=white, bottom color=orange!50,very thick, inner sep=.5em, text centered},    	
	    myarrow/.style={->, >=latex', shorten >=1pt, ultra thick},
	    myline/.style={-, =latex', shorten >=1pt, rounded corners, ultra thick},
	    mylabel/.style={text width=7em, text centered} 
	} 
	\node[mynode] (master) {Węzeł nadrzędny \\ (ang. \textit{master})};  
	\node[mynode, below right=of master] (slave1) {Węzeł podrzędny 1 \\ (ang. \textit{slave})};
	\node[mynode, right=of slave1] (slave2) {Węzeł podrzędny 2 \\ (ang. \textit{slave})};
	\node[mynode, right=of slave2] (slaven) {Węzeł podrzędny n \\ (ang. \textit{slave})};  	 	 		
	
	\draw[myarrow,black] (master.350) -| (slave1.135);
	\draw[myarrow,black] (slave1.45) -- ++(0,1.5) -| (slave2.135);
	\draw[myarrow,black,dotted] (slave2.45) -- ++(0,1.5) -| (slaven.135);	
	\draw[myarrow,black] (slaven.45) |- (master.20);	
 
\end{tikzpicture} 
\caption{Przykładowa topologia sieci}
\label{etherCAT:topologia}
\end{figure} %
\begin{figure}[htbp]
 \centering
        \tikzstyle{background grid}=[draw, black!50,step=.25cm]
	\begin{tikzpicture}[node distance=2mm, auto]%, show background grid]
	\tikzset{
    	mynode/.style={rectangle,rounded corners,draw=black, top color=white, very thick, inner sep=4mm, 		text centered,font=\footnotesize},
    	mynodemini/.style={rectangle,rounded corners,draw=black, top color=white, thick, inner sep=2mm, text centered,font=\scriptsize},    	
	    myarrow/.style={->, >=latex', shorten >=1pt, ultra thick},
	    myline/.style={-, =latex', shorten >=1pt, rounded corners, ultra thick},
	    mylabel/.style={text centered, font=\scriptsize\bfseries} 
	} 
	\node[bottom color=gray!50, mynode] (ethhdr) {Ethernet header};  
	\node[bottom color=gray!50, mynodemini, below=of ethhdr.205] (da) {DA};
	\node[mylabel, below=1mm of da] (das) {6B};	
	\node[bottom color=gray!50, mynodemini, right=of da] (sa) {SA};  
	\node[mylabel, below=1mm of sa] (sas) {6B};	
	\node[bottom color=gray!50, mynodemini, right=of sa] (typ) {Typ};	
	\node[mylabel, below=1mm of typ] (typs) {2/4B};	
	\node[mylabel, below=of sas] (ethhdradr) {EtherType 0x88A4};
	
	\node[bottom color=yellow!50, mynode, right=of ethhdr, text width=2cm] (ecat) {EtherCAT};
	\node[bottom color=yellow!50, mynodemini, below=of ecat, text width=2.4cm] (ecathdr) {EtherCAT header};
	\node[mylabel, below=1mm of ecathdr] (ecathdrs) {2B};
	
	\node[bottom color=yellow!50, mynode, right=of ecat, text width=6.3cm] (ecatt) {EtherCAT telegram};
	\node[bottom color=yellow!50, mynodemini, below=of ecatt.193] (ecatd1) {Datagram 1};  	 
	\node[mylabel, below=1mm of ecatd1] (ecatd1s) {(10+n+2)B};
	\node[bottom color=yellow!50, mynodemini, right=of ecatd1] (ecatd2) {Datagram 2};
	\node[mylabel, below=1mm of ecatd2] (ecatd2s) {(10+m+2)B};  	 	 		
	\node[bottom color=yellow!50, mynodemini, right=0.9cm of ecatd2] (ecatdn) {Datagram n};
	\node[mylabel, below=1mm of ecatdn] (ecatdns) {(10+k+2)B};
	\node [fit=(ecatd1s) (ecatdns) (ecatdns)] (fit) {};  
 	%\draw [decorate, xshift=-20pt,line width=4pt] (fit.south east) -- (fit.north east);
	\draw [decorate,decoration={brace,amplitude=10pt}, line width=1pt] (fit.south east) ++(-0.3,0.3) -- ++(-6.9,0) (fit.south west);		
	\node[mylabel, below=1mm of fit] (fits) {44--1498B};
			
	\node[bottom color=gray!50, mynode, right=of ecatt] (eth) {Ethernet};
	\node[bottom color=gray!50, mynodemini, below=of eth.222] (pad) {Pad.};
	\node[mylabel, below=1mm of pad] (pads) {0--32B};
	\node[bottom color=gray!50, mynodemini, right=of pad] (fcs) {FCS}; 	 
	\node[mylabel, below=1mm of fcs] (fcss) {4B}; 		
 
	\draw[myline,black,dotted] (ecatd2) -- (ecatdn); 	
	
	\node[mylabel, below=of ethhdradr] (dal) {DA -- Destination Address};
	\node[mylabel, right=of dal] (sal) {SA -- Source Address};
	\node[mylabel, right=of sal] (padl) {Pad. -- Payload};
	\node[mylabel, right=of padl] (fcsl) {FCS -- Frame Check Sequance (CRC)};			
	
\end{tikzpicture} 
\caption{Ramka w transmisji EtherCAT i jej podział na datagramy}
\label{etherCAT:ramka}
\end{figure} %
\begin{figure}[htbp]
 \centering
        \tikzstyle{background grid}=[draw, black!50,step=.25cm]
	\begin{tikzpicture}[node distance=2mm, auto]%, show background grid]
	\tikzset{
    	mynode/.style={rectangle,rounded corners,draw=black, top color=white, very thick, inner sep=4mm, 		text centered,font=\footnotesize},
    	mynodemini/.style={rectangle,rounded corners,draw=black, top color=white, thick, inner sep=2mm, text centered,font=\scriptsize},    	
	    myarrow/.style={->, >=latex', shorten >=1pt, ultra thick},
	    myline/.style={-, =latex', shorten >=1pt, rounded corners, ultra thick},
	    mylabel/.style={text centered, font=\scriptsize\bfseries} 
	} 
	\node[bottom color=gray!50, mynode] (datagram) {Datagram};  
	\node[bottom color=gray!50, mynode, below=1cm of datagram] (part1) {Dane};   		
	\node[bottom color=gray!50, mynode, left=of part1] (part2) {Nagłówek};   		
	\node[bottom color=gray!50, mynode, right=of part1] (part3) {Licznik};   				
 
	\draw[myline,black,dotted] (datagram.south west) -- (part2.north west); 	
	\draw[myline,black,dotted] (datagram.south east) -- (part3.north east); 
\end{tikzpicture} 
\caption{Budowa datagramu}
\label{etherCAT:ramka}
\end{figure} %
\input{tikz/wykorzystanie} %