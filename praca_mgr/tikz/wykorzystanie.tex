\begin{figure}[htbp]
\centering
\begin{tikzpicture}[scale=1.2]
\begin{axis}[
		x=3cm,
		ybar stacked, bar width=\baselineskip,
%		grid=major,
		ymin=0,ymax=110, 
		ylabel={Wykorzystanie [\%]},
		yticklabels={0, 20, 40, 60, 80, 100},
		xtick=data,
		xticklabel style={text centered, text width=2.5cm},
	    xticklabels={Odpytywanie w~przedziale czasowym,Nadawanie Master/Slave,EtherCAT},
	    enlarge x limits=0.2,
		every node near coord/.style={font=\tiny},
	    ]  
	    
\addplot[fill=gray!50] coordinates
{(1,2) (2,20) (3,0)};
\addplot[fill=yellow!50] coordinates
{(1,0) (2,0) (3,80)};
\addplot[fill=gray!80] coordinates
{(1,3) (2,10) (3,0)};
\addplot[fill=yellow!80] coordinates
{(1,0) (2,0) (3,17)};

\node[above] at ($(axis cs:1,5)$) {2-5\%};
\node[above] at ($(axis cs:2,30)$) {20-30\%};
\node[above] at ($(axis cs:3,96)$) {90-97\%};
\end{axis}

\end{tikzpicture}
\caption{Współczynnik wykorzystania kanału transmisyjnego \\ w Ethernecie (dwa pierwsze wykresy od lewej strony) i EtherCAT.}
\label{etherCAT:wykorzystanie}
\end{figure}