\begin{figure}[htbp]
 \centering
        \tikzstyle{background grid}=[draw, black!50,step=.25cm]
	\begin{tikzpicture}[node distance=1cm, auto]%, show background grid]
	\tikzset{
    	mynode/.style={rectangle,rounded corners,draw=black, fill=red!15,very thick, inner sep=1.2em, minimum size=2.5em, 		text centered, text width=2.5cm},
    	mynodemini/.style={rectangle,rounded corners,draw=black, fill=red!15,very thick, inner sep=.5em, text centered, text width=2.5cm},    	
	    myarrow/.style={<->, >=latex', shorten >=1pt, ultra thick},
	    myline/.style={-, =latex', shorten >=1pt, rounded corners, ultra thick},
	    mylabel/.style={text width=7em, text centered} 
	} 
	\node[mynode] (asic_fpga) {ASIC/FPGA \\ EtherCAT};  
	\node[mylabel, left=of asic_fpga] (io) {Cyfrowe \\ wejścia/wyjścia};
	\node[mynodemini, right=of asic_fpga.north east] (layer1) {Warstwa \\ fizyczna};
	\node[right=of layer1] (empty1) {};
	\node[mynodemini, right=of asic_fpga.south east] (layer2) {Warstwa \\ fizyczna};  	 	 		
	\node[right=of layer2] (empty2) {};	

%	\draw[myarrow] (io) -- (asic_fpga);	
    \foreach \i in {-2,...,2}{% 
      \draw[myarrow] ([yshift=\i * 0.4 cm]io.east) -- ([yshift=\i * 0.4 cm]asic_fpga.west) ;}
	
	\draw[myarrow] (asic_fpga.north east) -- (layer1);	
	\draw[myarrow] (asic_fpga.south east) -- (layer2);	
				
	\draw[myarrow] (layer1) -- (empty1);	
	\draw[myarrow] (layer2) -- (empty2);	
 
\end{tikzpicture} 
\caption{Budowa węzła EtherCAT z~wykorzystaniem pojedynczego układu FPGA lub ASIC.}
\label{etherCAT:FPGA_ASIC}
\end{figure}