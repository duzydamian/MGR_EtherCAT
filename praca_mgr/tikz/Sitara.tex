\begin{figure}[htbp]
 \centering
        \tikzstyle{background grid}=[draw, black!50,step=.25cm]
	\begin{tikzpicture}[node distance=1cm, auto]%, show background grid]
	\tikzset{
    	mynode/.style={rectangle,rounded corners,draw=black, fill=red!15 ,very thick, inner sep=1em, minimum size=2.5em, 		text centered, text width=2.5cm},
    	mynodemini/.style={rectangle,rounded corners,draw=black, fill=red!15,very thick, inner sep=.5em, text centered, text width=2.5cm},    	
	    myarrow/.style={<->, >=latex', shorten >=1pt, ultra thick},
	    myline/.style={-, =latex', shorten >=1pt, rounded corners, ultra thick},
	    mylabel/.style={text width=7em, text centered} 
	} 
	
	\node[mynode, fill=blue!15, fit={(asic_fpga) (proc.east)}] (mikro) {}; 
	
	\node[mynode] (asic_fpga) {Interfejs \\ EtherCAT};  
	\node[mylabel, left=1mm of asic_fpga] (proc) {Procesor};	
	
	\node[mynodemini, right=2cm of asic_fpga.north east] (layer1) {Warstwa \\ fizyczna};
	\node[right=of layer1] (empty1) {};
	\node[mynodemini, right=2cm of asic_fpga.south east] (layer2) {Warstwa \\ fizyczna};  	 	 		
	\node[right=of layer2] (empty2) {};	
	
	\draw[myarrow] (asic_fpga.north east) ++(0.5,0) -- (layer1);	
	\draw[myarrow] (asic_fpga.south east) ++(0.5,0) -- (layer2);	
				
	\draw[myarrow] (layer1) -- (empty1);	
	\draw[myarrow] (layer2) -- (empty2);	
 
\end{tikzpicture} 
\caption{Budowa mikrokontrolera Sitara AM335x wyposażonego w programowalną jednostkę czasu rzeczywistego.}
\label{etherCAT:Sitara}
\end{figure}