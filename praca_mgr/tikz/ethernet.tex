 \vspace{-1.1cm}
\begin{figure}[htbp]
 \centering
        \tikzstyle{background grid}=[draw, black!50,step=.25cm]
	\begin{tikzpicture}[node distance=1cm, auto]%, show background grid]
	\tikzset{
    	mynode/.style={rectangle,rounded corners,draw=black, fill=red!15,very thick, inner sep=1.2em, minimum size=2.5em, 		text centered, text width=2.5cm},
    	mynodemini/.style={rectangle,rounded corners,draw=black, fill=red!15,very thick, inner sep=.5em, text centered, text width=2.5cm},    	
	    myarrow/.style={->, >=latex', shorten >=1pt, ultra thick, blue},
	} 
	\node[mynode] (device2) {Urządzenie 2};  
	\node[mynode, fill=blue!15, left=5cm of device2] (device1) {Urządzenie 1};
	\node[right=0.1cm of device2] (loop) {};
	
	\draw[myarrow, red] (loop) to [out=290,in=70,looseness=20] (loop) node[right=0.7cm,black, text width=1.5cm] {Czas reakcji węzła};
	\draw[myarrow] ([yshift=5mm]device1.east) -- ([yshift=5mm]device2.west) node [midway,above,black] {4 bajty informacji};		
	\draw[myarrow] ([yshift=-5mm]device2.west) -- ([yshift=-5mm]device1.east) node [midway,below,black] {4 bajty odpowiedzi };		
\end{tikzpicture} 
 \vspace{-1.1cm}
\caption{Przykład transmisji małej ilości danych (4 bajty) zwykłym Ethernetem.}
\label{etherCAT:ethernet}
\end{figure}