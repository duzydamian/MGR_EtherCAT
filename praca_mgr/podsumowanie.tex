\section{Uruchamianie i testowanie}
W rozdziale zawarto podsumowanie przebiegu prac nad projektem. Opisane zostaną tu~problemy, które wystąpiły w~czasie realizacji projektu. Ponadto zawarto~tu opis przebiegu procesu testowania.

%\subsection{Przebieg testowania}
%W procesie weryfikacji poprawności działania projektu zastosowano testowanie wstępujące. 
%
%Głównym testerem był autor projektu więc większość testów przebiegała na zasadzie białej skrzynki (ang. \emph{white box}), bardzo często z~użyciem podglądu stanu w środowisku TwinCAT System Manager. Takie testowanie pozwala stosunkowo łatwo wyszukać źródło błędu i~je wyeliminować.
%
%Autor kilka razy przeprowadzał testy stosując metodę czarnej skrzynki (ang. \emph{black box}), nie~biorąc pod uwagę zależności wykonywanych czynności, od~realizowanego przez sterownik kodu. Kilkukrotnie w czasie realizacji projektu do testów zgłaszały się osoby trzecie, które były nim zaciekawione. Testy wykonane przez takie osoby są niezwykle cenne ze względu na dużą nieprzewidywalność oraz całkowitą niezależność działań od rozwiązań ze~względu na~brak ich znajomości.
%
%W~czasie realizacji autor stosował testowanie oparte na dwóch metodach analizy. Testowanie oprogramowania można wykonywać pod kątem analizy statycznej i~dynamicznej. Analiza statyczna polega na~sprawdzaniu kodu źródłowego i~znajdowaniu w~nim błędów bez uruchamiania sprawdzanego kodu. Ta~metoda była stosowana poza laboratorium, gdzie brak był dostępu do sterownika i~modelu. Podczas analizy dynamicznej oprogramowanie jest uruchamiane i~badane pod kątem ścieżki przebiegu i~czasu wykonywania. Ta metoda z~kolei była najważniejsza i~często wyniki tych testów były zaskakujące w~stosunku do~przeprowadzonych wcześniej z~zastosowaniem analizy statycznej.
%
%Ostatnim etapem testów były~te przeprowadzone w~obecności promotora oraz te wykonane przez niego. Ostatecznie oprogramowanie zostało zatwierdzone i~uznane za spełniające wszystkie wstępne założenia przedstawione w~podrozdziale 1.2.
%\newpage
\subsection{Napotkane problemy}
\label{subsec:problemy}
Podczas tworzenia projektu napotkane i~przeanalizowane zostały następujące problemy:
\begin{itemize}
\item Problem z automatycznym uruchamianiem stworzonego projektu PLC:\\[1mm]
Po utworzeniu oprogramowania sterownika~PLC~oraz po odpowiednim skonfigurowaniu go w oprogramowaniu TwinCAT System Manager tj. linkowaniu zmiennych programu do~odpowiednich fizycznych wejść oraz wyjść modelu oraz aktywowaniu tak przygotowanej konfiguracji, które z~kolei wymusza zresetowanie systemu nie~następuje uruchomienie projektu sterownika PLC. W początkowej fazie autor przełączał się na oprogramowanie TwinCAT PLC Control gdzie logował się do~sterownika i~ręcznie uruchamiał stworzony przez siebie kod. Niestety to~rozwiązanie na~dłuższą metę okazało~się niewystarczające i~czasochłonne. Okazało się, że~w~sterownikach firmy Beckhoff trzeba w~specjalny sposób przygotować oprogramowanie, które ma~być uruchamiane w~sposób automatyczny, tzn. trzeba utworzyć projekt, który jest bootowalny. Początkowo takie podejście wydało się autorowi bardzo dziwne, ale po~dłuższym zastanowieniu oraz kilku rozmowach z~bardziej doświadczonymi w~branży osobami okazało się, że~ma~ono swoje plusy. Przykładowo w~przypadku tworzenia oprogramowania w~fazie rozwojowej reset urządzenia pozwala przerwać całkowicie wykonywanie oprogramowania zawierającego błędy mające destrukcyjny wpływ na~model lub w~praktyce na~obiekt przemysłowy. Po~zastosowaniu nowej metody uruchamianie~i testowanie tworzonego oprogramowania stało~się zdecydowanie prostsze.

\item Problem z~wykryciem jednego z~silników:\\[1mm]
Postępując zgodnie z~informacjami znalezionymi w~odpowiednich materiałach szkoleniowych udało się wykryć w~środowisku silnik tylko na jednej osi. Autor przyjrzał się silnikom i znalazł na nich tabliczki znamionowe z~informacją, że oba silniki są identyczne. Rozwiązanie zostało znalezione dopiero po zapoznaniu się z~przykładową konfiguracjom stanowiska dostarczoną przez promotora. Okazało się, że drugi silnik(o innym niż pierwszy symbolu) był skonfigurowany na sztywno. Autor zastosował to rozwiązanie u siebie, co poskutkowało uruchomieniem drugiego silnika. Dokładniejsza analiza silników wykazała, że na jednym z~silników znajduje się druga tabliczka znamionowa z~innym symbolem. Tak więc wbrew wcześniejszym ustaleniom na stanowisku znajdują się dwa odmienne silniki różniące się od siebie podłączeniem do napędu oraz wyposażeniem wewnętrznym. Z~tego właśnie powodu jeden z~silników nie był wykrywany, ponieważ nie udostępnia on takiej możliwości.

\item Problem ze zbyt wysokim napięciem napędu serwomechanizmów:\\[1mm]
Pomimo skonfigurowania stanowiska zgodnie z~materiałami szkoleniowymi producenta \cite{kurs2, silniki} napędy serwomechanizmów AX5203 zgłaszały błąd zbyt wysokiego napięcia. Rozwiązanie znajdowało się w~jednym z wymienionych wcześniej materiałów szkoleniowych, a~mianowicie konfigurację parametrów napędu serwomechanizmów należało dodać do specjalnej 'Listy startowej' modułu. Po zastosowaniu tej opcji oraz ponownym uruchomieniu stanowiska napęd zwrócił informację, że obie osie są gotowe do działania.

\item Problem z~ruszeniem silników:\\[1mm]
Problem objawiał się tym, że pomimo wykluczenia dwóch poprzednich błędów silniki nie ruszały z miejsca. Silniki zgłaszały gotowość do pracy, ale mimo to ręczne wymuszenie ruchu silnika w~przód lub tył nie przynosiło efektu, co więcej w~środowisku pojawiała sie informacja, że silnik pracuje oraz porusza się w żądanym kierunku. Niestety informacje ze środowiska nie zgadzały się ze stanem fizycznym i~silnik stał w~miejscu.

\item Utrata komunikacji z~jednym ze sterowników:\\[1mm]
Przy pewnej modyfikacji konfiguracji stanowiska typu~CX związanej z~próbą uruchomienia silników została utracona komunikacja z urządzeniem. Precyzując, przestało ono odpowiadać na~poprzednim adresie. Sytuacja była o~tyle dziwna, że system na urządzeniu działał całkowicie normalnie. Po~podpięciu zewnętrznego monitora okazało się, że w systemie operacyjnym karty sieciowe są skonfigurowane prawidłowo oraz udaje się nawiązać połączenie i~uzyskać żądany adres (konfiguracja adresów jest statyczna). Próby wykorzystania programu ping nie~dały początkowo żadnego efektu, ponieważ urządzenie nie odpowiadało. W licznych próbach i~pomysłach udało się ustalić, że urządzenie podczas uruchamiania (dokładnie podczas uruchamiania systemu operacyjnego) odpowiada na kilka zapytań (od 4 do 6 w kilku próbach), podobnie w momencie zamykania systemu. To~odkrycie zasugerowała autorowi, że~coś blokuje, przekonfigurowywuje lub wywłaszcza urządzenia. Pierwszy został przeanalizowany autostart systemu Windows lecz okazał się on pusty. Pojawił się pomysł przywrócenia urządzenia do ustawień fabrycznych lecz nie~udało się odnaleźć takiej możliwości. Problem udało się rozwiązać uruchamiając w~systemie operacyjnym standardowy menadżer plików (w~tym przypadku explorer), odnajdując stworzone pliki konfiguracji TwinCAT i~usuwając~je. ($\backslash Hard Disk\backslash TwinCAT\backslash Boot\backslash$)

\end{itemize}
\indent
\indent Wszystkie problemy zostały rozwiązane i~w~ostatecznej wersji oprogramowania nie wpływają one w~negatywny sposób na~pracę modelu.
