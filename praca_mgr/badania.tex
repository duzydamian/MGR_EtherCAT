\section{Badania}
W~niniejszym rozdziale opisany został przebieg przeprowadzonych badań oraz analiza ich wyników. W~kolejnych podrozdziałach zostaną przedstawione kolejne różne eksperymenty. 

%\subsection{Opóźnienia pojedynczego odcinka sieci}
%\subsection{Wpływ topologi na opóźnienia}
\subsection{Czas stabilizacji sieci po zmianach}
Badanie miało na~celu sprawdzenie czy zgodnie z~zapewnieniami producenta faktycznie możliwe jest rozłączanie i~podłączanie urządzeń do~sieci.
\subsubsection{Pojedyncze urządzenie}
2,6s do 2,7s

\subsubsection{Wyspa z modułami I/O}
1,5s wyspa i każdy kolejny moduł I/O z opóźnieniem 4ms
\begin{figure}[htbp]
 \centering
 \begin{tikzpicture}[x=1cm,y=5cm]


 \draw[latex-latex, thin, draw=gray] (0,0)--(10,0) node [right] {$x$}; % l'axe des abscisses
 \draw[latex-latex, thin, draw=gray] (0,0)--(0,2) node [above] {$t[s]$}; % l'axe des ordonnées
 \draw[thick] (0,1.5)--(10,1.54); % l'axe des abscisses
  \draw[thick,red] (0,1.6)--(10,1.64); % l'axe des abscisses

    \foreach \i in {0,...,10}{% 
\foreach \Point in {(\i ,1.5+0.004*\i)}{
    \node at \Point {\textbullet}; } ;}     

    \foreach \i in {0,...,10}{% 
\foreach \Point in {(\i ,1.6+0.004*\i)}{
    \node[red] at \Point {\textbullet}; } ;}
    
% to ensure that the points are being properly centered:
\draw [dotted, gray] (0,0) grid (10,2);

\end{tikzpicture}
\caption{Pomiary czasu ponownego podłączenia oraz obliczona wartość średnia}
\label{badania:wykres:stabilizacja_wyspa}
\end{figure}

\subsection{Badania niewykonalne}

\subsubsection{Zbadanie innych topologii}

\subsubsection{Zbadanie opóźnień na poziomie transmisji pojedynczych ramek}
%Różne kable
%Długość kabla
%
%Połączyć do jednego sterownika oba napędy kolejno i zrobić coś na zasadzie inkrementacji i sprawdzić czy się przypadkiem nie rozjedzie
%
%Mamy opóźnienie na jednym odcinku
%
%Ewentualnie jeden kabel można zamienić na dłuższy i sprawdzić czy nie ma różnicy.
%
%Wymyślić jak sprawdzić czas ponownego włączenia do sieci.