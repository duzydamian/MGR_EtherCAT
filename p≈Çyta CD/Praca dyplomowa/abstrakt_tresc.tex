Głównym celem pracy było zaprojektowanie i~zrealizowanie stanowiska laboratoryjnego do~zbadania parametrów sieci EtherCAT z~wykorzystaniem sprzętu dostępnego w~laboratorium Zespołu Przemysłowych Zastosowań Informatyki wydziału Automatyki, Elektroniki i Informatyki Politechniki Śląskiej. Zbudowane stanowisko miało służyć prezentacji zależności czasowych w~tej sieci.

Podstawowym zadaniem było  skonfigurowanie i~oprogramowania sterownika swobodnie programowalnego oraz prostej wizualizacji. W~ramach pracy, oprócz projektu i wykonania stanowiska, wykonano badania dotyczące opóźnień w~sieci EtherCAT i~zaproponowano rodzaje eksperymentów badawczych możliwych do zrealizowania na stanowisku.
Stworzone stanowisko posłuży jako podstawa do przygotowania zajęć dydaktycznych prowadzonych przez pracowników Zespołu Przemysłowych Zastosowań Informatyki.

Praca zdaniem autora stanowi dobrą podstawę do poznania i~zrozumienia zasad działania protokołu EtherCAT. Zostały w niej ponadto opisane różne metody sprzętowej realizacji urządzeń pracujących w~tym standardzie. 
Badania przeprowadzone na wykonanym stanowisku udowodniły, że protokół jest bardzo nowoczesny i~spełnia wysokie wymagania stawiane protokołom przemysłowym.

Podczas realizacji zostały rozwiązane problemy charakterystyczne dla produktów firmy Beckhoff, z~których najważniejszym zdaniem autora było skonfigurowanie i~poprawne uruchomienie silników. Przedstawione zostały również liczne możliwe do~zrealizowania i~przeprowadzenia w przyszłości eksperymenty, mogące posłużyć do pogłębienia wiedzy o~protokole, jak również sprawdzenia opóźnień czasowych występujących w protokole w~inny sposób niż wykorzystany w~ramach pracy.
\clearpage