\lstset{language=VBScript,
        basicstyle=\footnotesize\ttfamily,
        breaklines=true,
        tabsize=2,
        numbers=left,
        numberstyle=\tiny,
        numbersep=7pt,
        showspaces=false,
        keywordstyle=\color{Blue}\textbf,
        commentstyle=\color{Red}\emph,
        showstringspaces=false,
        stringstyle=\color{BurntOrange}
        }
\section{Wizualizacja HMI}
Zaimplementowana przez autora wizualizacja ma na~celu zobrazowanie działania modelu oraz umożliwienie operatorowi wpływania na~jego działanie. Kolejne podrozdziały zawierają opis specyfikacji zewnętrznej oraz wewnętrznej. Część odnosząca się do specyfikacji zewnętrznej jest skróconą instrukcją obsługi użytkownika. Specyfikacja wewnętrzna jest opisem, jak zostały zrealizowane poszczególne elementy~i~w jaki sposób wizualizacja współpracuje ze~sterownikiem.

\subsection{Specyfikacja zewnętrzna}
Specyfikacja zewnętrzna przedstawiona w dalszej części podrozdziału stanowi skróconą instrukcję obsługi wizualizacji oraz opis możliwości oferowanych przez poszczególne ekrany.

Autor projektu wykorzystał w~swojej pracy szereg elementów dostępnych standardowo w~środowisku Simatic WinCC flexible. Podstawowymi elementami sterującymi są przyciski w~trybie tekstowym oraz przeźroczystym. Głównymi obiektami służącymi do prezentacji informacji są: pola tekstowe, pola wejściowo-wyjściowe oraz pola daty i~godziny. Dodatkowo celem uatrakcyjnienia wizualizacji wykorzystane zostały suwaki (ang. \emph{slider}), obrazki oraz zegarek. 

Obsługa tej części projektu jest realizowana za pomocą myszy i~klawiatury podłączonych do~komputera. Za~pomocą klawiatury wybieramy interesujący nas ekran lub wprowadzamy żądaną wartość pozycji docelowej na~ekranie testowania trybu automatycznego.

\subsection{Specyfikacja wewnętrzna}
Wizualizacja komunikuje się z~komputera klasy~PC ze~sterownikiem za~pośrednictwem protokołu Ethernet w~sieci lokalnej.
Odniesienia do~odpowiednich adresów w~pamięci sterownika dokonywane są za~pomocą nazw symbolicznych zdefiniowanych w~tablicy Tags. Do działania wizualizacja używa tylko jednej zmiennej wewnętrznej i~jest~to zmienna tablicowa \emph{MagazynDTEnable} z elementami typu bool. Elementy te odpowiadają za~wyświetlanie dat oraz godzin na~ekranie ze~stanem magazynu po~kliknięciu na~wybraną komórkę. Obsługa wyświetlania dat polega na tym, że po kliknięciu w wybrane pole ustawiana jest odpowiednia zmienna w~tej tablicy na~wartośc \emph{true}, a~po zwolnieniu klawisza myszki na wartość \emph{false}. Za zmiany te odpowiadają niewidzialne przyciski umieszczone na~tych polach.

Wizualizacja wpływa na pracę sterownika poprzez zmianę pojedynczych bitów za~pomocą umieszczonych na~ekranie przycisków. Wpływa ona również poprzez modyfikowanie wybranych zmiennych odpowiadających pozycjom docelowym lub poprzez dodawanie odpowiednich zadań do~kolejki. Bardziej zaawansowane operacje zostały zrealizowane za~pomocą skryptów napisanych w~języku VBScript, które są bardzo prostą i~szybką opcją wykonywania bardziej zaawansowanych czynności. 