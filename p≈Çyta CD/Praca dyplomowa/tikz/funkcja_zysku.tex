\begin{figure}[htb]
 \centering
 \tikzset{
%Define standard arrow tip
>=stealth',
%Define style for different line styles
help lines/.style={dashed, thick},
axis/.style={<->},
important line/.style={thick},
connection/.style={thick, dotted},
}
\newcommand\A{\ensuremath{\mathcal{A}}}
\newcommand\B{\ensuremath{\mathcal{B}}}
	\subfloat[System typu ,,hard'']{
		\label{wstęp:funkcja_zysku:1}	
		\begin{tikzpicture}

		% horizontal axis
		\draw[->] (-0.2,0) -- (5,0) node[anchor=north] {$t$};
		% labels
		\draw	(1.1,0) node[anchor=north east] {$t_0$}
				(3,0) node[anchor=north east] {$t_T$};
		
		% vertical axis
		\draw[->] (0,-1) -- (0,3) node[anchor=east] {$z(t)$};
		% labels
		\draw	(0.1,1.9) node[anchor=south east] {$u$}
				(0,-0.1) node[anchor=south east] {$0$}		
				(0,-1.2) node[anchor=south east] {$-\infty$};
						
		% ciągłe
		\draw[ultra thick] (1,2) -- (3,2);
		
		% Przerywane
		\draw[thick,dashed] (-0.2,2) -- (1,2);
		\draw[thick,dashed] (1,-0.2) -- (1,2);
		\draw[ultra thick,dashed, <-] (3,-0.9) -- (3,2);		
		
		\draw (8,1.5) node {
		$z(t)=\left\{
			\begin{array}{c l}     
			    u & t_0<t<t_T\\
		       -\infty & t \geq t_T
			\end{array}\right.$ };
					
		\end{tikzpicture}	
	}                
	
	\subfloat[System typu ,,firm'']{ 
		\label{wstęp:funkcja_zysku:2}
		\begin{tikzpicture}

		% horizontal axis
		\draw[->] (-0.2,0) -- (5,0) node[anchor=north] {$t$};
		% labels
		\draw	(1.1,0) node[anchor=north east] {$t_0$}
				(3,0) node[anchor=north east] {$t_T$};
		
		% vertical axis
		\draw[->] (0,-0.5) -- (0,3) node[anchor=east] {$z(t)$};
		% labels
		\draw	(0.1,1.9) node[anchor=south east] {$u$}
				(0,-0.1) node[anchor=south east] {$0$};
						
		% ciągłe
		\draw[ultra thick] (1,2) -- (3,2);
		\draw[ultra thick, ->] (3,0) -- (5,0);		
		
		% Przerywane
		\draw[thick,dashed] (-0.2,2) -- (1,2);
		\draw[thick,dashed] (1,-0.2) -- (1,2);
		\draw[ultra thick,dashed] (3,0) -- (3,2);		
		
		\draw (8,1.5) node {
		$z(t)=\left\{
			\begin{array}{c l}     
			    u & t_0<t<t_T\\
		        0 & t \geq t_T
			\end{array}\right.$ };
			
		\end{tikzpicture}		
	}
	
	\subfloat[System typu ,,soft'']{
		\label{wstęp:funkcja_zysku:3}	
		\begin{tikzpicture}

		% horizontal axis
		\draw[->] (-0.2,0) -- (5,0) node[anchor=north] {$t$};
		% labels
		\draw	(1.1,0) node[anchor=north east] {$t_0$}
				(2.1,0) node[anchor=north east] {$t_1$}
				(3,0) node[anchor=north] {$t_T$};
		
		% vertical axis
		\draw[->] (0,-0.5) -- (0,3) node[anchor=east] {$z(t)$};
		% labels
		\draw	(0.1,1.9) node[anchor=south east] {$u$}
				(0,-0.1) node[anchor=south east] {$0$};
						
		% ciągłe
		\draw[ultra thick] (1,2) -- (2,2);
		\draw[ultra thick] (2,2) -- (3,0);		
		\draw[ultra thick, ->] (3,0) -- (5,0);		
		
		% Przerywane
		\draw[thick,dashed] (-0.2,2) -- (1,2);
		\draw[thick,dashed] (1,-0.2) -- (1,2);
		\draw[thick,dashed] (2,-0.2) -- (2,2);		

		\draw (8,1.5) node {
		$z(t)=\left\{
			\begin{array}{c l}     
			    u & t_0<t<t_1\\
			    u \frac{t_T-t}{t_T-t_1} & t_1<t<t_T\\
		        0 & t \geq t_T
			\end{array}\right.$ };
			
		\end{tikzpicture}
	}
\caption{Funkcja zysku dla systemów czasu rzeczywistego.}
\label{wstep:funkcja_zysku}
\end{figure}